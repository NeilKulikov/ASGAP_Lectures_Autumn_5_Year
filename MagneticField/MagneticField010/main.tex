%!TEX encoding = UTF-8 Unicode
\documentclass[a4paper, 14pt, russian]{article}
\usepackage[a4paper]{geometry}
\usepackage[T2A]{fontenc}
\usepackage[utf8]{inputenc}
\usepackage[russian]{babel}
\usepackage{physics}
\usepackage{tcolorbox}
\usepackage{hyperref}
\usepackage{fancyhdr}
\usepackage{indentfirst}
\usepackage{amssymb, amsmath, amsfonts}

\title{Физика магнитных явлений}
\author{Иосиф Давидович Токман}
\date{}

\newcommand{\be}{\begin{equation}}
\newcommand{\ee}{\end{equation}}
\newcommand{\bea}{\begin{equation}\begin{array}}
\newcommand{\eea}{\end{array}\end{equation}}

\newcommand{\pa}{\partial}
\newcommand{\rot}{\textbf{rot}~}
\renewcommand{\div}{\textbf{div}~}
\renewcommand{\grad}{\textbf{grad}~}

\setcounter{section}{14}

\begin{document}
	\maketitle

	Рассмотрим 2-е сферические частицы одинакового радиуса $R$. Приготовленные из
	ферромагнетика с изотропией "лёгкая ось". В отличие от предыдущего случая
	будем считать, что $K$ - большая величина.

	Рассмотрим 2-е ситуации:
	\begin{itemize}
		\item Однородная намагниченность
		\item Частица разделена на 2-а одинаковых домена 
			(с разным направлением намагниченности) с плоской доменной стенкой.
	\end{itemize}

	В первом случае есть только магнитостатическая энергия. Тогда:
	\be
		\label{eq170}
		E_1 = - \frac{1}{2} M_0 (- \frac{4\pi}{3} M_0) \frac{4\pi}{3} R^3 
			= \frac{8}{9} \pi^2 M_0^2 R^3;
	\ee

	Здесь второй множитель - собственное поле частицы.

	А во втором случае у нас есть ещё и энергия стенки (блоховской).
	Второй член здесь отвечает стенке.
	\be
		\label{eq171}
		E_2 \approx \frac{1}{2} \pi^2 M_0^2 R^3 + \pi R^2 \sigma_\text{wall}
			\approx \frac{4}{9} \pi^2 M_0^2 R^3 + \pi^2 \sqrt{\abs{K} A} R^2;
	\ee

	Двухдоменная частица становится неустойчивой, при $E_1 < E_2$.
	Критический радиус:
	\be
		\label{eq172}
		R < R_\text{cr} \approx \frac{9}{4} \sqrt{\frac{A \abs{K}}{M_0^4}}
			\sim \frac{\lambda_2^2}{\lambda_1};
	\ee

	Это был иллюстрационный раздел.

	\section{Ферромагнетик во внешнем квазистатическом поле.}

	Если ферромагнетик помещен во внешнее поле $\vec{\mathcal H}_0$. 
	То к его энергии должно быть добавлено слагаемое:
	\be
		\label{eq173}
		- \int \vec{\mathcal H}_0 \vec M dv;
	\ee

	В таком случае рассмотрим поведение однородно намагниченного ферромагнетика во внешнем
	однородном стационароном магнитном поле. Пусть это будет малая однодоменная
	ферромагнитная частица в форме эллипсоида вращения, вытяннутого вдоль оси $z$.
	Из соображений симметрии видно6 что вектор $\vec M$ лежит в плоскости, проходящей через
	ось $z$ и вектор $\vec{\mathcal H}_0$. Пусть это будет плоскость $zx$.

	Вследствии однодоменности энергия частицы складывается из её собственной магнитостатической
	энергии и магнитостатической энергии во внещнем поле. Мы не учитываем обменную энергию,
	потому что намагниченность однородная.
	\be
		\label{eq174}
		E = -\frac{1}{2} M_z {\mathcal H}_{M, z} - \frac{1}{2} M_x {\mathcal H}_{M, x}
			- M_z \mathcal{H}_{0,z} - M_x \mathcal{H}_{0,x};
	\ee

	С точностью до несущественного постоянного члена можно записать:
	\be
		\label{eq175}
		E = - \frac{\beta M_0^2}{2} (\cos \theta)^2 - M_0 \mathcal{H}_{0,z} \cos \theta
			- M_0 \mathcal{H}_{0,x} \sin \theta;
	\ee

	Здесь $\beta = 4\pi (N_{xx} - N_{zz}) > 0$, поскольку $N_{zz} < N_{xx}$.
	Несколько упростим задачу: $\mathcal{H}_{0,x} = 0,~ \mathcal{H}_{0,z} > 0$.
	\be
		E = - \frac{\beta M_0^2}{2} (\cos \theta)^2 - M_0 \mathcal{H}_{0,z} \cos \theta;
	\ee  

	Тогда попробуем найти минимум по $\theta$:
	\be
		\left.\frac{dE}{d\theta}\right\rvert_{\theta  = \theta_p} = 0;
	\ee

	Решение:
	\be
		\sin \theta_p (\beta M_0^2 \cos \theta_p + M_0 \mathcal{H}_{0,z}) = 0;
	\ee

	В общем случае есть 3-и решения:
	\be
		\theta_1 = 0,~\theta_2 = \pi,~ 
			\cos \theta_3 = - \frac{\mathcal{H}_{0,z}}{\beta M_0};
	\ee

	Посмотрим какие состояния из этих устойчивы. Равновесие устойчиво, если:
	\be
		\frac{d^2 E}{d\theta^2} = - \beta M_0^2 \sin^2 \theta_p  + 
			\beta M_062 \cos^2 \theta_p + M_0 \mathcal{H}_{0,z} \cos \theta_p > 0;
	\ee

	Таким образом $\theta = 0$ - устойчиво, при любых $\mathcal{H}_{0,z} > 0$;
	$\theta= \pi$ -  устойчиво, при $\beta M_0 > \mathcal{H}_{0,z} > 0$ и 
	неустойчиво, если $\beta M_0 < \mathcal{H}_{0,z}$;
	$\cos \theta =  - \frac{\mathcal{H}_{0,z}}{\beta M_0}$ - неустойчиво, если
	$0 < \mathcal{H}_{0,z} < \beta M_0$.

	В таком случае соответствующие энергии состояний:
	\be
		E(\theta = 0) = - \frac{\beta M_0^2 }{2} - M_0 \mathcal{H}_{0,z};
	\ee
	\be
		E(\theta = \pi) = - \frac{\beta M_0^2 }{2} + M_0 \mathcal{H}_{0,z};
	\ee
	\be
		E(\theta =  -\frac{\mathcal{H}_{0, z}}{\beta M_0}) = \frac{\mathcal{H}_{0,z}^2 }{2\beta};
	\ee

	Допустим, что мы перемагничиваем частицу, т.е.
	увеличиваем магнитное поле от 0. А направаление
	магнитного поля $\theta  = \frac{\pi}{2}$. 
	Если $T \rightarrow 0$, то намагниченность будет иметь первоначальное
	направление, т.е. частица не будет перемагничиваться, пока поле не достигнет 
	величины $\mathcal{H}_{0, z} = \mathcal{H}_{0,z,\text{cr}} = \beta M_0$.

	При  $T \neq 0$  и при уcловии 
	$0 < \mathcal{H}_{0,z} < \beta M_0 = \mathcal{H}_{0,z,\text{cr}}$.

	При достижении $\mathcal{H}_{0,z,\text{cr}}$ направление намагниченности меняет направление
	с $\theta = \pi$ до $\theta  = 0$. Зависимость намагниченности от магнитного поля имеет вид
	прямоугольника на плоскости $\mathcal{H}_{z},~M_z$. Т.е. тут наблюдается некий гистерезис.
	По оси $\mathcal{H}_z$ у нас прямоугольник длится с $-\beta M_0 \hdots \beta M_0$.
	А по оси $M_z$ имеет размер $-M_0 \hdots M_0$. Видно, что такое гистерезисное поведение
	, обусловленное тем, что состояния, отличающиеся направлением намагниченности отделены друг
	от друга энергетическим барьером.

	Здесь это явление обусловленно формой, а бывает, что и кристаллографическими свойствами.
	Это чисто классическое рассмотрение. Если рассматривать спин квантовомеханически, 
	то при малых размерах будет совершенно иное поведение.

	\begin{tcolorbox}
		Идём в хорошем темпе. Курс закончится в конце ноября. Поэтому можно сдать экзамен 
		в первую неделю декабря. 

		Нужно распространить задачи. И разобрать их.
	\end{tcolorbox}
\end{document}
