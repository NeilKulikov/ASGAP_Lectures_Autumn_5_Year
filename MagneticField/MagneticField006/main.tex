%!TEX encoding = UTF-8 Unicode
\documentclass[a4paper, 14pt, russian]{article}
\usepackage[a4paper]{geometry}
\usepackage[T2A]{fontenc}
\usepackage[utf8]{inputenc}
\usepackage[russian]{babel}
\usepackage{physics}
\usepackage{tcolorbox}
\usepackage{hyperref}
\usepackage{fancyhdr}
\usepackage{indentfirst}
\usepackage{amssymb, amsmath, amsfonts}

\title{Физика магнитных явлений}
\author{Иосиф Давидович Токман}
\date{}

\newcommand{\be}{\begin{equation}}
\newcommand{\ee}{\end{equation}}
\newcommand{\bea}{\begin{equation}\begin{array}}
\newcommand{\eea}{\end{array}\end{equation}}

\newcommand{\pa}{\partial}
\newcommand{\rot}{\textbf{rot}~}
\renewcommand{\div}{\textbf{div}~}
\renewcommand{\grad}{\textbf{grad}~}

\setcounter{section}{10}

\begin{document}
	\maketitle

	\section{Спиновой обменный оператор Дирака. Взаимодействие Ван-Флека-Гейзенберга.}

	К чему мы пришли? В основном мы занимались обменным взаимодействием. Это так причина,
	которая может объяснять эффективное азаимодействие частиц со спином. Это может объяснять
	ферромагнетизм и антиферромагнетизм.

	В качестве объектов мы в основном рассматривали атом. И поняли почему может быть 
	нескомпенсированный момент.

	А что происходит в твердом теле? Есть ли какое-то упорядочение? Мы начнем с простых моделей.

	Легко убедиться, что собственные значения и ссобственные ф-ии оператора:
	\be
		\label{eq83}
		\hat{V}_{ex} = - \frac{1}{2} J_{ex} ( 1 + 4 \hat{S}_1 \cdot \hat{S}_2);
	\ee

	Действующего в пространстве спиноров:
	\begin{eqnarray}
		\Phi_1 = 
			\begin{bmatrix}
				1 \\ 0
			\end{bmatrix}_1
			\begin{bmatrix}
				1 \\ 0
			\end{bmatrix}_2;\\
		\Phi_3 = 
			\begin{bmatrix}
				0 \\ 1
			\end{bmatrix}_1
			\begin{bmatrix}
				1 \\ 0
			\end{bmatrix}_2;\\
		\Phi_2 = 
			\begin{bmatrix}
				1 \\ 0
			\end{bmatrix}_1
			\begin{bmatrix}
				0 \\ 1
			\end{bmatrix}_2;\\
		\Phi_4 = 
			\begin{bmatrix}
				0 \\ 1
			\end{bmatrix}_1
			\begin{bmatrix}
				0 \\ 1
			\end{bmatrix}_2;
	\end{eqnarray}

	Тогда собственные значения получаются равными:
	\begin{eqnarray}
		\label{eq84}
		J_{ex} \rightarrow \Phi_2 - \Phi_3 = \Phi_{a}(1,2);\\
		- J_{ex} \rightarrow \Phi_1 = \Phi_{s}^{+1}(1,2);\\
		- J_{ex} \rightarrow \Phi_2 + \Phi_3 = \Phi_{s}^{0}(1,2);\\
		- J_{ex} \rightarrow \Phi_4 = \Phi_{s}^{-1}(1,2);
	\end{eqnarray}

	Таким образом, при рассмотрении примера в разделе VIII мы интересовались бы 
	лишь смещением уровней из-за обменного взаимодействия, то достаточно было
	бы рассмотреть оператор $\hat{V}_{ex}$ - спиновой обменный оператор Дирака.

	И у нас сейчас единственная степень свободы  - просто спин. А дальше надо
	сделать другие шаги.

	Из $\hat{V}_{ex}$ видно, что \textit{взаимодействие спинов изотропно и зависит лишь от 
	взаимной ориентации спинов.} Т.е. состояния бесконечно вырождены по направлениям спинов.

	Спиновой обменный оператор Дирака допускает важное обобщение. Пусть атомы с отличными от 
	0 спиновыми моментами располагаются в узлах кристаллической решётки. И пусть 
	в следствии вида собственных значений между электронами соседних атомов существует 
	обменное взаимодействие. Тогда, пользуясь усредненными величинами, оператор 
	обменного взаимодействия между спиновыми моментами атомов может быть записан ввиде
	Гейзенберговского гамильтониана:
	\be
		\label{eq85}
		\hat{H}_{ex} = - \frac{1}{2} \sum_{i\neq j} J_{ij} \hat{\vec S}_i \cdot \hat{\vec S}_j;
	\ee

	Т.е. мы считаем обменные энергии зависящими лтшь от разницы координат частиц:
	\be
		J_{ij} = J(\abs{\vec{r}_i - \vec{r}_j});
	\ee

	Такой гамильтониан \ref{eq85} был введен Ван-Флеком, а ферромагнетизм был рассмотрен
	подробно Гейзенбергом. А что за полные оператьоры спина? Это должен быть
	оператор, действующий на атом целиком. В целом эта штука должны быть похожа
	на оператор полного момента. Если у нас $n$ электронов в таком атоме, то 
	матрицы $\hat S$ должын быть размером $2 n + 1$ - т.е. любому собственному орбитальному
	моменту мы сопоставляем отдельную "координату".

	При этом ферромагнетизм отвечает $J_{ij} < 0$ - тогда энергия в состоянии со спинами в 
	одну сторону будет наименьшей.

	\section{Локализованные невзаимодействующие моменты. Парамагнетизм.}

	Если есть невзаимодействующие спины но во внешнем поле. Пусть система 
	состоит из $N$ невзаимодействующих атомов, обладающих моментом $J$.
	Во внешнем статическом поле $\vec{\mathcal H}$ устроен следующим образом:
	\be
		\label{eq86}
		\hat H = \sum g_j \mu_B m_i \mathcal H;
	\ee

	Здесь $m = J_z = J \hdots - J$ - проекция момента отдельного атома, а 
	$\vec z \Uparrow \vec{\mathcal H}$. Здесь мы пользуемся неявно векторной
	моделью атома. Считаем, что точными интегралами являются $J^2,~J_z$,
	а хорошими интегралами $L,~S$.

	Понятно почему сохраняется $J_z$, а почему $J^2$ - ? Потому что здесь
	аналогия с вырожденной теорией возмущения - там мы берем в первом порядке 
	возмущения функции соответствующие начальному вырожденному сотсоянию, не
	примешивая сторонние.

	Как мы помним, при наложении внешнего поля у нас получалось:
	\be
		\mu_B (\vec L + 2 \vec S) \vec{\mathcal H} = \mu_B (\vec J + \vec S) \vec{\mathcal H} =
			\mu_B ({\vec J}_z + {\vec S}_z)\vec{\mathcal H};
	\ee

	Можем предположить:
	\be
		S_z = \abs{\vec S} \cos(\vec S ; \vec J)\cos(\vec J;\vec{\mathcal H});
	\ee

	Раскладывая по проекциям можем получить:
	\be
		\cos(\vec J; \vec{\mathcal H}) = \frac{J_z}{\abs{\vec J}};
	\ee

	А также запишем, исходя из коммутационных соотношений:
	\be
		\abs{\vec S}\abs{\vec J} \cos(\vec S; \vec J) = \frac{1}{2} 
			\big(J(J + 1) - L(L+1) + S(S+1)\big);
	\ee

	В таком случае:
	\be
		S_z = \frac{\sqrt{S(S+1)}}{\sqrt{S(S+1)}} \frac{J_z}{J(J+1)} 
			\frac{1}{2} \big(J(J + 1) - L(L+1) + S(S+1)\big);
	\ee

	Тогда в итоге у нас получится:
	\be
		\mu_B(\vec L  + 2 \vec S)\vec{\mathcal H} = \mu_B J_z \big(
			1+ \frac{J(J + 1) - L(L+1) + S(S+1)}{2J(J+1)}\big) \vec{\mathcal H};
	\ee

	Это у нас получается фактор Ланде. Теперь попробуем вычислить статсумму.
	\be
		\label{eq87}
		Z_N = \left(\sum_{m = - J}^J \exp(-\frac{g_J \mu_B m \mathcal H}{T})\right)^N;
	\ee

	Это в случае отсутствия взаимодействия - тогда статсумма полной системы - просто
	произведение статсумм отдельных элементов. Можно показать, что:
	\be
		Z_N = \left(\frac{\sinh(\frac{2J + 1}{2J} x)}{\sinh(\frac{x}{2J})}\right)^N = (Z)^N;
	\ee

	При этом безразмерная величина:
	\be
		\label{eq88}
		x = \frac{g_J \mu_B J \mathcal H}{T};
	\ee

	Отсюда видно понятие "высокой температуры" - когда $x \ll 1$.

	В равновесии у нас проекция получится равной:
	\be
		\langle M_z \rangle =  \sum_{conf} \big(-\sum_{i = 1}^N g_J \mu_B m_i\big)
			\exp(-\frac{\sum_{i=1}^N g_J \mu_B m_i \mathcal H}{T}) / Z_N;
	\ee

	И когда мы все это посчитаем, то увидим:
	\be
		\label{eq89}
		\langle M_z \rangle = \frac{Ng_J \mu_B \sum_{m=-J}^J m \exp(mx / J)}{Z};
	\ee

	Легко показать, что:
	\be
		\langle M_z \rangle = N g_J \mu_B J B_J(x);
	\ee

	Где подразумевается т.н. функция Бриллюэна:
	\be
		\label{eq90}
		B_J(x) = \frac{2J+1}{2J} \coth(\frac{2J+1}{2J} x) - \frac{1}{2J} \coth(\frac{x}{2J});
	\ee

	Рассмотрим предельный случай высокой температуры  $x \ll 1$. Тогда:
	\be
		B_J(x) \approx \frac{J+1}{3J} x;
	\ee

	В таком случае средний момент:
	\be
		\label{eq91}
		\langle M_z \rangle \approx \frac{N g_J^2 J(J+1)}{3T} \mathcal H;
	\ee


	Но есть еще и магнитная \textit{восприимчивость}. В данном случае
	она подчиняется закону Кюри:
	\be
		\chi = \frac{\pa \langle M_z \rangle}{\pa \mathcal H} = 
			\frac{Ng_J^2 \mu_B^2 J(J+1)}{3} \frac{1}{T};
	\ee

	А что будет, когда температура низка?
	\be
		\langle M_z \rangle \approx Ng_J \mu_B J \left(1 - \frac{1}{J}
			\exp(-\frac{g_J \mu_B \mathcal H}{T})\right);
	\ee

	Можно видеть что при абсолютном нуле:
	\be
		\label{eq92}
		\langle M_z \rangle \rightarrow_{T\rightarrow 0} Ng_J \mu_B J;
	\ee

	Но тогда восприимчивость:
	\be
		\chi = \frac{Ng_J^2 \mu_B^2}{T} \exp(-\frac{g_J \mu_B \mathcal H}{T});
	\ee

	Видно, что она обращается в ноль. Это за счет того, что
	все атомы и так уже упорядочены.

	А теперь попробуем все это упорядочить за счет еще и взаимодействия
	самих атомов.

	\section{Ферромагнетизм "на пальцах". Модель Кюри-Вейса. Приближение среднего (молекулярного) поля.}

	Первое, что можно предположить - пусть остальные частицы создают некоторое 
	среднее поле, тогда все остальное идёт уже по накатанной.

	Гейзенберговский гамильтониан \ref{eq85} является подходящей основой для теориии
	магнетизма в диэлектриках, где электроны достаточно хорошо локализованны, а 
	магнитный мент связан именно со спинами. Это предположение в точности выполняется
	в атомах или ионах, где $L=0$, неапример в $Mn^{2+},~Gd^{2+}$. Кроме того,
	\ref{eq85} может описывать магнетизм и в переходных металлах $Fe,~Co,~Ni$. 
	Намагниченность этих металлов обусловлена спинами $d$ электронов, которые хорошо
	локализованны.
	
	Вспомним, что мы рассматривали $Fe$ - последние 2-е оболочки $d,~s$, причем $d$ - 
	не заполнено до конца, а еще она по радиусу меньше. Это в отдельном атоме.
	При этом в кристалле электронные состояния становятся делокализованными.
	Это - причина того, что энергетический уровень превращается в зону.

	Когда делокализуется $d$ электрон - зона получается очень узкой. У них
	оказывается очень большая масса, низка подвихность, поэтому они оказываются
	практически неподвижными.

	Поэтому в грубой модели есть 2-е группы электронов - легкие и подвижные $s$
	электроны, а еще и "локализованные" $d$ электроны. Т.е. мы "забываем" про то, что
	они образуют зону.

	Из \ref{eq85} видно, что если $J_{ij} >0$, то энергетически выгодным при $T = 0$
	является состояние, где все спины $\vec S$ сонаправленны - ферромагнитное упорядочение.
	В образце возникает спонтанный магнитный момент $\vec M$. По мере роста температуры
	происходит расупорядочение спинов (спонтанный магнитный момент уменьшается).

	Точка, при которой спонтанный магнитный момент обращается нуль $\vec M = 0$.
	Простейшее описание ферромагнетика можно получить в рамках среднего поля. 
	Рассматривается спин отдельного атома. Все взаимодействие спина этого атома со спинами
	остальных атомов (в рамках Гейзенберговсеого гамильтониана) заменяется взаимодействием
	с некоторым эффективным полем. \textit{По идее Вейса это эффективное (как бы магнитное) поле пропорционально среднему истинному
	магнитному моменту кристалла.}

	Выделим в \ref{eq85} спин отдельного атома. Т.е. запишем исходя из \ref{eq85}
	гамильтониан одного атома.
	\be
		\label{eq93}
		\hat{H}_{ex,i} = - \hat{\vec S}_i J_0 \sum_{j = 1}^Z \hat{\vec S}_j;
	\ee

	В \ref{eq93} учтено взаимодействие с ближайшими соседями. И положено
	$J_{ij} = J_0$. Если соспоставить этому гамильтониану взаимоджействие 
	с неким "мигнитным" полем $\mathcal{H}_{eff}$, то:
	\be
		\label{eq94}
		\hat{H}_{ex} = g_s \mu_B \hat{\vec S}_i \mathcal{H}_{eff};
	\ee

	В соответствии с идеей Вейса в \ref{eq93} заменим $\hat{\vec S}_j 
	\rightarrow \langle \hat{\vec S}_j \rangle$. Тогда наше эффективное 
	поле:
	\be
		\label{eq95}
		\mathcal{H}_{eff}  = -\frac{J_0}{g_s\mu_B} \sum_{j=1}^z
		 \langle \hat{\vec S}_j \rangle = -\frac{J_0 z}{g_s\mu_B}
		 \langle \hat{\vec S} \rangle;
	\ee

	Тогда для полного магнитного момента кристалла получим что-то вроде:
	\be
		\label{eq96}
		\vec M = - N g_s \mu_B \langle \hat{\vec S} \rangle;
	\ee

	Из этих соотношений \ref{eq95}, \ref{eq96} мы имеем:
	\be
		\label{eq97}
		\mathcal{H}_{eff} =  \frac{zJ_0}{N g_s^2 \mu_B^2} \vec M = \gamma \vec M;
	\ee

	Здесь $\gamma$ - коэффициент молеклярного поля Вейса:
	\be
		\gamma = \frac{zJ_0}{N g_s^2 \mu_B^2};
	\ee

	теперь мы сможем применять уже готовые формулы для внешнего магнитного поля.
\end{document}
