%!TEX encoding = UTF-8 Unicode
\documentclass[a4paper, 14pt, russian]{article}
\usepackage[a4paper]{geometry}
\usepackage[T2A]{fontenc}
\usepackage[utf8]{inputenc}
\usepackage[russian]{babel}
\usepackage{physics}
\usepackage{tcolorbox}
\usepackage{hyperref}
\usepackage{fancyhdr}
\usepackage{indentfirst}
\usepackage{amssymb, amsmath, amsfonts}

\title{Физика магнитных явлений}
\author{Иосиф Давидович Токман}
\date{}

\newcommand{\be}{\begin{equation}}
\newcommand{\ee}{\end{equation}}
\newcommand{\bea}{\begin{equation}\begin{array}}
\newcommand{\eea}{\end{array}\end{equation}}

\newcommand{\pa}{\partial}
\newcommand{\rot}{\textbf{rot}~}
\renewcommand{\div}{\textbf{div}~}
\renewcommand{\grad}{\textbf{grad}~}

\setcounter{section}{14}

\begin{document}
	\maketitle

	\section{Магнитооптика}

	Взаимодействие электромагнитного поля с материей. Постоянные и переменные
	поля в магнетике. 

	Взаимодействие с зарядом (которые связаны с магнитным порядком) в твёрдом теле. 
	Прежде чем углубляться в глубокие материи рассмотрим самое простое.

	Рассмотрим следующую модель:
	Некая среда - набор одинаковых заряженных осцилляторов. Осцилляторы с 
	\textit{исчезающе малым затуханием}. На эту среду наложено
	постоянное магнитное поле $\vec{\mathcal H}_0 \uparrow\uparrow OZ$.
	Исследуем диэлектрические свойства такой среды 
	в пренебрежении пространственной дисперсии. Декларация без точного
	математиеского описания - бесполезна.

	\be
		\ddot x + \omega_0^2 x = \frac{e}{m} E_{0x} e^{-i\omega t} 
				+ \frac{e{\mathcal H}_0}{mc} \dot y;
	\ee
	\be
		\ddot y + \omega_0^2 y = \frac{e}{m} E_{0y} e^{-i\omega t} 
				- \frac{e{\mathcal H}_0}{mc} \dot x;
	\ee
	\be
		\ddot z + \omega_0^2 z = \frac{e}{m} E_{0} e^{-i\omega t};
	\ee

	Стационарное решение:
	\be
		x = \alpha_{xx} E_{0x} e^{-i\omega t} + \alpha_{xy} E_{0y} e^{-i\omega t};
	\ee
	\be
		y = \alpha_{yy} E_{0y} e^{-i\omega t} + \alpha_{yx} E_{0x} e^{-i\omega t};
	\ee
	\be
		z = \alpha_{zz} E_{0z} e^{-i\omega t};
	\ee

	Это решение на временах, много меньше времени затухания.  Рассмотрим
	компоненты тензора поляризуемости.

	\be
		\alpha_{xx} = \alpha_{yy} = \frac{e}{m} \frac{(\omega_0^2 - \omega^2)}
			{(\omega_0^2 - \omega^2)^2 - 4 \omega^2 \omega_0^2};
	\ee
	\be
		\alpha_{xy} = - \alpha_{yx} = -i\frac{e}{m} \frac{2\omega \omega_L}
			{(\omega_0^2 - \omega^2)^2 - 4 \omega^2 \omega_0^2};
	\ee
	\be
		\alpha_{zz} = \frac{e}{m} \frac{1}{(\omega_0^2 - \omega^2)};
	\ee

	Соответствующий тензор диэлектрической проницаемости:
	\be
		\epsilon_{ij} = \delta_{ij} + 4\pi N \alpha_{ij};
	\ee

	Если записывать как матрицу:
	\be
		\hat{\epsilon} = 
			\begin{bmatrix}
				\frac{1}{2} (\epsilon_+ + \epsilon_-)	& \frac{i}{2} (\epsilon_+ - \epsilon_-)	& 0\\
				-\frac{i}{2} (\epsilon_+ - \epsilon_-)	& \frac{1}{2} (\epsilon_+ + \epsilon_-) & 0\\
				0 & 0 & \epsilon_0
			\end{bmatrix};
	\ee

	Здесь подразумевается:
	\be
		\epsilon_\pm = 1 - \frac{\omega_p^2}{\omega(\omega \pm 2 \omega_L) - \omega^2};
	\ee
	\be
		\epsilon_0 = 1 - \frac{\omega_p^2}{\omega^2 - \omega_0^2};
	\ee

	При этом: $\omega_L = \frac{e \mathcal{H}_0}{2mc},~\omega_p^2  = \frac{4\pi e^2 N}{m}$. Затухания нет - значит тензор эрмитов.

	В чем разница между Ларморовской и гиро- частотой? Они отличаются в 2-а раза.

	Рассмотренная модель соответствует гироэлектрической среде - наличию комлексных
	недиагональных элементов тензора.

	Какие собственные волны распространяются в такой среде?

	Пусть волна распространяется по оси $Z$ - что дальше? Волна распадается
	на собственные моды - циркулярно поляризованные. Это называется эффектом
	Фарадея.

	Ситуация была оень простая. А что может дать квантовая механика?
	Можем полуить сам вид того самого тензора восприимивости.
	Оказывается, то изменятся только числа. 

	Наш же подход хорош простотой и сохранением симметрии. Выше была Рассмотренна ситуация, когда внешнее магнитное поле изменяет
	диэлектриеские свойства среды. Аналогично можно рассмотреть ситуацию, когда магнитные свойства среды изменяются под действием внешнего магнитного поля.

	Наверное существуют частоты, когда надо учитывать и магнитное поле волны.
	Пусть среда - совокупность идентиных классических спинов во внешнем однородном
	постоянном магнитном поле. В отсутствии других магнитных полей. Спины прецессируют
	с бесконечно малым затуханием. Наложим однородное переменное магнитное поле. Тогда гамильтониан одного спина:
	\be
		\label{eq177}
		\hat{H} = g \mu_B (S_z \mathcal{H}_0 + S_z \mathcal{H}_{0z} e^{-\omega t}
			+ S_y \mathcal{H}_{0y} e^{-\omega t} + S_x \mathcal{H}_{0x} e^{-\omega t});
	\ee

	$\mathcal{H}_{0\alpha}$ - амплитуды переменного магнитного поля. $\omega$ - его частота.

	Для магнитного момента одного спина $\vec M$ имеем:
	\be
		\label{eq178}
		\hat{H} = - m_z (\mathcal{H}_0 + \mathcal{H}_{0z}e^{-i\omega t})
		- m_x \mathcal{H}_{0x}e^{-i\omega t} - m_z \mathcal{H}_{0y}e^{-i\omega t};
	\ee

	Воспользуемся тем, что мы уже писали для классиеской цепочки спинов.
	\be
		[m_\alpha; m_\beta] = \frac{g\mu_B}{\hbar} m_\gamma = 
			\frac{g\abs{e}}{2mc} m_\gamma = \zeta m_\gamma;
	\ee

	Тогда мы имеем уравнение движения. Из него мы собираемся получить уравнение связи.
	Нам это важно, тобы совать в уравнения для полей.
	\be
		\label{eq180}
		\dot{m}_x = -(\mathcal{H}_0 + \mathcal{H}_{0z}e^{-i\omega t}) \zeta m_y 
			+ \mathcal{H}_{0y}e^{-i\omega t} \zeta m_z;
	\ee
	\be
		\dot{m}_y = -(\mathcal{H}_0 + \mathcal{H}_{0z}e^{-i\omega t}) \zeta m_x 
			+ \mathcal{H}_{0x}e^{-i\omega t} \zeta m_z;
	\ee
	\be
		\dot{m}_z =
			\mathcal{H}_{0x}e^{-i\omega t} \zeta m_y - \mathcal{H}_{0y}e^{-i\omega t} \zeta m_x;
	\ee

	Будем считать, что $\mathcal{H}_{0z} \ll \mathcal{H}_0$. Кроме того 
	будем считать $m_z \approx m_0$, а также $\Delta m_z \ll m_x,~m_y \ll m_0$.

	Тогда в линейном приближении будем иметь ($\omega_0 = \zeta \mathcal{H}_0$ - частота однородной прецессии):
	\be
		\label{eq181}
		\dot{m}_x = - \omega_0 m_y + \zeta \mathcal{H}_{0y} m_0 e^{-i\omega t};
	\ee
	\be
		\dot{m}_y = \omega_0 m_x - \zeta \mathcal{H}_{0x} m_0 e^{-i\omega t};
	\ee
	\be
		\Delta \dot{m}_z = 0;
	\ee

	Стационарное решение выглядит так:
	\be
			m_x = \frac{\zeta m_0 \omega_0}{\omega_0^2 - \omega^2} 
				\mathcal{H}_{0x} e^{-i\omega t} - i 
				\frac{\zeta m_0 \omega}{\omega_0^2 - \omega^2} 
				\mathcal{H}_{0y} e^{-i\omega t};
	\ee
	\be
			m_y = \frac{\zeta m_0 \omega_0}{\omega_0^2 - \omega^2} 
				\mathcal{H}_{0y} e^{-i\omega t} + i 
				\frac{\zeta m_0 \omega}{\omega_0^2 - \omega^2} 
				\mathcal{H}_{0x} e^{-i\omega t};
	\ee

	Тогда для одного атома будем иметь:
	\be
		\chi_{xx} = \chi_{yy} = \frac{\zeta m_0 \omega_0}{\omega_0^2 - \omega^2};
	\ee
	\be
		\chi_{xy} = -\chi_{yx} = -i \frac{\zeta m_0 \omega}{\omega_0^2 - \omega^2};
	\ee

	В таком случае тензор магнитной проницаемости будет таким:
	\be
		\label{eq182}
		\hat{\mu} = 
			\begin{bmatrix}
				1 + \frac{4\pi N \zeta m_0 \omega_0}{\omega_0^2 - \omega^2 }
				& 
				-i \frac{4 \pi N \zeta m_0 \omega}{\omega_0^2 - \omega^2}
				&
				0\\
				i \frac{4 \pi N \zeta m_0 \omega}{\omega_0^2 - \omega^2}
				&
				1 + \frac{4\pi N \zeta m_0 \omega_0}{\omega_0^2 - \omega^2 }
				& 
				0\\
				0 & 0 & 1
			\end{bmatrix};
	\ee

	Эта модель соответствует гиромагнитной среде - наличие комплексной 
	недиагональной компоненты в тензоре магнитной проницаемости.

	Какие будут собственные волны? Что с ними будет происходить?


	В реальности что из этого срабатывает - гироэлектриеские эффекты или
	гиромагнитные?

	На самом деле - и то и другое. А характерное поведение зависит 
	в основном от частоты.

	Теперь попробуем рассмотреть то же самое в простой квантовомеханической 
	модели.
	
	\section{Магнитооптика в ферромагнитной среде.}

	Рассмотрим в качестве модели гипотетический магнитоактивный атом, который устрен 
	следуюзим образом - у группы электронов этого атома отличен от нудя полный спин,
	но равен нулю полный орбитальный момент.
	
	С этой группой электронов взаимодействует один оптический электрон, спин
	которого не будем учитывать.

	Соответствующий одноэлектронный гамильтониан запишем как $\vec s \neq 0,~L=0$:
	\be
		\label{eq184}
		\hat{H}_0 = \hat{H}_s + A(\hat{\vec l} \cdot \hat{\vec s});
	\ee

	Второй член здесь описывает взаимодействие с оптическим электроном.

	Что такое оптический атом? Пусть есть внешняя оболочка со спонтанным спиновым моментом. И пусть на ней есть один активный электрон.

	Спиновое состояние атома будем ситать заданным - т.е.:
	\be
		\hat{\vec S} \Rightarrow \langle \hat{\vec S} \rangle = \hat{S};
	\ee

	Гамильтониан \ref{eq184} описывает поведение оптиеского элеткрона в заданном поле,
	имеющем симметрию магнитного поля. Сам оператор соответствует релятивистскому 
	спин-орбитальному взаимодействию.

	В отсутствии спин-орбитального взаимодействия будем считать, чтот основному состоянию $\hat{H}_S$ - соответствует функция $\psi_1$  с $l=0$, а первое возбуждённое состояние 3-ёх кратно вырождено: $\psi_2(l=1, m = -1),~\psi_3(l=1,m=0),
	\psi_4(l=1,m=1)$.

	Тогда второй слен снимает вырождение с собственных функций начального гамильтониана.
	Будет такой порядок уровней: $E_1 < E_2 < E_3 < E_4$ (индексы - номера функций).
	$E_4 - E_3 = E_3 - E_2 = AS$, а с основным состояние расщепление $\hbar \omega_0$.

	Пусть взаимодействие ферромагнитных атомов устанавливает определённый порядок. Мы имеем модель для рассмотрениея оптических свойств ферромагнетика.

	Рассмотрим взаимодействие оптического электрона с полем в дипольном приближении:
	\be
		\label{eq185}
		\hbar{H}_\text{int}(t) = - \hat{\vec d} \vec E(t);
	\ee

	Если поле гармоническое. То:
	\be
		\label{eq186}
		\hbar{H}_\text{int} = -d_\alpha E_{0\alpha}e^{-i\omega t};
	\ee

	С учётом этого полный гамильтониан имеет вид:
	\be
		\label{eq187}
		\hat{H} = \hat{H}_\text{int} + \hat{H}_0;
	\ee

	Матрица плотности в таком слуае подчиняется уравнению:
	\be
		\label{eq188}
		-i\hbar \frac{\pa \rho}{\pa t} = [\hat{H}_\text{int};\hat{\rho}];
	\ee

	В чем смысл матрицы плотности:
	\be
		\psi = \sum c_n \psi_n;
	\ee

	А дальше мы мереем оператор $\hat A$:
	\be
		A_{ij} = \int \psi_i^{*} \hat{A} \psi_j dv; 
	\ee

	Тогда средее на заданной $\psi$:
	\be
		\langle A \rangle = \sum_{ij} c_i^{*} c_{j} A_{ij};
	\ee

	Но иногда не удается описать таким образом. Например когда происходит
	взаимодействие с другой системой. Или например можно так учитывать взаимодействие подсистем.

	Ураввнение \ref{eq188} соответствует представлению взаиимодействия. В этом представлении операторы:
	\be
		\label{eq189}
		\hat{A}(t) = e^{i \hat{H}_0 t / \hbar} \hat{A} e^{-i \hat{H}_0 t / \hbar};
	\ee

	Это шредингеровское представление. Тогда получим выражение для матрицы плотности:
	\be
		\label{eq190}
		\hat{\rho} (t) = \hat{\rho}^{(0)} +\frac{1}{i\hbar}\int_{-\infty}^t
			[\hat{H}_\text{int};\rho(t_1)] dt_1;
	\ee

	Здесь мы ситаем то взаимодействие ввечно. $\hat{\rho}^{(0)} = \hat{\rho}(-\infty)$.
	Тогда в первом приближении:
	\be
		\hat{\rho} (t) \approx \hat{\rho}^{(0)} + \hat{\rho}^{(1)}(t);
	\ee

	Таким образом:
	\be
		\label{eq191}
		\hat{\rho}^{(1)}(t) = \frac{1}{i\hbar} \int_{-\infty}^t
			[\hat{H}_\text{int};\rho^{(0)}] dt_1;
	\ee

	Поскольку реь идет про средние по ансамблю (измерений), то в первом приближении возникающий дипольный момент:
	\be
		\langle \hat{A} \rangle = \text{Sp}(\hat{\rho}\hat{A});
	\ee

	Пользуясь таким правилом можно получить:
	\begin{multline}
		\langle \hat{d}_\alpha (t) \rangle = \sum_\beta \left(\frac{i}{\hbar}
		\int_{-\infty}^t \text{Sp}([\hat{d}_\beta(t_1) E_\rho(t_1);
		\hat{\rho}^{(0)}])\right) dt_1 = \\
		\sum_{\beta, m, n, k} \frac{i}{\hbar}\int_{-\infty}^t
		\{\{(d_\beta(t_1))_{mn}(d_\alpha(t_1))_{nm} - (d_\beta(t_1))_{nk}(d_\alpha(t_1))_{kn}\}\rho_{nn}^{(0)} E_\beta(t_1)\} dt_1;
	\end{multline}

	Здесь $m, n, k$ - индексы матричных элементов. $\alpha,\beta$ - просто координаты.
	Ненулевые компоненты матриного представления дипольного момента в Шредингеровском представлении:
	\be
		(d_x)_{12} = (d_x)_{21} = (d_x)_{14} = (d_x)_{41} = d;
	\ee
	\be
		(d_y)_{12} = - id;~(d_y)_{21} = id;~(dy)_{14} = id;~(d_y)_{41} = -id;
	\ee
	\be
		(d_z) = d';
	\ee

	Рассмотрим $\alpha  = \beta = x$, тогда:
	\be
		\langle d_x(t)\rangle = \sum_{n, m, k} \frac{i}{\hbar}
		\int_{-\infty}^t \{(d_x(t))_{nm} (d_x(t_1))_{mn} - (d_x(t_1))_{nk} (d_x(t))_{kn}\} \rho_{nn}^{(0)} E_{0x} e^{-i\omega t} dt_1;
	\ee

	Получим:
	\begin{multline}
	\langle d_x(t)\rangle = \frac{i}{\hbar} d^2 E_{0x}\left(
		(\rho_{11}^{(0)} - \rho_{22}^{(0)})\int_{\infty}^0(
		e^{i(\omega_{21} + \omega ) \tau} - e^{i(-\omega_{21}+ \omega)\tau}) d\tau\right. + \\ \left.(\rho_{11}^{(0)} - \rho_{44}^{(0)})\int_{\infty}^0(
			e^{i(\omega_{41} + \omega ) \tau} - e^{i(-\omega_{41}+ \omega)\tau}) d\tau\right) e^{-i\omega t};
	\end{multline}

	К сожалению эти интегралы расходятся, но можно сделать трюк с бесконечно малым затуханием.

	Формально можно учесть медленное затухание, как:
	\be
		E_{0x}e^{-i\omega t} e^{\epsilon t} \Rightarrow \omega \rightarrow \omega + i \epsilon;
	\ee

	В таком случае ответ будет равным:
	\be
		\label{eq195}
		\langle d_x (t) \rangle = \frac{1}{\hbar} d^2 \left\{
			(\rho_{11}^{(0)} - \rho_{22}^{(0)})\frac{2\omega_{21}}{\omega_{21}^2 - \omega^2} + (\rho_{11}^{(0)} - \rho_{44}^{(0)})\frac{2\omega_{41}}{\omega_{41}^2 - \omega^2}\right\} E_{0x} e^{-i\omega t};
	\ee

	При этом это работает, только, когда $\omega \neq \omega_{21},\omega{41}$. Что соответствует
	отсутствию поглощения.
\end{document}
