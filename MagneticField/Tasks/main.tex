%!TEX encoding = UTF-8 Unicode
\documentclass[a4paper, 14pt, russian]{article}
\usepackage[a4paper]{geometry}
\usepackage[T2A]{fontenc}
\usepackage[utf8]{inputenc}
\usepackage[russian]{babel}
\usepackage{physics}
\usepackage{tcolorbox}
\usepackage{hyperref}
\usepackage{fancyhdr}
\usepackage{indentfirst}
\usepackage{amssymb, amsmath, amsfonts}

\title{Физика магнитных явлений. Список заданий}
\author{Иосиф Давидович Токман}
\date{}

\newcommand{\be}{\begin{equation}}
\newcommand{\ee}{\end{equation}}
\newcommand{\bea}{\begin{equation}\begin{array}}
\newcommand{\eea}{\end{array}\end{equation}}

\newcommand{\pa}{\partial}
\newcommand{\rot}{\textbf{rot}~}
\renewcommand{\div}{\textbf{div}~}
\renewcommand{\grad}{\textbf{grad}~}

\begin{document}
	\maketitle

	\section{13/09/19}

	\begin{tcolorbox}
		\textbf{Задание}: залезть в ЛЛ. Посмотреть на то, как трансляция связана с 
		оператором момента импульса. Пользуясь этим знанием написать оператор 
		поворота, записанный через оператор момента импульса или спиновые операторы.
	\end{tcolorbox}

	\section{22/09/19}

	\begin{tcolorbox}
		\textbf{Задание:} В некоторой системе координат задан спинор:
		\be
			\begin{bmatrix}
				\sqrt{i}\\
				10^{27}
			\end{bmatrix}
		\ee

		Надо выяснить какое среднее значение проекции спина на ось, которая 
		имеет углы $\alpha,~\beta,~\gamma$ с осями координат.
	\end{tcolorbox}
	\begin{tcolorbox}
		\textbf{Задание:} Проверить, что это утверждение справедливо 
		и посмотреть соответствующий раздел в курсе теоретической физики ЛЛ.
	\end{tcolorbox}
	\begin{tcolorbox}
		\textbf{Задание:} Каким образом можно вытащить электрон с нижних состояний? 
		(Родить электрон-позитронную пару)
	\end{tcolorbox}

	\section{27/09/19}

	\begin{tcolorbox}
		\textbf{Задача:} подумать над тем как экспериментально можно
		проверить принцип тождественности.
	\end{tcolorbox}
	\begin{tcolorbox}
		\textbf{Задача:} рассмотреть атом гелия.

		\textbf{Примечание:} если $\psi_n$, $\psi_m$,
		принадлежат атому, то $E_0 \sim J_{ex}'$.
	\end{tcolorbox}

	\section{4/10/19}

	\begin{tcolorbox}
		\textbf{Задача:} Задача после \S 67  ЛЛ3. С 3 $p$ электронами.
		Найти полную волновую функцию.
	\end{tcolorbox}

	\section{25/10/19}

	\begin{tcolorbox}
		\textbf{Задача:} В чем отличие между поведениями магнитных и электрических диполей 
		в однородном внешнем поле. При том, что гамильтонианы у них одинаковые.
	\end{tcolorbox}

	\section{7/10/19}
	\begin{tcolorbox}
		\textbf{Задача:} Что такое низкие температуры? Чем она обусловлена?
		Какую точность обеспечивает это приближение? 
	\end{tcolorbox}


\end{document}
