%!TEX encoding = UTF-8 Unicode
\documentclass[a4paper, 14pt, russian]{article}
\usepackage[a4paper]{geometry}
\usepackage[T2A]{fontenc}
\usepackage[utf8]{inputenc}
\usepackage[russian]{babel}
\usepackage{physics}
\usepackage{tcolorbox}
\usepackage{hyperref}
\usepackage{fancyhdr}
\usepackage{indentfirst}
\usepackage{enumerate}
%\usepackage{enumitem}
\usepackage{amssymb, amsmath, amsfonts}

\title{Физика магнитных явлений}
\author{Иосиф Давидович Токман}
\date{}

\newcommand{\be}{\begin{equation}}
\newcommand{\ee}{\end{equation}}
\newcommand{\bea}{\begin{equation}\begin{array}}
\newcommand{\eea}{\end{array}\end{equation}}

\newcommand{\pa}{\partial}
\newcommand{\rot}{\textbf{rot}~}
\renewcommand{\div}{\textbf{div}~}
\renewcommand{\grad}{\textbf{grad}~}

\newcommand{\mybox}[1]{\fbox{\parbox{1.\textwidth}{#1}}}

\setcounter{equation}{11}
\setcounter{section}{5}

\begin{document}
	\maketitle

	\begin{tcolorbox}
		\textbf{Задание:} В некоторой системе координат задан спинор:
		\be
			\begin{bmatrix}
				\sqrt{i}\\
				10^{27}
			\end{bmatrix}
		\ee

		Надо выяснить какое среднее значение проекции спина на ось, которая 
		имеет углы $\alpha,~\beta,~\gamma$ с осями координат.
	\end{tcolorbox}

	\section{Появление спина}

	Он появился с одной стороны из эксперимента Штерна-Герлаха, а с другой  - из 
	эффекта Зеемана. И было видно, что есть не только пространственные координаты, 
	но и что-то ещё.

	Если же говорить про теорию - то это следствие уравнения Дирака.

	Видно, что в соответствующем гамильтониане у нас фигурируют матрицы $4 \times 4$.
	Соответственно и спиноры должны быть 4-х компонентными. Можно зааметить, что
	прпоизводная по времени - первая. Значит задание волновой функции в начальный 
	момент времени задаёт ее эволюцию на все оставшееся время.

	А также в гамильтониан координаты входят так же, как и производная по-времени.
	С этой точки зрения время от координат  не отличимо. Операторы 
	$\hat{\vec a},~\hat{b}$ - такие, что:
	\be
		\hat{H}^2 = c^2 (\hat{p}_x^2 + \hat{p}_y^2 + \hat{p}_z^2) + m^2 c^4 = E^2;
	\ee

	То есть есть соблюдение релятивистского закона дисперсии. И эта связь не 
	только для одной частицы, но и для целой системы.

	Как преобразуется энергия при смене системы координат? Оказывается, что
	масса в этой записи остаётся неизменной - нет никаких масс покоя и масс движения.

	Какая масса фотона? Их закона дисперсии фотона $E = \hbar \omega$.
	Тогда его масса равняется нулю. 
	
	Если мы возьмем два фотона, движущихся друг на встречу друг другу.
	Тогда её полный импульс $\vec p = 0$, а вот энергия $E = 2 \hbar \omega$.
	Отсюда можно видеть, что масса пролучается отличной от нуля.

	\begin{tcolorbox}
		\textbf{Задание:} Проверить, что это утверждение справедливо 
		и посмотреть соответствующий раздел в курсе теоретической физики ЛЛ.
	\end{tcolorbox}

	Из уравнения Дирака видно, что волновая функция представляет собой 4-х компонентный
	столбец - биспинор Дирака.
	\be
		\psi = \begin{bmatrix} \psi_1\\ \psi_2\\ \psi_3\\ \psi_4 \end{bmatrix};
	\ee

	Тогда уравнение покомпонентно уравнение дирака имеет такой вид:
	\be
			i \hbar \pa_t \psi_1 = c (p_x  - i p_y) \psi_4 + c p_z \psi_3 + mc^2 \psi_1;
	\ee
	\be
			i \hbar \pa_t \psi_2 = c (p_x  + i p_y) \psi_3 - c p_z \psi_4 + mc^2 \psi_2;
	\ee
	\be
			i \hbar \pa_t \psi_3 = c (p_x  - i p_y) \psi_2 + c p_z \psi_1 - mc^2 \psi_3;
	\ee
	\be
			i \hbar \pa_t \psi_4 = c (p_x  + i p_y) \psi_1 + c p_z \psi_2 - mc^2 \psi_4;
	\ee

	В случае, если частица с зарядом $e$ движется в элетромагнитном поле $(\vec A; \psi)$,
	то уравнения Дирака записываются как:
	\be
		i\hbar \pa_t \psi = \left(c \hat{\vec a} \big(\hat{\vec p} - \frac{e}{c} \hat{\vec A}\big)
		+ e \phi +mc^2 \hat{\beta}\right) \psi;
	\ee

	И все это находится в  полном соответствии с классической механикой.
	
	Теперь зададимся вопросом: вернемся в классическую механику - когда сохраняется момент импульса?
	Только когда при поворотах вокруг какой-то оси ничего не меняется - то есть есть симметрия 
	относительно поворота.

	Что мы должны сделать в данном случае? проверить коммутативность какого-то оператора с гамильтонианом?
	То есть, если коммутатор какой-то величины с гамильтонианом зануляется - эта величина сохраняется.

	Это можно объяснить при помощи Хейзенберговского представления:
	\be
			\dot{\hat A} = [\hat{H}; \hat{A}];
	\ee

	Вернемся к уравнению свободной частици Дирака - например там ьбудет коммутировать импульс с гамильтонианом,
	значит импульс будет сохраняться. Но также будет сохраняться и момент импульса в силу изотропности 
	пространства. Соответствуют ди оператору $\hbar \hat{\vec l} = [\hat{\vec r} \times \hat{\vec p}]$ -
	сохраняющаяся величина доля свободной частицы? Рассммотрим произвольную компоненту $z$:
	\be
		\hbar \dot{\hat{l}}_z = \hbar (\hat{H} \hat{l}_z - \hat{l}_z \hat{H}) = 
		i \hbar c (\hat{a}_y \hat{p}_x - \hat{a}_x \hat{p}_y) \neq 0;
	\ee

	И аналогично для $x,~y$ компонент. Отсюда следует, что орибитальный момент - не является интегралом движения.

	Но также ясно, что какой-то момент, назовём его полным - будет сохраняться. Определим его как:
	\be
			\hat{j}_z = \hat{l}_z + \hat{\square}_z;
	\ee

	Аналогично будет и для $x,~y$. Мы хотим найти эту неизвестную величину. Чтобы 
	полный момент сохранялся, нужно, чтобы коммутатор гамильтониана с этой добавкой был равен:
	\be
			[\hat{H}; \hat{\square}_z] = ic \hbar (\hat{a}_x \hat{p}_y - \hat{a}_y \hat{p}_x);
	\ee

	Можно убедиться, что это удовлетворятся когда:
	\be
		\hat{\square}_z = \frac{1}{2} 
			\begin{bmatrix}
				\hat{\sigma}_z	&	0\\
				0	&	\hat{\sigma}_z
			\end{bmatrix} = \hat{S}_z;
	\ee

	Аналогично (тоже диагональные матрицы такого же вида) для оставшихся координат $x,~y$.
	При этом гамильтониан оказался одновременно релятивистским и линейным по импульсам и времени.

	Таким образом мы получаем $\hat{S}_x,~\hat{S}_y,~\hat{S}_x$ - оператиоры спина, действующие 
	в проостранстве биспиноров. Таким образом оператор полного момента для частицы - сумма
	орбитального и собственного момента импульса:
	\be
			\hat{\vec j} = \hat{\vec l} + \hat{\vec S};
	\ee

	Что мы с этого имеем? Как это влиет на классическую квантовую механику. Решим уравнение Дирака
	для свободной частицы, подставив вид плоской волны:
	\be
			\psi_i = \psi_{i0} \exp{-i\frac{Et}{\hbar}} \exp{i \frac{\vec p \cdot \vec r}{\hbar}};
	\ee

	Таким образом получим систему алгебраических уравнений:
	\be
			E\psi_{10} = c(p_x - i p_y) \psi_{40} + c p_z \psi_{30} + mc^2 \psi_{10};
	\ee
	\be
			E\psi_{20} = c(p_x + i p_y) \psi_{30} - c p_z \psi_{40} + mc^2 \psi_{20};
	\ee
	\be
			E\psi_{30} = c(p_x - i p_y) \psi_{20} + c px_z \psi_{10} - mc^2 \psi_{30};
	\ee
	\be
			E\psi_{40} = c(p_x + i p_y) \psi_{10} - c px_z \psi_{20} - mc^2 \psi_{40};
	\ee

	Для наличия решения такой системы уравнений мы должны приравнять нулю детерминант - 
	по сути мы получим связь между импульсами и энергией.

	\be
			E^2 = c^2 \vec{p}^2 + m^2 c^4;
	\ee

	И формально у нас получится неоднозначная связь энергии и импульса.
	\be
			E_{+} = \sqrt{c^2 \vec{p}^2 + m^2 c^4};
	\ee
	\be
			E_{-} = -\sqrt{c^2 \vec{p}^2 + m^2 c^4};
	\ee

	Если перейти к нерелятивистскому пределу $p \ll mc$, то получим:
	\be
			E_{+} \approx mc^2,~E_{-} = -mc^2;
	\ee

	А что делают дальше? Можно подставить это в уравнения и получить чуть упрощенные уравнения.
	Тогда в результате такого действия в нерелятивистком пределе получим:
	\be
		\Phi_{+} = 
			\begin{bmatrix}
				\psi_1\\
				\psi_2\\
				O(v/c)\\
				O(v/c)
			\end{bmatrix};
	\ee

	И аналогично для отрицательной энергии:
	\be
		\Phi_{+} = 
			\begin{bmatrix}
				O(v/c)\\
				O(v/c)\\
				\psi_3\\
				\psi_4
			\end{bmatrix};
	\ee

	Тогда получим что-то вроде предела:
	\be
		\Phi_{+} \rightarrow \begin{bmatrix} \psi_1\\ \psi_2\end{bmatrix};
	\ee
	\be
		\Phi_{-} \rightarrow \begin{bmatrix} \psi_3\\ \psi_4\end{bmatrix};
	\ee

	И что же с этим делать? Наличие частиц с отрицательными энергиями соответствует тому,
	что есть какие-то античастицы - или отсутствие нормальных частиц.

	Представим, что все состояние, имеющие отрицательную энергию - заняты. А у нас свой 
	мир с частицами положительной энергии. И у нас своя жизнь, а у отрицательных - своя.
	Тогда можем ли мы вытащить эти частицы? Можем - эффектом Клейна или с помощью двух 
	фотонов - там могут быть проблемы с сохранением импульса.

	\begin{tcolorbox}
		\textbf{Задание:} Каким образом можно вытащить электрон с нижних состояний?
	\end{tcolorbox}

	\mybox{
		\textbf{По чужим записям!}

		\section{Релятивистские взаимодействия}

		При переходе в нерелятивистский предел 
		у нас получается уравнение Паули:
		\be
			\hat H = \frac{1}{2m} 
				\big(\hat{\vec p} - \frac{e}{c}\vec A \big)^2
				+ e \phi - \underbrace{\frac{e\hbar}{2mc} \hat{\vec \sigma}}_\text{Почти магнитный момент}
				\vec{\mathcal H} = \hat{H}_\text{P};
		\ee

		А что будет в следующем приближении?
		\be
			\hat H = \hat{H}_\text{P} - \frac{1}{8m^3c^2}
				\big(\hat{\vec p} - \frac{e}{c}\vec A \big)^4
				+ \frac{e\hbar^2}{8m^2 c}\grad \phi - 
				\frac{e \hbar}{4^2 m^2 c^2} \hat{\vec \sigma}
				\Big[\vec E \times \big(\hat{\vec p} - \frac{e}{c}\vec A \big)\Big] 
				+ \hdots;
		\ee

		В случае наличия лишь электростатического поля:
		\be
			\label{eq49}
			\hat H = \frac{\hat{\vec p}^2}{2m} + e \phi - 
				\frac{\hat{\vec p}^4}{8m^2c^2} + 
				\frac{e \hbar^2}{8m^2 c^2} \grad \phi - 
				\underbrace{\frac{e \hbar}{4^2 m^2 c^2} \hat{\vec \sigma}
				\Big[\vec E \times \hat{\vec p}\Big]}_\text{Спин-орбитальное взаимодействие};
		\ee	

		Пусть у нас есть центрально-симметричное поле:
		\be
			\vec E = - \grad \phi = - \frac{\vec r}{r} \frac{d\Phi(r)}{dr};
		\ee

		В таком случае член спин-орбитального взаимодействия:
		\be
			\label{eq50}
			-\frac{e \hbar}{4^2 m^2 c^2} \hat{\vec \sigma}
			\Big[\vec E \times \hat{\vec p}\Big] = 
			\frac{e \hbar}{4^2 m^2 c^2}
			\Big[\frac{\vec r}{r} \frac{d\Phi(r)}{dr} \times \hat{\vec p}\Big] =
			\frac{\hbar}{4^2 m^2 c^2} \frac{1}{r} \frac{dU}{dr} 
			[\vec r \times \hat{\vec p}] \cdot \hat{\vec \sigma} = 
			\frac{\hbar^2}{4m^2 c^2} \frac{1}{r} \frac{dU}{dr} 
			\cdot (\hat{\vec l} \cdot \hat{\vec \sigma});
		\ee


		Оказываются запутанными операторы орбитального момента и спина.
		Этот член описывает как бы взаимодействие спинового и орбитального движения.

		Рассмотрим систему 2-х электронов и учтём взаимодействие
		в релятивистском приближении с точностью $(\frac{v}{c})^2$:
		\be
			\label{eq51}
			\hat{H} = \frac{\hat{\vec p}_1^2 + \hat{\vec p}_2^2}{2m}
			- \frac{\hat{\vec p}_1^2 + \hat{\vec p}_2^2}{8m^3c^2}
			+ U(\hat{\vec p}_1, \hat{\vec p}_2), \vec r);
		\ee

		Здесь $\vec r = \vec{r}_1 - \vec{r}_2$, а "потенциал":
		\begin{multline}
			\label{eq52}
			U(\hat{\vec p}_1, \hat{\vec p}_2), \vec r) = 
				\underbrace{\frac{e^2}{r}}_\text{I} - \pi \big(\frac{e\hbar}{mc}\big)^2 \delta(\vec r)
				- \underbrace{\frac{e^2}{2m^2 c^2 r} \left(\hat{\vec p}_1\hat{\vec p}_2
				+\frac{\vec r(\vec r \cdot \hat{\vec p}_1)\hat{\vec p}_2}{r^2}\right)}_\text{II}\\
				+ \underbrace{\frac{e^2 \hbar}{4m^2c^2 r^3}
				\left\{-(\hat{\vec \sigma}_1 + 2 \hat{\vec \sigma}_2)
				\cdot [\vec r \times \hat{\vec p}_1]
				+ (2\hat{\vec \sigma}_1 + \hat{\vec \sigma}_2)
				\cdot [\vec r \times \hat{\vec p}_2]\right\}}_\text{III}\\
				+ \frac{1}{4} \left(\frac{e\hbar}{mc}\right)^2
				\left\{\underbrace{\frac{\hat{\vec \sigma}_1 \cdot \hat{\vec \sigma}_2}{r^3}
				- \frac{3(\hat{\vec \sigma}_1\cdot \vec r)
				(\hat{\vec \sigma}_2\cdot \vec r)}{r^5}}_\text{IV}
				-\underbrace{\frac{8\pi}{3} (\hat{\vec \sigma}_1 \cdot \hat{\vec \sigma}_2) 
				\delta(\vec r)}_\text{V}\right\};
		\end{multline}
	}

	\mybox{
		Здесь некоторые члены поддаются интерпретации:
		\begin{enumerate}[I]
			\item Взаимодействие по Кулону
			\item Взаимодействие движужихся заряженных частиц
			\item Взаимодействие витка с током и магнита
			\item Взаимодействие 2-х магнитов
			\item Близкодействующее взаимодействие
		\end{enumerate}

		\begin{tcolorbox}
			\textbf{Задача:} Попробовать получить интерпретацию
			последнего V члена в рамках классической физики.
		\end{tcolorbox}

		Взаимодействие IV типа орбита-орбита происходит как с собственной,
		так и с чужой орбитой. Здесь все действует в пространстве
		простых спиноров.

		Обратимся к рассмотрению именно момента электронной оболочки
		атома, так как магнитный момент ядра мал. Отвлеччёмся
		от слабых релятивистских взаимодействий между электронами и с ядром.
		Оказывается, что кулоновское взаимодействие одного электрона
		с ядром и остальными электронами можно рассмотреть как
		взаимодействие с неким самосогласованным полем, которое вдобавок
		сфериччески симметрично. При этои состояние отдельного электрона
		в этом приближении можно характеризовать квантовыми числами $n,~l,~m,~S_z$.
		
		Так как самосогласованное поле не кулоновское - энергия отдельного
		электрона зависит как от $l$, так и от $m$. Поэтому корректно
		указывать все 4-е квантовых числа. Отвлекаясь от релятивистских эффектов
		высшего порядка можно считать, ччто сохраняется полный момент электронной оболочки,
		орбитальный момент и спин электрона.

		Сохравнение каждого из этих векторов имеет тот смысл, что возможны
		состояния, в которых одновременно измеримы модули каждого вектора и его
		проекции на выделенную ось.

	}
\end{document}
