%!TEX encoding = UTF-8 Unicode
\documentclass[a4paper, 14pt, russian]{article}
\usepackage[a4paper]{geometry}
\usepackage[T2A]{fontenc}
\usepackage[utf8]{inputenc}
\usepackage[russian]{babel}
\usepackage{physics}
\usepackage{tcolorbox}
\usepackage{hyperref}
\usepackage{fancyhdr}
\usepackage{indentfirst}
\usepackage{amssymb, amsmath, amsfonts}

\title{Физика магнитных явлений}
\author{Иосиф Давидович Токман}
\date{}

\newcommand{\be}{\begin{equation}}
\newcommand{\ee}{\end{equation}}
\newcommand{\bea}{\begin{equation}\begin{array}}
\newcommand{\eea}{\end{array}\end{equation}}

\newcommand{\pa}{\partial}
\newcommand{\rot}{\textbf{rot}~}
\renewcommand{\div}{\textbf{div}~}
\renewcommand{\grad}{\textbf{grad}~}

\setcounter{section}{14}

\begin{document}
	\maketitle

	\section{Магнитооптика}

	Как оптическое излучение взаимодействует со средой, имеющей спонтанную
	намагниченность. Мы рассмотрели такую среду - ферромагнетик.

	Начнём с того, что представляет электромагнитное поле, точнее - 
	с чем оно взаимодействует. Электрическое поле действует на заряды, а магнитное
	поле - в первую очередь на спиновые переменные.

	Мы рассматривали "узкую" область - воздействие электричческого поля.
	В первую оччередь это влиет на движение электронов. Заряды "чувствуют"
	намагниченность и спин-орбитальное взаимодействие. 

	Может такое случиться, то их спин "чувствует" поле упорядоченной 
	системы - мы эьто опускаем. Электроны, взаимодействующие с полем 
	называются оптическими. Поле действует на все электроны, но эти - 
	сильнее всего откликаются. Сами эти электроны взаимодействуют спином 
	с магнитным полем упорядоченной системы.

	Что бывает с электронами, с орбитальным движением, когда на 
	них действует поле - мы рассматривали. Мы рассматривапем чисто 
	квантовый случай с заданными наперёд дискретными уровнями.
	У нас есть система 4-ёх уровней, а их волновые функции считаем базовыми.

	В классике - мы что делаем? Получаем уравнения на функцию распределения.
	А у нас получается то же самое, но для матрицы плотности. С её помощью
	можно вычислить все квантовомеханические средние.

	В каких условиях мы не можем описать систему волновой функции? Например,
	когда есть классический силовой центр, и волновая функция в начале. Тогда 
	мы можем всегда описать при помощи волновых функций. А если у нас есть 
	квантовый источник, тогда происходит запутывание. Если мы следим
	за переменными только своей системы - тогда у нас не получится описать. 
	Работает всё хорошо только для глобальной волновой функции.

	Рассмотрим систему из 4-ёх уровней в термостате. Вероятности обнаружения
	ччастицы на уровня: $P_i \propto \frac{1}{z} \exp(-\frac{E_i}{T})$.
	Вместо того, чтобы заморачиваться с этим, можнл ввести матрицу плотности,
	которая удобнее - можно просто вычислять средние.

	Мы посчитали матрицу плотности в первом порядке возмущений. Оператор
	возмущения - дипольный момент.

	Рассмотрим предельный случай $T \rightarrow 0$. Тогда у нас останется 
	по-сути гибсовского распределение. $\rho_{11}^{(0)} \rightarrow 1$, в таком
	случае электрический дипольный момент выходит равным:
	\be
		\label{eq196}
		\langle d_{x}(t) \rangle = \frac{2}{\hbar} d^2 \left\{
			\frac{\omega_{21}}{\omega_{21}^2 - \omega^2} + 
			\frac{\omega_{41}}{\omega_{41}^2 - \omega^2}\right\}
			E_{0x} e^{-i\omega t};
	\ee


	Смотрим на картинку с переходами. Введём $\omega_0 = \frac{E_3 - E_1}{\hbar}$ - 
	"собственная частота"  атома, соответствующая переходу между основным состоянием и 
	средним уроавнем триплета. Аналогично: $\omega_{21} = \frac{E_2 - E_1}{\hbar},~
	\omega_{41} = \frac{E_4 - E_1}{\hbar}$. Также введём аналог Лорморовской частоты:
	$\omega_L = \frac{As}{\hbar}$. В этих обозначениях получим:
	\be
		\label{eq197}
		\langle d_{x}(t) \rangle = \frac{4}{\hbar} d^2 
			\frac{\omega_0 \big((\omega_0^2 - \omega_L^2) - \omega^2\big)}
			{\big((\omega_0^2 - \omega_L^2) - \omega^2\big)
			\big((\omega_0^2 + \omega_L^2) - \omega^2\big)} E_{0x} e^{-i\omega t};
	\ee

	Здесь можно увидеть аналог классического тензора поляризуемости:
	\be
		\label{eq198}
		\alpha_{xx} = = \frac{4}{\hbar} d^2 
			\frac{\omega_0 \big((\omega_0^2 - \omega_L^2) - \omega^2\big)}
			{\big((\omega_0^2 - \omega_L^2) - \omega^2\big)
			\big((\omega_0^2 + \omega_L^2) - \omega^2\big)};
	\ee

	Но почему у нас отклик только по одной оси? $\alpha_{xx} = \alpha_{yy}$.
	Оччевидно, то существуют и недиагональные компоненты. В том же приближении
	и в отсутствии поглощения можно получить:
	\be
		\label{eq199}
		\langle d_{y}(t) \rangle = - i \frac{8d^2}{\hbar} 
			\frac{\omega_0 \omega_L \omega}
			{\big((\omega_0^2 - \omega_L^2) - \omega^2\big)
			\big((\omega_0^2 + \omega_L^2) - \omega^2\big)} E_{0x} e^{-i\omega t};
	\ee

	\be
		\label{eq200}
		\alpha_{xy} = - \alpha_{yx} = - i \frac{8d^2}{\hbar} 
			\frac{\omega_0 \omega_L \omega}
			{\big((\omega_0^2 - \omega_L^2) - \omega^2\big)
			\big((\omega_0^2 + \omega_L^2) - \omega^2\big)};
	\ee

	Это нечётная функция ларморовой ччастоты. Таким образом 
	это модель гироэлектрической среды.

	Вообще это достаточно традиционная процедура в электродинамике. 

	\section{Обращение времени и теорема Крамерса}

	Случай классической механики. Система описывается состоянием из координат 
	и импульсов $\vec p,~\vec q$. Исходное состояние $g$ - описывается теми же 
	обобщёными координатыми, но противоположными импульсами. такое состояние
	называется сопряжёным по времени6 по отношению к исходному. Уравнения Гамильтона:
	\be
		\label{eq201}
		\dot{\vec q} = \frac{\pa H}{\pa \vec p};
	\ee 
	\be
		\dot{\vec p} = - \frac{\pa H}{\pa \vec q};
	\ee

	Преобразование: $t \rightarrow - t,~ \vec p \rightarrow - \vec p,
	~\vec q \rightarrow \vec q$ - не меняет вид уравнений \ref{eq201}
	в случае, если:
	\be
		\label{eq202}
		H(\vec p, \vec q) = H(-\vec p, \vec q)
	\ee

	Это значит, что если $q(t),~p(t)$ - решения уравнений Гамильтона, то 
	$q(-t),~-p(-t)$ - тоже решение уравнения Гамильтона.

	Магнитное поле нарушает такой принцип. Чтобы восстановить обратимость 
	во времени - нужно будет поле направлять в обратную сторону.

	Если есть полностью изолированная система с магнитными частицами
	это всегда можно сделать.

	Рассмотрим случай квантовой механики и бесспиновой частицы.
	\be
		\label{eq203}
		\hat{H}(\hat{\vec r}, \hat{\vec p}) \psi(\vec r, t) = i \hbar 
			\frac{\pa \psi}{\pa t};
	\ee

	Рассмотрим оператор $\hat T$ - обращения во времени: замена $t \rightarrow - t$,
	комплексное сопряжение:
	\be
		\label{eq204}
		\hat{H}^{*}(\hat{\vec r}, \hat{\vec p}) \psi^{*} = -i\hbar \frac{\pa \psi^{*}}{\pa (-t)};
	\ee

	Из этого видно, что если:
	\be
		\label{eq205}
		\hat{H}^{*}(\hat{\vec r}, \hat{\vec p}) = \hat{H}(\vec r, -\vec p)
		 = \hat{H}(\vec r, \vec p);
	\ee

	И если $\psi(\vec r, t)$ - является решением, то и $\psi^{*} (\vec r, -t)$ - 
	будет тоже решением уравнения $\ref{eq204}$.

	Рассмотрим теперь случай частицы со спином $\frac{1}{2}$. Тогда 
	оператор Гамильтона будет выглядеть:
	\be
		\label{eq206}
		\hat{H}(\vec r, \vec p, \vec \sigma)
		\begin{bmatrix} \psi_1(\vec r, t)\\ \psi_2(\vec r, t)\end{bmatrix}
		 = i \hbar \frac{\pa}{\pa(t)}
		\begin{bmatrix} \psi_1(\vec r, t)\\ \psi_2(\vec r, t)\end{bmatrix};
	\ee

	Подействуем оператором $\hat T$, определённым для бесспиновой частицы. 
	\be
		\label{eq207}
		\hat{H}(\vec r, -\vec p, \sigma_x, -\sigma_y, \sigma_z)
		\begin{bmatrix} \psi_1^{*}(\vec r, -t)\\ \psi_2^{*}(\vec r, -t)\end{bmatrix}
		 = -i \hbar \frac{\pa}{\pa(-t)} 
		\begin{bmatrix} \psi_1^{*}(\vec r, -t)\\ \psi_2^{*}(\vec r, -t)\end{bmatrix};
	\ee

	Подействуем на \ref{eq207} оператором $i\hat{\sigma}_y$:
	\be
		(i\hat{\sigma}_y)\hat{H}(\vec r, -\vec p, \sigma_x, -\sigma_y, \sigma_z)
		\begin{bmatrix} \psi_1^{*}(\vec r, -t)\\ \psi_2^{*}(\vec r, -t)\end{bmatrix}
 		= -i \hbar \frac{\pa}{\pa(-t)} (i\hat{\sigma}_y)
		\begin{bmatrix} \psi_1^{*}(\vec r, -t)\\ \psi_2^{*}(\vec r, -t)\end{bmatrix};
	\ee

	А теперь будем действовать формально:
	\be
		\hat{H}(\vec r, -\vec p, -\sigma_x, -\sigma_y, -\sigma_z) (i\hat{\sigma}_y)
		\begin{bmatrix} \psi_1^{*}(\vec r, -t)\\ \psi_2^{*}(\vec r, -t)\end{bmatrix}
 		= -i \hbar \frac{\pa}{\pa(-t)} (i\hat{\sigma}_y)
		\begin{bmatrix} \psi_1^{*}(\vec r, -t)\\ \psi_2^{*}(\vec r, -t)\end{bmatrix};
	\ee
	Это можно переписать более кратко:
	\be
		\label{eq208}
		\hat{H}(\vec r, -\vec p, -\vec{\sigma}) (i\hat{\sigma}_y)
		\begin{bmatrix} \psi_1^{*}(\vec r, -t)\\ \psi_2^{*}(\vec r, -t)\end{bmatrix}
 		= -i \hbar \frac{\pa}{\pa(-t)} (i\hat{\sigma}_y)
		\begin{bmatrix} \psi_1^{*}(\vec r, -t)\\ \psi_2^{*}(\vec r, -t)\end{bmatrix};
	\ee

	Теперь по аналогии с классчисеским случаем потребуем:
	\be
		\label{eq209}
		\hat{H}(\vec r, -\vec p, -\vec \sigma) = \hat{H}(\vec r, \vec p, \vec \sigma);
	\ee

	В таком случае мы получим, что волновые функции $i\hat{\sigma}_y)
	\begin{bmatrix} \psi_1^{*}(\vec r, -t)\\ \psi_2^{*}(\vec r, -t)\end{bmatrix}$ - тоже
	являются решением системы.

	Имеет смысл ввести оператор обращения времени для частицы с таким спином $\hat{T}'$:
	\be
		\label{eq211}
		\hat{T}' \begin{bmatrix} \psi_1(\vec r, t)\\ \psi_2(\vec r, t)\end{bmatrix}
			= (i \hat{\sigma}_y) \hat{T} \begin{bmatrix} \psi_1(\vec r, t)\\ \psi_2(\vec r, t)\end{bmatrix}
			= \begin{bmatrix} \psi_2^{*}(\vec r, -t)\\ -\psi_1^{*}(\vec r, -t)\end{bmatrix};
	\ee

	Ситуация \ref{eq209} соответствует: $\hat{T}' \hat{H} = \hat{H} \hat{T}'$. 
	А из  \ref{eq210}, \ref{eq211} следует, что:
	\be
		\label{eq213}
		\hat{T}' \begin{bmatrix} \psi_1(\vec r, t)\\ \psi_2(\vec r, t)\end{bmatrix}
		=  \begin{bmatrix} \psi_2^{*}(\vec r, -t)\\ -\psi_1^{*}(\vec r, -t)\end{bmatrix};
	\ee
	 
	-  есть решение уравнения Шредингера. 

	\begin{tcolorbox}
		Воздействие оператора $\hat{T}'$ - даёт другую функцию или исходную (с точностью до
		постоянного множителя)?
	\end{tcolorbox}

	Пусть это так, тогда:
	\be
		\label{eq214}
		\hat{T}' \Phi = C \Phi;
	\ee

	Применим второй раз и получем:
	\be
		\label{eq215}
		\hat{T}' \hat{T}' \Phi = \hat{T}' C \Phi = C^{*} C \Phi = \abs{C}^2 \Phi;
	\ee


	Но, выполняя требования \ref{eq213} получим:
	\be
		\label{eq216}
		\hat{T}' \hat{T}' \Phi = - \Phi;
	\ee

	Таким образом это не совместные уравнения, это значит, что если 
	$\Phi$ - решение уравнения Шредингера, то $\hat{T}' \hat{T}' \Phi$ - 
	толже решение, но:
	\be
		\label{eq217}
		\hat{T}' \Phi \neq C \Phi;
	\ee

	Тогда одному энергетическому уровню принадлежат \textit{как минимум}
	два состояния:
	\be
		\label{eq218}
		\hat H \Phi = E \Phi;
	\ee 

	И тогда:
	\be
		\label{eq219}
		\hat H \hat{T}' \Phi = E \hat{T}' \Phi;
	\ee 

	Значит есть как минимум двухкратное вырождение. Наличие этого вырождения
	составляет предмет \textit{теоремы Крамерса}. 

	Можно сформулировать так: Если есть спиновая частица,
	движущаяся в таком гамильтониане, то вырождение это никак не изменить.

	Но вот добавки, не обладающие таким свойством - испортят эту смимметрию, но если
	мы включим в рассмотрение и источники поля, то всё снова станет прекрасно. 

	Рассмотрим случай системы частиц со спином  $\frac{1}{2}$, если есть
	симметрия, рассмотренная в прошлом и оно действует для изолированной системы и 
	для такой же системы во внешнем электрическом поле (внешнее магнитное поле отсутствует),
	то если система состоит из нечётного числа электронов - имеет место Крамерсово
	вырождение.

	В силу того что волновая функция - сумма произведений:
	\be
		\hat{T}' \hat{T}' \psi^N = - \psi^N,~N = 2k + 1;
	\ee
	\be
		\hat{T}' \hat{T}' \psi^N = \psi^N,~N = 2k;
	\ee

	$\Phi^N$- N электронная волновая функция частиц со спином $\frac{1}{2}$.


\end{document} 
