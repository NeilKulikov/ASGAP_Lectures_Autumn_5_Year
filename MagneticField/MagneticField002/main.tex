%!TEX encoding = UTF-8 Unicode
\documentclass[a4paper, 14pt, russian]{article}
\usepackage[a4paper]{geometry}
\usepackage[T2A]{fontenc}
\usepackage[utf8]{inputenc}
\usepackage[russian]{babel}
\usepackage{physics}
\usepackage{tcolorbox}
\usepackage{hyperref}
\usepackage{fancyhdr}
\usepackage{indentfirst}
\usepackage{amssymb, amsmath, amsfonts}

\title{Физика магнитных явлений}
\author{Иосиф Давидович Токман}
\date{}

\newcommand{\be}{\begin{equation}}
\newcommand{\ee}{\end{equation}}
\newcommand{\bea}{\begin{equation}\begin{array}}
\newcommand{\eea}{\end{array}\end{equation}}

\newcommand{\pa}{\partial}
\newcommand{\rot}{\textbf{rot}~}
\renewcommand{\div}{\textbf{div}~}
\renewcommand{\grad}{\textbf{grad}~}

\setcounter{equation}{11}
\setcounter{section}{2}

\begin{document}
	\maketitle

	Оператор момента импульса множества частиц, будет суммой операторов, для каждой:
	\be
		\hat{\vec L}  = \sum_i \hat{\vec l}_i;
	\ee

	При этом момент импульса разных частиц коммутирует -  в силу зависимости от разных координат:

	\be
		[\hat{L}_x; \hat{L}_y] = i \hat{L}_z;
	\ee

	И аналогично при циклических перестановках:
	\be
		[\hat{L}_y; \hat{L}_z] = i \hat{L}_x;~[\hat{L}_z; \hat{L}_x] = i \hat{L}_y;
	\ee

	Можно показать, что справедливы все коммутационные соотношения для одной частицы,
	с заменой $\hat{\vec l} \rightarrow \hat{\vec L}$. Очевидно, что существунт
	такая $\psi_{L,M}$, что:
	\be
		\hat{\vec L}^2 \psi_{LM} = L(L+1)\psi_{LM};
	\ee
	\be
		\hat{L}_z \psi_{LM} = M\psi_{LM};
	\ee

	При этом могут быть значения:$M \in [-L, L] \in \mathbb{Z}$. И полностью аналогично Можно
	заменять $m \rightarrow M$.

	\section{Спин}

	Как показывает опытзадание волновой функции частицы,
	как описания ее положения в пространстве не исчерпывает все
	степени свободы частицы. При этом рекчь может идти как о сложной частице (ядре),
	так и об элементарной частице - например электроне.

	Спин это фактически - внутренний момент частицы. В разделе 1 мы стартовали с того, что 
	определили:

	\be
		\hbar \hat{\vec l} = [\hat{\vec r} \times \hat{\vec p}];
	\ee

	Оператор момента импульса, действующий на координаты частицы. О спине хаговорили после эксперимента 
	Штерна-Герлаха - пропускали пучок электронов или других спиновых частиц через неоджнородное магнитное 
	поле и наблюдали его расщепление.

	Значит то что у нас летело обладало каким то магнитным моментом. При том это наблюдалось даже
	в электронейтральных атомах - что значило, что момент внтуренний, а не орбитальный.

	Дальше идет сухая теория, которая описывает два состояния при одном орбитальном моменте. 
	Введем формально операторы $\hat{\vec s}_x,~\hat{\vec s}_y,~\hat{\vec s}_z$. И скажем что у них джолжны быть
	такие же коммутационные соотношения, что и у обычных проекций момента импульса:
	\be
		[\hat{s}_x;\hat{s}_y] = i\hat{s}_z;
	\ee

	И аналогично с циклической перестановкой. однако обычнор все это вводится через матрицы Паули:
	\be
		\hat{s}_i = \frac{1}{2} \hat{\sigma}_i;
	\ee

	И для него такие же коммутационные соотношения с циклическими перестановками:
	\be
		[\hat{\sigma}_x;\hat{\sigma}_y] = i 2 \hat{\sigma}_z;
	\ee

	А на что должны действовать такие операторы? В данном случае, поскольку у нас всего 
	2 состояния - используются волновые функции в виде векторов (спиноров) и
	операторы в виде матриц $2 \times 2$.

	Т.е. операторы $\hat{s}~(\hat \sigma)$ действуют в каком-то неизвестном нам 
	ёпространстве функций. Тогда совершенно формально, в случае, если у нас всего два 
	состояния то и функции размером $2 \times 1$.

	Пусть базисные функции таковы, что матрица $\hat{\sigma}_z$ - диагональна, а её
	собственные значения $\pm 1$. Тогда получится:
	\be
		\hat{\sigma}_z = 
			\begin{bmatrix}
				1	&	0\\
				0	&	-1
			\end{bmatrix};
	\ee

	Естественно потребовать, чтобы собственные значения $\hat{\sigma}_x,~\hat{\sigma}_y$
	аналогично равнялись $\pm 1$. Таким образом мы получим:
	\be
		\hat{\sigma}_x^2 = \hat{\sigma}_y^2 = \hat{\sigma}_z^2 = \hat I;
	\ee

	Но тогда из коммутационнных соотношений следует:
	\be
		\hat{\sigma}_x \hat{\sigma}_y = - \hat{\sigma}_y \hat{\sigma}_x = i \hat{\sigma}_z;
	\ee
	\be
		\hat{\sigma}_y \hat{\sigma}_z = - \hat{\sigma}_z \hat{\sigma}_y = i \hat{\sigma}_x;
	\ee
	\be
		\hat{\sigma}_z \hat{\sigma}_x = - \hat{\sigma}_x \hat{\sigma}_z = i \hat{\sigma}_z;
	\ee

	Тогда вид для остальных матриц в силу эрмитовости:
	\be
		\hat{\sigma}_x = 
			\begin{bmatrix}
				a_{11}	&	a_{12}\\
				a_{12}^{*}	&	a_{22}
			\end{bmatrix};
	\ee
	\be
		\hat{\sigma}_y = 
			\begin{bmatrix}
				b_{11}	&	b_{12}\\
				b_{12}^{*}	&	b_{22}
			\end{bmatrix};
	\ee

	Но также можно показать:
	\be
		\hat{\sigma}_x = 
			\begin{bmatrix}
				0	&	a_{12}\\
				a_{12}^{*}	&	0
			\end{bmatrix};
	\ee

	Если возвести в квадрат:
	\be
		\hat{\sigma}_x^2 = 
			\begin{bmatrix}
				a_{12} a_{12}^{*}	&	0\\
				0	&	a_{12}^{*} a_{12}
			\end{bmatrix};
	\ee

	И из условия на квадраты имеем:
	\be
			a_{12} = e^{i\phi};
	\ee

	Иными словами:
	\be
		\hat{\sigma}_x = 
			\begin{bmatrix}
				0	&	e^{i\phi}\\
				e^{-i\phi}	&	0
			\end{bmatrix};
	\ee

	По аналогии:
	\be
		\hat{\sigma}_y = 
			\begin{bmatrix}
				0	&	e^{i\beta}\\
				e^{-i\beta}	&	0
			\end{bmatrix};
	\ee

	А из условия на их умножение будем иметь:
	\be
		e^{i(\phi - \beta)} = e^{-i(\phi - \beta)} = i;
	\ee

	Откуда с неизбежностью сдедует: $\phi = 0,~\beta = -i\frac{\pi}{2}$.
	
	Тогда получим:
	\be
		\hat{\sigma}_x = 
			\begin{bmatrix}
				0	&	1\\
				1	&	0
			\end{bmatrix};
	\ee
	\be
		\hat{\sigma}_y = 
			\begin{bmatrix}
				0	&	-i\\
				i	&	0
			\end{bmatrix};
	\ee

	Надо напомнить, что настоящие операторы спина:
	\be
		\hat{s}_i = \frac{1}{2} \hat{\sigma}_i;	
	\ee

	При этом собственные числа для спина будут равны $\pm \frac{1}{2}$,
	а квадрат спина:
	\be
		\hat{\vec s}^2 = \hat{s}_x^2 + \hat{s}_y^2 + \hat{s}_z^2 = \frac{3}{4};
	\ee

	Что полностью соответствует формуле для оператора орбитального момента. Почему
	мы используем такой архаичный подход? Мы используем некоторую не единственность 
	и собственный выбор. Однако этот выбор устоявшийся и не снижающий общности.

	Таким образом существуют 2 состояния, в которых проекция спина на определёную 
	остальных равна $\pm \frac{1}{2}$. В соответствии с матричным формализмом,
	операторы, выражаемые матрицами $2 \times 2$ действуют в фпространстве функций:
	\be
		\Phi = \begin{bmatrix} \psi_1 \\ \psi_2 \end{bmatrix};
	\ee
	\be
		\Phi^{*} = [ \psi_1^{*} ; \psi_2^{*} ];
	\ee

	И можно получить, что такой спинор - собственная функция оператора $\hat{s}_z$.
	Тогда функция соответствующая проекции спина $\frac{1}{2}$:
	\be
		\Phi_{1/2} = \begin{bmatrix} 1 \\ 0 \end{bmatrix};
	\ee

	А функция, соотвествующая состоянию с $-\frac{1}{2}$:
	\be
		\Phi_{-1/2} = \begin{bmatrix} 0 \\ 1 \end{bmatrix};
	\ee

	Большая часть квантовой механики может быть проиллюстрированнна такими спинорами.
	Но все это лишь часть более общего квантово-механического формализма. 
	Твердотельный магнетизм практически целиком оказывается связан именно со спином.

	\section{Преобразование спина при преобразовании системы координат.}

	Рассмотрим просто поворот для начала. Для начала посмотрим на пролизвольную 
	квантовую систему и мы её описываем в какой-то конкретной системе координат.
	Мы можем вращать саму физическую систему или вращать выбранную систему координат.

	Пусть в определённой системе координат $x,y,z$ волновая функция описывается 
	спинором $\Phi$. А в системе координат $x',y',z'$ это же состояние описывется спинором:
	$\Phi'$ Для простоты рассмотрим случай, где переход от одной системы к другой осуществляется
	поворотом вокруг оси $z$ на угол $\gamma$.

	При этом связь координат:
	\be
		\begin{bmatrix} x\\ y \end{bmatrix} = 
			\begin{bmatrix}
				\cos \gamma & - \sin \gamma\\
				\sin \gamma & \cos \gamma
			\end{bmatrix} 
		\begin{bmatrix} x' \\ y' \end{bmatrix};
	\ee

	Т.е. такому преобразованию соответствует марица преобразования спинора.
	\be
		\Phi ' = \hat{T}_{z} (\gamma) \Phi;
	\ee

	Тогда спиноры преобразуются (аналогично т для других координат):
	\be
		\hat{s}_x ' = \hat{T}_{z} (\gamma) \hat{s}_x \hat{T}_{z}^{-1} (\gamma);
	\ee

	Здесь $\hat{s}_{x,y,z}'$ - операторы, действующие в штрихованнной системе координат.
	Это операторы проекции спина на старые оси координат $x,y,z$, наприсаннные в новом 
	представлении, связанном с $x',y',z'$.
	
	А чем являются эти операторы ещё? Это операторы, соответстующие некоторым векторам $\hat{\vec S}$. 
	Тогда они должны преобразовываться по тем же законам.
	\be
		\begin{bmatrix} \hat{s}_x\\ \hat{s}_y \\ \hat{s}_z \end{bmatrix} = 
			\begin{bmatrix}
				\cos \gamma & - \sin \gamma	& 0\\
				\sin \gamma & \cos \gamma	& 0\\
				0			&				& 1
			\end{bmatrix} 
		\begin{bmatrix} \hat{s}_x'\\ \hat{s}_y' \\ \hat{s}_z' \end{bmatrix};
	\ee

	Здесь штрихованные операторы - соответствуют проекциям на оси $x',y',z'$,
	взятые в одном и том же представлении, связанном с $x,y,z$.

	Но такая связь справедлива в любом представлнении. 
	\be
		\begin{bmatrix} \hat{s}_x '\\ \hat{s}_y '\\ \hat{s}_z '\end{bmatrix} = 
			\begin{bmatrix}
				\cos \gamma & - \sin \gamma	& 0\\
				\sin \gamma & \cos \gamma	& 0\\
				0			&				& 1
			\end{bmatrix} 
		\begin{bmatrix} \hat{s}_{x'}'\\ \hat{s}_{y'}' \\ \hat{s}_{z'}' \end{bmatrix};
	\ee

	Тогда это будет равно (в силу независимомти от представлений):
	\be
		\begin{bmatrix} \hat{s}_x '\\ \hat{s}_y '\\ \hat{s}_z '\end{bmatrix} = 
			\begin{bmatrix}
				\cos \gamma & - \sin \gamma	& 0\\
				\sin \gamma & \cos \gamma	& 0\\
				0			&				& 1
			\end{bmatrix} 
		\begin{bmatrix} \hat{s}_{x}\\ \hat{s}_{y} \\ \hat{s}_{z} \end{bmatrix};
	\ee

	Так как $\hat{s}_{x'}' = \hat{s}_{x}, \hat{s}_{y'}' = \hat{s}_{y}, \hat{s}_{z'}' = \hat{s}_{z}$.
	Тогда если сравнить это с тем, что мы писали преобразование через некоторые операторы поворота 
	$\hat{T}_z(\gamma)$. Поскольку мы выбрали $\hat{s}_z$ - диагональной, то и 
	$\hat{T}_z$ - диагональная (чтобы они коммутировали).

	Тогда для неё можно написать:
	\be
		\hat{T}_z = 
			\begin{bmatrix}
					a	&	0\\
					0	&	b
			\end{bmatrix};
	\ee

	Поскольку по определению и в силу унитарности у нас должно быть:
	\be
		\hat{T}_z \hat{T}_z^{-1} = \hat{I} = \hat{T}_z \hat{T}_z^\dagger;
	\ee

	В виде матричном виде:
	\be
		\hat{I} = 
			\begin{bmatrix}
				\abs{a}^2	&	0\\
				0	&	\abs{b}^2
			\end{bmatrix}
	\ee

	Отсюда можно видеть $\phi_1 - \phi_2 = \gamma$:
	\be
		\hat{T}_z = 
			\begin{bmatrix}
					e^{\phi_2}	&	0\\
					0	&	e^{i\phi_2}
			\end{bmatrix};
	\ee

	Поэтому можно выбрать симметрично:
	\be
		\hat{T}_z = 
			\begin{bmatrix}
					e^{-\gamma / 2}	&	0\\
					0	&	e^{-i\gamma / 2}
			\end{bmatrix};
	\ee

	Рассмотрим волновую функцию для двух частиц, 
	для каждой из которых волновая функция - простой спинор.
	Тогда полная волновая функция:
	\be
		\Phi_0(1,2) = \frac{1}{\sqrt 2} 
			\left(
				\begin{bmatrix} 1 \\ 0 \end{bmatrix}_1
				\begin{bmatrix} 0 \\ 1 \end{bmatrix}_2
					-
				\begin{bmatrix} 0 \\ 1 \end{bmatrix}_2
				\begin{bmatrix} 1 \\ 0 \end{bmatrix}_1
			\right);
	\ee

	Здесь индексы у спиноров - соответствуют частицам.
	Вообще полная волновая функция будет антисимметричной,
	относительно перестановок частиц. А операторы останутся теми же
	в силу своей аддитивности:
	\be
		\hat A = \sum_i \hat{A}_i;
	\ee

	Легко показать, что эта функция описывает состояние с полным спином 
	$\hat S = 0$. 
	\be
		\sum_i = (\hat{s}_{1i} + \hat{s}_{2i})^2 \Phi_0(1,2) = 0;
	\ee
	
	Очевидно, что при вращении сисетмы координат такая функция не должна изменяться.
	В виде формулы, где $\hat T$ - произвольное вращение:
	\be
		\hat{T} \Phi_0(1,2) = \Phi_0(1,2);
	\ee

	А сам оператор поворота запишется как:
	\be
		\hat T = \begin{bmatrix} a	& b\\	c	&	d \end{bmatrix};
	\ee

	Но каждый из электронов то может изменяться? Что с этим всем делать?

	Как мы можем совместить эти два условия? Напишем:
	\be
		\begin{bmatrix} a	& b\\	c	&	d \end{bmatrix}
		\left(
			\begin{bmatrix} 1 \\ 0 \end{bmatrix}_1
			\begin{bmatrix} 0 \\ 1 \end{bmatrix}_2
				-
			\begin{bmatrix} 0 \\ 1 \end{bmatrix}_2
			\begin{bmatrix} 1 \\ 0 \end{bmatrix}_1
		\right) = 
		\left(
				\begin{bmatrix} a \\ c \end{bmatrix}_1
				\begin{bmatrix} b \\ d \end{bmatrix}_2
					-
				\begin{bmatrix} b \\ d \end{bmatrix}_2
				\begin{bmatrix} a \\ c \end{bmatrix}_1
		\right) = 
		\left(
			\begin{bmatrix} 1 \\ 0 \end{bmatrix}_1
			\begin{bmatrix} 0 \\ 1 \end{bmatrix}_2
				-
			\begin{bmatrix} 0 \\ 1 \end{bmatrix}_2
			\begin{bmatrix} 1 \\ 0 \end{bmatrix}_1
		\right)
	\ee

	А последнее равенство выще из того, что мы хотим получить. 
	Отсюда следует, что:
	\be
		ad - bc = 1;
	\ee

	Сравнив матрицу и ее вид для вращения вокруг $z$ получим:
	\be
		e^{i(\phi_1 + \phi_2)}  =1;
	\ee

	Тогда например можно сделать так: $\phi_1 + \phi_2 = 0$.
	Так же как делали и в прошлый раз. 

	Вспомним про такую классную штуку как циклические координаты 
	и симметрии из теоретической механике. 
	Например импульс соответствует симметрии трансляции,
	а момент импульса - повороту вокруг оси, энергия - однородности времени.

	Тогда чтобы проверить сохранение проекции момента (пусть даже спина) - 
	нежно повернуть систему. Можно написать волновую функцию в повернутых 
	координатах при помощи оператора момента импульса.

	\begin{tcolorbox}
		\textbf{Задание}: залезть в ЛЛ. Посмотреть на то, как трансляция связана с 
		оператором момента импульса. Пользуясь этим знанием написать оператор 
		поворота, записанный через оператор момента импульса или спиновые операторы.
	\end{tcolorbox}

	Замечание: как пишется простой спинор? Обычно так:
	\be
		\begin{bmatrix} \psi_1(\vec r) \\ \psi_2(\vec r) \end{bmatrix}
			\neq \psi(\vec r) \begin{bmatrix} a \\ b \end{bmatrix};
	\ee

	Единственный вариант, когда можно так написать - если операторы пространственные 
	и спинорные действуют на разные координаты.

	\section{"Появление" спина.}

	Как спин вооьще влияет на уравнение? В первом приближении это уравнение Паули.
	Чуть более продвинутый предел - спиниорбитальное взаимодействие. На самом деле
	все это сидит в уравнении Дирака, а все, что мы наблюдаем - некоторые его
	нерелятивистские приближения.

	Уравнение Дирака для свободной частицы совпадает с уравнением Шредингера:
	\be
		i\hbar \pa_t psi = \hat{H} \psi;
	\ee

	А сам гамильтониан выглядит как:
	\be
		\hat{H} = c \hat{\vec \alpha} \hat{\vec p} = m c^2 \hat{\beta};
	\ee

	Масс покоя и просто масс не существует. Есть просто масса.

	Тогда получим:
	\be
		\hat{H} = c (\hat{\alpha}_x \hat{p}_x + \hat{\alpha}_y \hat{p}_y ) 
			+ \hat{\alpha}_z \hat{p}_z) + mc^2 \beta;
	\ee

	А если писать альфа- и бета-матрицы:
	\be
			\hat{\alpha}_i = 
				\begin{bmatrix} 
					0 & \hat{\sigma}_i\\
					\hat{\sigma}_i	& 0
				\end{bmatrix};
	\ee
	\be
			\hat{\beta}_i = 
				\begin{bmatrix} 
					\hat{I} & 0\\
					0	& -\hat{I}
				\end{bmatrix};
	\ee
\end{document}
