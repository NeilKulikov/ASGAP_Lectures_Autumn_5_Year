%!TEX encoding = UTF-8 Unicode
\documentclass[a4paper, 14pt, russian]{article}
\usepackage[a4paper]{geometry}
\usepackage[T2A]{fontenc}
\usepackage[utf8]{inputenc}
\usepackage[russian]{babel}
\usepackage{physics}
\usepackage{hyperref}
\usepackage{fancyhdr}
\usepackage{indentfirst}
\usepackage{amssymb, amsmath}

\title{Физика магнитных явлений}
\author{Иосиф Давидович Токман}
\date{}

\newcommand{\be}{\begin{equation}}
\newcommand{\ee}{\end{equation}}
\newcommand{\bea}{\begin{equation}\begin{array}}
\newcommand{\eea}{\end{array}\end{equation}}

\newcommand{\pa}{\partial}
\newcommand{\rot}{\textbf{rot}~}
\renewcommand{\div}{\textbf{div}~}
\renewcommand{\grad}{\textbf{grad}~}

\begin{document}
	\maketitle

	Лекции будут перемежаться с практическими занятиями и семинарами. 
	Будут домашние задания. Запланированно 18 лекций по 3 часа.

	На экзамене можно пользоваться чем угодно. Во время ответа будем 
	отвечать за каждую букву.

	Это самозванная наука: компиляция из электродинамики и квантовой механики.
	Что из магнетизма будем рассматривать? В основном - магнетизм твёрдого тела: 
	токи и спин. Спин органично получается в квантовой электродинамике. Типичный 
	представитель - уравнение Дирака. От этого никуда не деться на микроуровне.

	Начнем с квантовой механики. Первые лекции - будут экскурсом в неё, но с упором
	на магнитные явления.

	\section{Момент импульса. Орбитальное движение отдельной частицы.}

	Определяется момент импульса так:

	\be
		\hbar \hat{\vec l} = [\hat{\vec r} \times \hat{\vec p}];
	\ee

	В классической механике постоянной Планка нет. 

	Сам оператор $\hat{\vec l}$ действует в пространстве волновых функций.
	Из этого мы можем получить например матричные элементы оператора момента 
	импульса.

	Получим его правила коммутации:
	\be
		[\hat{r}_k;\hat{p}_l] = i \hbar \delta_{k,l};~k,l = x,y,z;
	\ee

	Из этих коммутационнных соотношений следуют соотношения и для самого оператора
	момента импульса ($l_{+} = l_x + i l_y,~l_{-} = l_x - i l_y$):
	\be
		[l_x; l_y] = i l_z;~[l_y; l_z] = i l_x;~[l_z; l_x] = i l_y;
	\ee
	\be
		[l_{+}; l_{-}] = 2l_z;~[l_z;l_{+}]= l_{+};~[l_z;l_{-}] = -l_{-};
	\ee
	\be
		[{\vec l}^2; l_i] = 0;~i=x,y,z;
	\ee

	Дополнительное соотношение:
	\be
		{\vec l}^2 = l_{-}l_{+} + l_z^2 + l_z = l_{+}l_{-} + l_z^2 - l_z;
	\ee

	Отсутствие коммутации операторов - эквивалентно тому, что мы не можем выбрать 
	систему собственных функций для них.

	Полезно знать связь в сферических координат:
	\bea
		l_z = - i \pa_\phi;\\
		l_\pm = e^{\pm i \phi} (\pm \pa_\theta + i \cot \theta \pa_\phi);\\
		l^2 = - \left(\frac{\pa_{\phi \phi}}{\sin^2 \theta} + 
			\frac{1}{\sin \theta} \pa_\theta (\sin\theta \pa_\theta)\right);
	\eea

	А как выглядят собственные функции? Начнем с оператора $l_z$:
	\be
		l_z \psi = -i \pa_\phi \psi;
	\ee

	Тогда собственная функция (из требования однозначности
	и непрерывности функции):
	\be
		\psi_m = e^{i m \phi} / \sqrt{2\pi};~ m\in \mathbb{Z};
	\ee

	Но собственная функция квадрата оператора момента импульса:
	\be
		\Psi = f(r;\theta) \cdot \psi_{l_z}(\phi);
	\ee

	Из уравнения на коммутаторы следует, что существуют состояния, 
	где одновременно две величины ${\vec l}^2,~l_z$ - могут быть
	определены. Что здесь значит одновременно? Пусть дано:
	\be
		\hat{\vec l}^2 \psi = {\vec l}^2 \psi;
	\ee

	А также можно подействовать на ту же функцию:
	\be
		\hat{l}_z \psi = l_z \psi;
	\ee

	Вообще волновая функция - свойство ансамбля измерений,
	а не отдельной частицы. Отдельная частица коллапсирует во
	что - то.

	Будем считать, что функция $\psi$ - собственная ${\vec l}^2$.
	Тогда получается:
	\be
		{\vec l}^2 - l_z^2 = l_x^2 + l_y^2 \ge 0;
	\ee

	И это значит, что должно быть минимальное значение $\hat l = l$,
	такое, что при заданном $l$ значения $l_z  = -l;\hdots; l$.
	Но из соотношения на коммутаторы следует:
	\be
		\hat{l}_z \hat{l}_{\pm} \Psi_m = (m+1) \hat{l}_{\pm} \Psi_m;
	\ee

	А также:
	\be
		\hat{l}_{+} \Psi_m = const \cdot \Psi_{m+1};\\
	\ee
	\be
		\hat{l}_{-} \Psi_m = const \cdot \Psi_{m-1};
	\ee

	Это значит, что операторы $\hat{l}_{\pm}$ действуют как повышение или
	понижения квантового числа $m$ на единицу.
	Однако действие повышающего оператора на $\psi_l$ сведется
	к занулению в силу ограниченности его собственных чисел:
	\be
		\hat{l}_{+} \psi_l = 0;
	\ee

	Отсюда можно видеть:
	\be
		l_{-} l_{+} \psi_l = ({\vec l}^2 - l_z^2 - l_z) \psi_l = 0;
	\ee

	Важно отметить, что $l,~\vec{l}$ - отличаются. Тогда получим:
	\be
		{\vec l}^2 \psi_l - l^2 \psi_l -l \psi_l = 0;
	\ee

	Это значит, что мы получили собственные числа оператора квадрата 
	момента импульса:
	\be
		{\vec l}^2 = l(l+1);
	\ee

	Удобно записать собственную функцию операторов ${\vec l}^2,~l_z$ в виде:
	\be
		\psi(r; \theta; \phi) = Y_{l,m}(\theta, \phi) f(r);
	\ee

	Тогда можно написать уравнение на это выражениe:
	\be
		{\vec l}^2 Y_{l,m}(\theta, \phi) f(r) = l(l + 1) Y_{l,m}(\theta, \phi) f(r);
	\ee

	Но и для оператора $l_z$ тоже будет собственной функцией:
	\be
		l_z Y_{l,m}(\theta, \phi) f(r) = m Y_{l,m}(\theta, \phi) f(r);
	\ee

	В силу эрмитовости оператора $\vec l$:
	\be
		\bra{Y_{l,m-1}} l_{-} \ket{Y_{l,m}} = \bra{Y_{l,m}} l_{+} \ket{Y_{l,m - 1}} {}^{*};
	\ee

	Откуда получаем важное следствие:
	\be
		\bra{Y_{l,m}} l_{+} \ket{Y_{l,m - 1}} = \bra{Y_{l,m-1}} l_{-} \ket{Y_{l,m}} = \sqrt{(l+m)(l - m + 1)};
	\ee

	А для других компонент мы получим:
	\be
		\bra{l, m} l_x \ket{l, m-1} = \bra{l,m-1} l_x \ket{l,m} = \frac{1}{2} \sqrt{(l+m)(l-m+1)};
	\ee
	\be
		\bra{l, m} l_y \ket{l, m-1} = - \bra{l,m-1} l_y \ket{l,m} = -\frac{i}{2} \sqrt{(l+m)(l-m+1)};
	\ee

	Также заметим, что соотношение на собственные функции оператора, получается и в явном виде.
	Можно показать, что $Y_{l,m}$ - так называемые сферические функции.

	Литература:
	\begin{itemize}
		\item Ландау, Лифшиц "Том 3. Нерелятивистская квантовая теория"
		\item Блохинцев "Основы квантовой механики"
		\item Херми "Лекции по квантовой механике"
		\item Кринчик "Физика магнетизма"
		\item Ванцовский "Магнетизм"
		\item Вдовин, Левич, Мямлин "Курс теоретической физики"
	\end{itemize}

\end{document}
