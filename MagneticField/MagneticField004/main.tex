%!TEX encoding = UTF-8 Unicode
\documentclass[a4paper, 14pt, russian]{article}
\usepackage[a4paper]{geometry}
\usepackage[T2A]{fontenc}
\usepackage[utf8]{inputenc}
\usepackage[russian]{babel}
\usepackage{physics}
\usepackage{tcolorbox}
\usepackage{hyperref}
\usepackage{fancyhdr}
\usepackage{indentfirst}
\usepackage{amssymb, amsmath, amsfonts}

\title{Физика магнитных явлений}
\author{Иосиф Давидович Токман}
\date{}

\newcommand{\be}{\begin{equation}}
\newcommand{\ee}{\end{equation}}
\newcommand{\bea}{\begin{equation}\begin{array}}
\newcommand{\eea}{\end{array}\end{equation}}

\newcommand{\pa}{\partial}
\newcommand{\rot}{\textbf{rot}~}
\renewcommand{\div}{\textbf{div}~}
\renewcommand{\grad}{\textbf{grad}~}

\setcounter{section}{7}

\begin{document}
	\maketitle

	\section{Атом}

	Атомы есть и их не нужно искуственно создавать. Поскольку они нас всюду окружают, то 
	к ним прикован пристальный интерес. Какое отношение это имеет к магнетизму?

	Какие предположения мы делали на счет его природы? Для начала можно предположить 
	отсутствие взаимодействия электронов друг-с другом и спиновое взаимодействие с ядром.

	Почему мы можем так сделать? Потому что можно на время пренебречь релятивистскими
	эффектами, которыми и является спин.

	Рассмотрение конкретных атомов начато с $Be$. Дальше можно рассмотреть 
	атом $Al(z = 13)$. У нас есть 3 квантовых числа:  $n,~l,~m,~s$.
	\begin{itemize}
		\item $1s^2$ - два электрона с разным спином (как и всегда),
		\item $2s^2$ - $n=2,~l=0$, 2 электрона,
		\item $2p^6$ - $n=2,~l=1$, 6 электронов,
		\item $3s^2$ - $n=3,~l=0$, 2 электрона,
		\item $3p$ - $n=3,~l=1,~n=1$, 1 электрон
	\end{itemize}

	Основной терм связан именно с внешним электроном ${}^2 P_{1/2}$.
	А вообще как определяется: ${}^{2s+1} (L)_{J}$, где $L=P,S,D,F,\hdots$
	орбитальное квантовое число, $S$ - спин, $J$ - полный момент.

	Другой пример $Fe(я = 26)$ - железо:
	\begin{itemize}
		\item $1s^2$,
		\item $2s^2$
		\item $2p^6$,
		\item $3s^2$,
		\item $2p^6$,
		\item $3d^6$,
		\item $4s^2$
	\end{itemize}

	Вопрос - почему мы, не заполнив $d$ орбиталь перешли на следующую $s$ орбиталь?
	Потому что это более энергетически выгодноя ситуация. Сейчас мы запишем
	соответствующий основной терм: ${}^5 D_4$, который соответствует 
	$s = 2,~L = 2,~J=4$. Этот трм отвечает самой низкой энергии.

	Рассмотренный выше подход: складываются в $\vec L$ - орбитиальные моменты электронов,
	в $\vec S$ - собственные моменты электронов, а полный момент $\vec J = \vec L + \vec S$
	соответствует связи Рассел-Сандерса, т.н. $LS$ связи. Такая связщь называется нормальной
	она законна лишь в приближении отсутствия релятивистских эффектов, т.е. когда 
	электро-статические взаимодействия существенно превышают релятивистские.

	А если бы мы написали $\hat{\vec J} = \hat{\vec L} + \hat{\vec S}$ - будет ли это точной
	записью взаимодействия? Нет, потому, что если у нас есть набор частиц, то оператор полного 
	орбитального и собственного момента будет:
	\be
		\hat{\vec L} = \sum_{i = 1}^{N} \hat{\vec l}_i;
	\ee

	\be
		\hat{\vec S} = \sum_{i = 1}^{N} \hat{\vec s}_i;
	\ee

	Но почему мы можем их складывать? Дело в том, что полная волновая функция системы частиц:
	$\Phi(\vec{r}_i,~\sigma_i,~\forall i)$. Если мы захотим ее представить через волновые функции 
	одночастичных состояний то получим какую-то символичную кашу:
	\be
		\Phi \sim \phi_{l_1,s_1} \cdot \phi_{l_2,s_2} \hdots;
	\ee

	Ни откуда не следует что эта функция будет собственной для наших операторов.
	Даже собственные числа одного электрона не могут быть получены.

	А спин здесь вообще ни при чем - ибо мы забили на релятивизм. Окажется, что для большинства 
	атомов с высокой точностью это приближение будет справедливым. 

	А если посмотреть на релятивиствкий случай - всегда ли мы можем сказать6 что и спиновой оператор
	и орбитальный момент имеют конкретные значения? Очень не всегда. А "хорошим" квантовым числом
	является именно полный момент, а эти лишь приближенными.

	Релятивистские взаимодействия можно учесть как поправку, в частности спин-орбитальное, как 
	наиболее сильное. Для электронной оболочки можно учесть, сопоставив этому взаимодействию оператор:
	\be
		\overline{\hat{V}_{sl}} = \overline{\sum_i \alpha_i \hat{\vec l}_i \hat{\vec s}_i};
	\ee

	Это усреднение по всей электронной оболочке. Тогда с хорошей точностью:
	\be
		\overline{\hat{V}_{sl}} \approx A \hat{\vec L} \hat{\vec S}; 
	\ee

	Электростатические взаимодействия в основном определяют энергию
	оболочки с параметрами $L,S$. А спин-орбитальное взаимодействие
	$\hat{V}_{L,S}$ приводит к тому, что энергия становится "слабо" зависящей от взаимной ориентации
	$L,S$, таким образом $V_{L,S}$ - определяет \textbf{тонкую структуру} атомных
	уровней. Т.е. Уроавни частично расщапляются, образуя мультиплеты.

	Сколько при этом образуется состояний, отличающихся энергией?ё
	Т.е. сколько состояний можно скомпоновать  из состояний с заданными $L,S$,
	но отличающихся $J$. Очевидно, что при $L \ge S$ - таких состояний найдется
	$2S + 1$, а если $S \ge L$ - тогда $2L + 1$. Такой пожход хорошо
	иллюстрируется при помощи "векторной" модели атома - это иллюстративная
	классическая модель. $\vec J = \vec L + \vec S$, в котором при постоянном 
	$\vec J$ вектора $\vec L$ и $\vec S$ складываются и прецессируют.

	При этом ваыполняются коммутационные соотношения:
	\be
		[\hat{V}_{LS}; \vec{J}^2] = 0;
	\ee
	\be
		[\hat{V}_{LS}; J_i] = 0;
	\ee
	\be
		[\hat{V}_{LS}; \vec{S}^2] = 0;
	\ee
	\be
		[\hat{V}_{LS}; \vec{L}^2] = 0;
	\ee
	\be
		[\hat{V}_{LS}; S_i] \neq 0;
	\ee
	\be
		[\hat{V}_{LS}; L_i] \neq 0;
	\ee

	Оценки величины $A$ показывают, что $A \sim Z^2$ -
	таким образом по мере увеличения порядкового номера 
	атома релятивистские взаимодействия квадратично растут,
	а $LS$ схема становится всё менее правдоподобной.

	Т.е. для тяжёлых атомов, когда елятивистские взаимодействия
	сравнимы с электростатическими6 для отдельного электрона
	$l,~s$ - являются "плохими" квантовыми числами, а сравнительно 
	"Хорошим" в этом случае является квантовое число  
	$\vec j = \vec l + \vec s$.

	А полный момент $\vec J$ складывается из $\vec j$ 
	отдельных электронов. Такая схема называется $jJ$ связью.

	\section{Понятие об "обменной" энергии}

	Если мы говорим про малый релятивизм, то мы говорим о присутствии
	в операторе Гамильтониана только слагаемых с кинетической энергией
	и потенциальной. Операторов спина в этом уравнении нет.

	Отсутствие в гамильтониане спиновых операторов казалось бы
	литшь означает, что полная волновая функция системы электронов 
	может быть записана через произведение 2-х функций - 1-а зависит только
	от координат частиц, другая - лишь от спиновых переменных частиц.
	
	Ну и что?

	Волновая функция системы тождественных частиц на самом деле должна обладать
	определённой симметрией по отношению к перестановке 
	любой пары частиц. Т.е. операция  замены $\xi_i \leftrightarrow \xi_j$
	должны приводить к тому что волновая функция будет умножаться на $\pm 1$ 
	(бозоны и фермионы). Для фермионов это приводит к принципу Паукли.

	Сказанное выше есть следствие математического выражения приципа неразличимости 
	(тождественности) частиц.

	\begin{tcolorbox}
		\textbf{Задача:} подумать над тем как экспериментально можно
		проверить принцип тождественности.
	\end{tcolorbox}

	Заметим, что если частиц $\ge 2$ волновая функция частиц
	вообще говоря не есть произведение функций, зависящих лишь от
	координат и функции, зависящей лишь от спина.

	Принцип тождественности приводит к важным физическим следствиям.
	Например пусть у нас есть гамильтониан:
	\be
		\hat H = \underbrace{\frac{\hat{\vec p}_1^2}{2m} + \frac{\hat{\vec p}_2^2}{2m} +
			U(\vec{r}_1) + U(\vec{r}_2)}_{\hat H (\vec{r}_1, \vec{r}_2)} + 
			\underbrace{U(\abs{\vec{r}_1 - \vec{r}_2})}_{\hat{H}_{int}};
	\ee

	Пусть $n$ - нумерует одночастичные состояния гамильтониана:
	\be
		\hat{H}_0' = \frac{\hat{\vec p}^2}{2m} + U(\vec r);
	\ee

	Составим волновые функции, соответствующие стационарному состоянию с 
	двухчастичным гамильтонианом, без взаимодействия $\hat{H}_0(\vec{r}_1,\vec{r}_2)$
	собственные функции такого гамильтониана должны быть произведением двух частей
	- пространственной и спиновой.

	Также такие функции должны быть:
	\begin{itemize}
		\item Антисимметричными
		\item Должно соответствовать ортонормированному базису.
	\end{itemize}

	Рассмотрим для начала координатную часть. Так как 
	$\hat{H}_0(\vec{r}_1,\vec{r}_2) = \hat{H}_0(\vec{r}_2,\vec{r}_1)$,
	то координатная часть может быть или симметричной:
	\be
		\Phi_s(\vec{r}_1,\vec{r}_2) = \frac{1}{\sqrt 2} \big(\psi_n(r_1)\psi_m(r_2) + \psi_n(r_2)\psi_m(r_1)\big);
	\ee

	и соответствовать энергии $E_n + E_m$, или антисимметричной:
	\be
		\Phi_a(\vec{r}_1,\vec{r}_2) = \frac{1}{\sqrt 2} \big(\psi_n(r_1)\psi_m(r_2) - \psi_n(r_2)\psi_m(r_1)\big);
	\ee  

	Соответственно и спиновая часть также может быть или симметричной, или антисимметричной.
	Для антисимметричной $S = 0$:
	\be
		\Phi_{a} = \frac{1}{\sqrt 2} \big(\psi_{1/2} (1) \psi_{-1/2} (2) - \psi_{1/2} (2) \psi_{-1/2} (1)\big);
	\ee

	Кроме того могут быть и функции, отвечающие другим значениям полного спина:
	\be
		\Phi^{+1}_s = \psi_{1/2}(1) \psi_{1/2}(2);
	\ee
	Что соответствует $S = 1,~S_z = 1$, или например:
	\be
		\Phi^{0}_s = \psi_{1/2}(1) \psi_{1/2}(2);
	\ee

	Соответсчтвует $S = 1,~S_z = 0$, а также:
	\be
		\Phi^{-1}_s = \psi_{1-/2}(1) \psi_{-1/2}(2);
	\ee

	При всем при этом:
	\be
		S_z \psi_{1/2,-1/2} = \pm \frac{1}{2} \psi_{1/2,-1/2};
	\ee

	Используя это мы можем составить полные антисимметричные функции.

	Построим синглетную функцию:
	\be
		\Phi_{sin} = \Phi_s(r_1,r_2) \Phi_a(1,2);
	\ee

	Для триплета:
	\be
		\Phi^{+1}_{tri} = \Phi_a(r_1,r_2) \Phi^{+1}_s(1,2);
	\ee
	\be
		\Phi^{0}_{tri} = \Phi_a(r_1,r_2) \Phi^{0}_s(1,2);
	\ee
	\be
		\Phi^{-1}_{tri} = \Phi_a(r_1,r_2) \Phi^{-1}_s(1,2);
	\ee

	Все триплетные функции соответствуют одной и той же энергии, если мы пренеьбрежем
	$H_{int}$.

	В состоянии синглета спины пары электронов противонаправленны. Теперь учтем $H_{int}$
	по теории возмущений. В первом приблдижении нам нужно 
	вычислить матричные элементы:
	\begin{multline}
		\bra{\Phi_{sin}} \hat{H}_{int} \ket{\Phi_{sin}} = \langle H_{int} \rangle (\uparrow \downarrow) = 
		\bra{(\psi_n(r_1) \psi_m(r_2))} U(\abs{r_1 - r_2}) \ket{(\psi_n(r_1) \psi_m(r_2))} + \\
		\bra{(\psi_n(r_1) \psi_m(r_2))} U(\abs{r_1 - r_2}) \ket{(\psi_n(r_2) \psi_m(r_1))};
	\end{multline}

	В таком случае полный матричный элмемент:
	\be
		\langle \hat{H}_{int} \rangle (\uparrow \downarrow) = E_0 + J_{ex} ';
	\ee

	А что получается с триплетом?
	\begin{multline}
		\bra{\Phi_{tri}} \hat{H}_{int} \ket{\Phi_{tri}} = \langle H_{int} \rangle (\uparrow \downarrow) = 
		\bra{(\psi_n(r_1) \psi_m(r_2))} U(\abs{r_1 - r_2}) \ket{(\psi_n(r_1) \psi_m(r_2))} - \\
		\bra{(\psi_n(r_1) \psi_m(r_2))} U(\abs{r_1 - r_2}) \ket{(\psi_n(r_2) \psi_m(r_1))};
	\end{multline}

	А энергия:
	\be
		\langle \hat{H}_{int} \rangle (\uparrow \downarrow) = E_0 - J_{ex} ';
	\ee

	Т.е. поправка тоже кулоновская. Т.е. Один электрон с координатой $r_1$ в состоянии n,
	а другой с $r_2$ - в состоянии m. При этом оба электрона в обьоих состояниях 
	одновременно. $J_{ex}$ - называется обменным интегралом.

	Еслти бы $J_{ex} >0$ Тогда $H_{int} (\uparrow \uparrow) < H_{int} (\uparrow \downarrow)$,
	тогда сонаправленная конфигурация была бы более энергетически выгодной. Это соответствовало
	ферромагнитному упорядочению спинов.

	В случае, если $J_{ex} >0$, тогда $H_{int} (\uparrow \uparrow) > H_{int} (\uparrow \downarrow)$
	соответствовало бы антиферромагнитному упорядочению.
	Но легко показать, что:
	\be
		J_{ex}' = \frac{1}{4\pi} \int \grad \phi \grad \phi ^{*} dr >0;
	\ee

	Где введено обозначение:
	\be
		\phi(\vec r) = \int \psi_n^{*} (\vec{r}_1) \psi_m(\vec{r}_1) \frac{e d\vec{r}_1}{\abs{\vec{r}_1 - \vec{r}}};
	\ee

	Эта модель качественно хорошо описывает ситуацию, когда оба электрона
	принадлежать  одному и тому же атому. очевидно, что рассматриваемая модельхороша в той мере, в какой
	хорошо работает теория возмущений, т.е. удачно выбрано невозмущенное
	состояние. 


	Модель, например хорошо описывает атом.
	\begin{tcolorbox}
		\textbf{Задача:} рассмотреть атом гелия.

		\textbf{Примечание:} если $\psi_n$, $\psi_m$,
		принадлежат атому, то $E_0 \sim J_{ex}'$.
	\end{tcolorbox}

	А теперь попробуем рассмотреть задачу, когда 2-а электрона принадлежат разным
	атомам. 

	Соответствующий гамильтонаиан будет иметь вид:
	\be
		\hat{H} = \frac{\hat{\vec p}_1^2}{2m} + \frac{\hat{\vec p}_2^2}{2m} + \
			U(\vec{r}_1 - \vec{R}_1) + U(\vec{r}_1 - \vec{R}_2) + 
			U(\vec{r}_2 - \vec{R}_1) + U(\vec{r}_2 - \vec{R}_2) + 
			\frac{e^2}{\abs{\vec{r}_1 - \vec{r}_2}};
	\ee

	Здесь $R_{1,2}$ - координаты ядер атомов.


\end{document}
