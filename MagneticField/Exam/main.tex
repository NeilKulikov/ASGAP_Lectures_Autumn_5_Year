%!TEX encoding = UTF-8 Unicode
\documentclass[a4paper, 14pt, russian]{article}
\usepackage[a4paper]{geometry}
\usepackage[T2A]{fontenc}
\usepackage[utf8]{inputenc}
\usepackage[russian]{babel}
\usepackage{physics}
\usepackage{tcolorbox}
\usepackage{hyperref}
\usepackage{fancyhdr}
\usepackage{indentfirst}
\usepackage{amssymb, amsmath, amsfonts}

\title{Физика магнитных явлений. Список билетов}
\author{Иосиф Давидович Токман}
\date{}

\newcommand{\be}{\begin{equation}}
\newcommand{\ee}{\end{equation}}
\newcommand{\bea}{\begin{equation}\begin{array}}
\newcommand{\eea}{\end{array}\end{equation}}

\newcommand{\pa}{\partial}
\newcommand{\rot}{\textbf{rot}~}
\renewcommand{\div}{\textbf{div}~}
\renewcommand{\grad}{\textbf{grad}~}
\newcommand{\ihline}{\noindent\rule{\textwidth}{1pt}}

\begin{document}
	\maketitle

	\noindent\rule{\textwidth}{1pt}

	\section{Билет}
		\subsection{Спин и уравнение Дирака}
		\subsection{Теплоёмкость газа магнонов}

	\ihline

	\section{Билет}
		\subsection{Доменная стенка}
		\subsection{Построить трёхэлектронную волновую функцию}
		для электронов, находящихся в p - состояних ($l = 1$),
		описывающую состояние с 
		$L = 1,~M_L = 2,~S=\frac{1}{2},~S_z = \frac{1}{2}$.

	\ihline

	\section{Билет}
		\subsection{Понятие об обменной энергии}
		\subsection{Гироэлектриеские среды}

	\ihline

	\section{Билет}
		\subsection{Динамика магнитной решётки ферромагнетика}
		в приближении обменного взаимодействия (квантовое рассмотрение).
		\subsection{Как преобразуется волновая функция}
		астицы с $S = 1$ при вращении вокруг оси $OZ$ системы координат?
		($\hat{T}_z = ?$).

	\ihline

	\section{Билет}
		\subsection{Модель Кюри-Вейса}
		- приближение молекулярного (среднего) поля.
		\subsection{Теорема Крамерса}
		
	\ihline

	\section{Билет}
		\subsection{Релятивистские взаимодействия}
		(следствие уравнения Дирака).
		\subsection{Малые ферромагнитные частицы}

	\ihline

	\section{Билет}
		\subsection{Магнитооптика в ферромагнитной среде}
		\subsection{Атом}

	\ihline

	\section{Билет}
		\subsection{Страйп-структура}
		многодоменного ферромагнетика
		\subsection{Преобразование спиноров при вращении}
		системы координат.

	\ihline
	
	\section{Билет}
		\subsection{Фактор Ланде}
		\subsection{Энергия магнитооптиеской анизотропии}

	\ihline

	\section{Билет}
		\subsection{Энергия магнитодипольного взаимодействия}
		\subsection{Спиновой обменный оператор Дирака.
		Взаимодействие Ва-Флека-Гейзенберга.}

	\ihline

	\section{Билет}
		\subsection{Динамика магнитной решётки ферромагнетика}
		в приближении обменного взаимодействия (классиеское рассмотрение).
		\subsection{Гиромагнитная среда}

	\ihline

	\section{Билет}
		\subsection{Атом}
		\subsection{Какова поляризация электромагнитной волны,}
		распространяющейся в гиромагнитной среде в 
		направлении внешнего магнитного поля?

	\ihline

	\section{Билет}
		\subsection{Спин}
		\subsection{Какова поляризация электромагнитной волны,}
		распространяющейся в гироэлектриеской среде в направлении
		её намагниенности?

	\ihline

	\section{Билет}
		\subsection{Страйп-структура}
		во внешнем магнитном поле.
		\subsection{Локализованные невзаимодействующие моменты.
		Парамагнетизм.}

	\ihline

	\section{Билет}
		\subsection{Понятие об обменной энергии}
		\subsection{Теорема Крамерса}

	\ihline

	\section{Билет}
		\subsection{Прохождение через и отражение от
		плоскопаралллельной пластины линейно поляризованного 
		излуения.}
		Пластина изготовлена из гироэлектриеской среды 
		с затуханием.
		\subsection{Обменное взаимодействие в модели:}
		два электрона на разных центрах. Ферромагнетизм
		и антиферромагнетизм.ё

	\ihline

	\section{Билет}
		\subsection{Правило Хундта}
		\subsection{Матрица плотности.}


	

\end{document}
