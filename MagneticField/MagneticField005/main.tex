%!TEX encoding = UTF-8 Unicode
\documentclass[a4paper, 14pt, russian]{article}
\usepackage[a4paper]{geometry}
\usepackage[T2A]{fontenc}
\usepackage[utf8]{inputenc}
\usepackage[russian]{babel}
\usepackage{physics}
\usepackage{tcolorbox}
\usepackage{hyperref}
\usepackage{fancyhdr}
\usepackage{indentfirst}
\usepackage{amssymb, amsmath, amsfonts}

\title{Физика магнитных явлений}
\author{Иосиф Давидович Токман}
\date{}

\newcommand{\be}{\begin{equation}}
\newcommand{\ee}{\end{equation}}
\newcommand{\bea}{\begin{equation}\begin{array}}
\newcommand{\eea}{\end{array}\end{equation}}

\newcommand{\pa}{\partial}
\newcommand{\rot}{\textbf{rot}~}
\renewcommand{\div}{\textbf{div}~}
\renewcommand{\grad}{\textbf{grad}~}

\setcounter{section}{7}

\begin{document}
	\maketitle

	\section{Обменная энергия}

	У нас была формула, которая описывает взаимодействие электронов в низком порядке
	по скорости. Т.е. вдияние орбитального движения одного электрона на орбитальное
	движение другого.

	Самая простая система - атом. Но ещё мы должны рассматривать обменное взаимодействие.
	При этом такое взаимодействие - должно учитывать тождественность частиц - симметрию
	гамильтониана при перестановке одинаковых частиц. На самом деле все это имеет корни 
	в квантовой электродинамике.

	Мы считали спин-спиновое взаимодействие. И оказалось, что например железо должно терять
	магнитные свойства при нагреве всего в пару градусов, если учитывать только спин - 
	спиновое взаимодействие.

	Был простейший пример - почти атом гелия. Для него запишем одноэлектронные волновые 
	функции:
	\be
		\psi_{n\uparrow}(\vec{r}_{1,2}) = \psi(\vec{r}_{1,2} - \vec{R}_n)
			\begin{bmatrix}1 \\ 2\end{bmatrix}{}_{1,2} = \psi_{n}(\vec{r}_{1,2}) 
			\begin{bmatrix}1 \\ 0\end{bmatrix}{}_{1,2};
	\ee

	И аналогично для другого центра $m$, и полностью аналогично, кроме спиновой части, 
	для обратного спина: $\psi_{m\uparrow}(\vec{r}_{1,2}), \psi_{n\downarrow}(\vec{r}_{1,2}),
	\psi_{m\downarrow}(\vec{r}_{1,2})$. Здесь и ниже индексы $m,~n$
	обозначают как центрированность волновых функций, так и какие-то квантовые числа.
	При этом координатные части таких волновых функци1 удовлетворяют уравнениям Шредингера:
	\be
		\big(\frac{\hat{\vec p}^2_{1,2}}{2m} + U(\vec{r}_{1,2} - \vec{R}_n)\big) 
			\psi_n(\vec{r}_1,2) = \hat{H}_n (\vec{r}_{1,2}) \psi_m = E_{0n} \psi_n;
	\ee

	И полностью аналогично для индекса $m$. При этом $\psi_n,~\psi_m$ - точные
	координатные волновые функции, соответствующие состояниям, при 
	$\abs{\vec{R}_n - \vec{R}_m} \rightarrow \infty$ - т.е. при отсутствии взаимодействия
	систем.

	Из этих функций составим двухлектронные функции, которые будем испол ьзовать в качестве функций
	нулевого приближения. В полной аналогии  с тем, что мы уже делали.

	Получаем одну функцию, являющуюся спиновым синглетом, и имеющую нулевой полный спин:
	\be
		\Phi_{sin} = \Phi_{s} (\vec{r}_1,\vec{r}_2) \Phi_a(1,2);
	\ee

	А также триплетные функции, имеющие разные спины $+1,0,-1$:
	\be
		\Phi_{tri}^{+1} = \Phi_{a} (\vec{r}_1,\vec{r}_2) \Phi_s^{+1}(1,2);
	\ee
	\be
		\Phi_{tri}^{0} = \Phi_{a} (\vec{r}_1,\vec{r}_2) \Phi_s^{0}(1,2);
	\ee
	\be
		\Phi_{tri}^{-1} = \Phi_{a} (\vec{r}_1,\vec{r}_2) \Phi_s^{-1}(1,2);
	\ee

	А как устроены симметричные и антисимметричные пространственные части?
	В данном случае так:
	\be
		\Phi_s(\vec{r}_1,\vec{r}_2) = \frac{1}{\sqrt{2 (1 + \abs{l}^2)}} 
			\big(\psi_n(r_1) \psi_m(r_2) + \psi_n(r_2) \psi_m(r_1)\big);
	\ee
	\be
		\Phi_a(\vec{r}_1,\vec{r}_2) = \frac{1}{\sqrt{2 (1 - \abs{l}^2)}} 
			\big(\psi_n(r_1) \psi_m(r_2) - \psi_n(r_2) \psi_m(r_1)\big);
	\ee

	А спиновые части в точности совпадают с тем, что мы писали
	для случая атома гелия.

	Дадим определению коэффициенту $l$ - это интеграл перекрытия волновых функций:
	\be
		l = \int \psi^{*}_n \psi_m d^3 \vec r;
	\ee

	Очевидно, что $m,~n$ - соответствуют разным центрам. Т.е. мы исключили из рассмотрения 
	состояния, когда электроны находятся на одном центре. Эти состояния лдавали бы 
	чрезмерно большую энергию возмущения $e^2 / \abs{\vec{r}_1 - \vec{r}_2}$.

	Используя такие приближенные функции, вычислим приближенные же хначения энергии,
	им соответствующие. Тогда получим:
	\be
		E_{s(a)} = \int \Phi_{s(a)}^{*}(\vec{r}_1,\vec{r}_2) \hat{H} \Phi_{s(a)}
			(\vec{r}_1,\vec{r}_2) d^3 \vec r;
	\ee

	Тогда получим для антисимметричной части:
	\be
		E_a = E_{0n} + E_{0m} + \frac{K - A}{1 - \abs{l}^2};
	\ee

	А для симметричного получим:
	\be
		E_s = E_{0n} + E_{0m} + \frac{K + A}{1 + \abs{l}^2};
	\ee

	Тогда у нас получается "довесок" к энергиям одноэлектронных состояний.
	При этом введены следующие обозначения:
	\begin{multline}
		K = \int \abs{\psi_n(\vec r)}^2 U(\vec r - \vec{R}_m) d^3 \vec r +
		\int \abs{\psi_m(\vec r)}^2 U(\vec r - \vec{R}_n) d^3 \vec r + \\
		\int \abs{\psi_n(\vec{r}_1)}^2 \abs{\psi_m(\vec{r}_2)}^2 
		\frac{e^2 d^3 \vec{r}_1 d^3 \vec{r}_2}{\abs{\vec{r}_1 - \vec{r}_2}};
	\end{multline}

	Т.е. $K$ - описывает энергию взаимодействия электронов
	с противоположными центрами и между собой.

	А другая компонента:
	\begin{multline}
		A = l^{*} \int \psi_n^{*} (\vec r) \psi_m (\vec r) U(\vec r - \vec{R}_n) d^3 \vec r + 
			l \int \psi_m^{*} (\vec r) \psi_n (\vec r) U(\vec r - \vec{R}_m) d^3 \vec r +\\
			\int \psi_m^{*}(\vec{r}_1) \psi_n^{*}(\vec{r}_2) \psi_m(\vec{r}_2) \psi_n(\vec{r}_1)
			\frac{e^2 d^3 \vec{r}_1 d^3 \vec{r}_2}{\abs{\vec{r}_1 - \vec{r}_2}}; 
	\end{multline}

	Используя это вычислим энергии состояний с сонаправленным расположением спина и с противонаправленным
	спином.

	\be
		E_s - E_a = 2\frac{A - K \abs{l}^2}{1 - \abs{l}^4} = 2J_{ex};
	\ee

	Важно, что величина энергии обменного взаимодействия может быть как положительной, так
	и отрицательной. В принципе это может послужить основой определения того,
	- будет ли основное состояние ферромагнитным или антиферромагнитным.

	В отсутствии перекрытия отсутствует и обмен:
	\be
		E_s \rightarrow_{l\rightarrow 0,~A\rightarrow 0} E_a;
	\ee

	Замечание: в случае 2-х центров $E_0 \nsim J_{ex}$.

	Учет спинов электронов приводит к тому, что энергия системы электронов 
	даже в нерелятивистском приближении (когда сам гамильтониан не зависит
	от спиновых переменных оказывается зависящей от спина). А именно - 
	благодаря принципу Паули координатная часть ВФ (ее симметрия)
	оказывается зависящей от спина (неявно).
	Но вид координатной части волновой функции как раз и определяет величину 
	Кулоновской энергии системы электронов.

	Раньше мы выписывали полный спин системы, когда говорили о термах
	атомов. Это было сделано именно для учёта такой неявной зависимости
	от спина. Компануя пространственную часть со спиновой, в случае только 
	кулоновского взаимодействия мы должны получить нечто, зависящее от спина.

	А в случае 3-х электронных функций мы будем получать почти то же самое, 
	только работать будем с детерминантами 3 на 3.

	\section{Магнитный момент электрона}

	Вернемся к гамильтониану уравнения Паули:
	\be
		\hat H = \frac{1}{2m} (\hat{\vec p} - \frac{e}{c} \vec A)^2 + e\phi - 
			\frac{e\hbar}{2mc} \hat{\vec \sigma} \vec{\mathcal H};
	\ee

	Можно раскрыть это как:
	\be
		\hat H = \frac{\hat{\vec p}^2}{2m}  - \frac{e}{2mc} (\hat{\vec p} \vec A +
			\vec A \hat{\vec p}) + \frac{e^2}{2mc} \vec{A}^2  + e\phi - 
			\frac{e\hbar}{2mc} \hat{\vec \sigma} \vec{\mathcal H};
	\ee

	Если мы выберем калибровку:
	\be
		\vec A = \frac{1}{2} [\vec{\mathcal H} \times \vec r];
	\ee

	То сможем ещё упростить уравнение Паули:
	\be
		\hat H = \hat{H}_0 - \frac{e}{2mc} \vec{\mathcal H} [\vec r \times \vec p]
			-\frac{e\hbar}{mc} \hat{\vec s} \vec{\mathcal H} = \hat{H}_0
			 -  \frac{e}{2mc} \vec{\mathcal H} \hat{\vec l} - \frac{e\hbar}{mc} 
			 \hat{\vec s} \vec{\mathcal H}
	 \ee
	 
	 В таком случае можем написать в ещё более красивой форме, приведя к одному виду:
	 \be
		 \hat H = \hat{H}_0 - \frac{\abs{e} \hbar}{2mc} (\hat{\vec l} +2 \hat{\vec s})
		 \vec{\mathcal H};
	 \ee

	 Отсюда видно, что орбитальному моменту электрона соответствует магнитный момент:
	 \be
	 	\hat{\vec{\mu}}_l = - \frac{\abs{e}\hbar}{2mc}\hat{\vec l};
	 \ee

	\begin{tcolorbox}
		\textbf{Задача:} Задача после \S 67  ЛЛ3. С 3 $p$ электронами.
		Найти полную волновую функцию.
	\end{tcolorbox}


	Спиновому моменту также соответствует какой-то магнитный момент:
	\be
		\hat{\vec{\mu}}_s = - \frac{\abs{e}\hbar}{mc} \hat{\vec s};
	\ee

	Таким образом полный магнитный момент электрона $\hat{\vec J}$ 
	соответствует магнитный момент:
	\be
		\hat{\vec \mu}_j = \frac{\abs{e} \hbar}{2mc} (\hat{\vec l} +2 \hat{\vec s});
	\ee

	Можно это немного упростить введя орбозначение магнетона Бора:
	\be
		\mu_B = \frac{\abs{e}\hbar}{2mc};
	\ee

	Если рассматривать электронную оболочку атома, то аналогично имеем:
	\be
		\hat H = \hat{H}_0 + \mu_B(\hat{\vec L}+2\hat{\vec S}) \vec{\mathcal H};
	\ee

	Поэтому магнитный момент электронной оболочки, связанный с орбитальным
	моментом:
	\be
		\hat{\vec \mu}_L = -\mu_B \hat{\vec L};
	\ee

	В таком случае вводится параметр:
	\be
		\mu_L = \mu_B \sqrt{L(L+1)};
	\ee

	Аналогично и со спином:
	\be
		\hat{\vec \mu}_S = - \mu_N 2 \hat{\vec S};
	\ee

	Что порождает параметр:
	\be
		\mu_S = 2 \mu_B \sqrt{S(S+1)};
	\ee

	Из этого следует, что:
	\be
		g_L \equiv 1 = \frac{\abs{\vec{M}_L}}{\hbar \abs{\vec L}} \frac{2mc}{\abs{e}};
	\ee

	Спиновое же движение характеризуется:
	\be
		g_S \equiv 2 = \frac{\abs{\vec{M}_S}}{\hbar \abs{\vec S}} \frac{2mc}{\abs{e}};
	\ee


	Очевидно, что и для полного магнитного момента,
	и для полного момента можно ввести соответствующее магнито-
	механическое соотношение. Будем считать, что справедлива
	$L,~S$ связь, т.е. $L,~S$ (по модулю) являются хорошими 
	интегралами движения. Это значит6 что в стационарном состоянии
	они хорошо определены. А в свою очередь $J,~J_z$ - 
	определены точно.

	Кроме того будем считать, что ось $z$ - по какой то причине выделена.
	Тогда можно написать для такого стационарного состояния:
	\be
		\langle \hat{\vec L} \hat{\vec J} \rangle = \frac{1}{2} \langle \hat{\vec J}^2 + 
			\hat{\vec L}^2 - \hat{\vec S}^2\rangle = \frac{1}{2} \big( J (J +1 ) +L (L + 1) 
			- S(S + 1)\big);
	\ee

	И полностью аналогично для спина:
	\be
		\langle \hat{\vec S} \hat{\vec J} \rangle = \frac{1}{2} \langle \hat{\vec J}^2 + 
			\hat{\vec S}^2 - \hat{\vec L}^2\rangle = \frac{1}{2} \big( J (J +1 ) +S (S + 1) 
			- L(L + 1)\big);
	\ee

	Все эти средние  - для удобного нам стационарного состояния. Тогда в рамках
	векторной модели мы получаем, что:
	\be
		\langle \cos{(\hat{\vec L};\hat{\vec J})} \rangle = \frac{J(J+1) + L(L+1)-S(S+1)}
			{2\sqrt{L(L+1)J(J+1)}};
	\ee
	\be
		\langle \cos{(\hat{\vec S};\hat{\vec J})} \rangle = \frac{J(J+1) - L(L+1)+S(S+1)}
			{2\sqrt{S(S+1)J(J+1)}};
	\ee

	Тогда мы можем представить, что вектор $\vec J$ - сладывается
	из векторов $\vec S,~\vec L$, которые "крутятся вокруг" $\vec J$.
	Сам же вектор $\vec J$ - прецессирует вокруг оси $z$ с фиксированным 
	$\vec{J}_z$.

	Из этой векторной модели можем найти проекцию на ось $\vec J$:
	\be
		\langle \vec{M}_L \rangle_{\vec J} = - \mu_B \sqrt{L(L+1)}
			\langle \cos(\hat{\vec L};\hat{\vec J})\rangle = 
			- \mu_B \frac{J(J+1) + L(L+1) - S(S+1)}{2\sqrt{J(J+1)}};
	\ee

	Полностью аналогично и для спиновой части:
	\be
		\langle \vec{M}_S \rangle_{\vec J} = - 2\mu_B \sqrt{s(s+1)}
			\langle \cos(\hat{\vec S};\hat{\vec J})\rangle = 
			- \mu_B \frac{J(J+1) - L(L+1) + S(S+1)}{\sqrt{J(J+1)}};
	\ee

	Таким образом получаем, что каждая из компонент магнитного момента 
	в проекции на $\vec J$ дает вклад в полный момент:
	\be
		M_J \equiv \langle \vec M \rangle_{\vec J} = \langle \hat{\vec M}_L\rangle_J
			+ \langle \hat{\vec M}_S\rangle_J;
	\ee

	В таком случае мы получаем по итогу:
	\be
		N_J = - \mu_B \underbrace{\left(1 + \frac{J(J+1) + S(S+1) - L(L+1)}{2J(J+1)}\right)
			\sqrt{J(J+1)}}_{g_J};
	\ee

	А само $g_J$ - называется фактором Ланде. Это полный аналог магнитомеханических соотношений.
	То, что $g_S \neq g_L$ - называется гиромагнитной аномалией спина. Из-за этой аномалии
	вектора $\vec{M}_J,~\vec J$ - не коллинеарны, в отличие от $\vec{M}_L,~\vec{L}$ и 
	$\vec{M}_S,~\vec S$.

	Замечание: по поводу спинового магнетизма ядер, можно заметить следующее: в формуле для 
	магнетона Бора можно подставить можно подставить массу протона и получим ядерный магнетон:
	\be
		\mu_n = \frac{\abs{e}\hbar}{2m_p c} \approx \frac{1}{1836} \mu_B;
	\ee

	В этом причина малости ядерного магнитного момента, в  сравнении с магнетизмом электронной 
	оболочки.

	Мы узнали, что отдельные атомы не обладают нескомпенсированным магнитным моментом.
	Стало понятно, что произойдет, если эти атомы выстроены в цепочку. Теперь можно перейти к коллективным явлениям.

	\section{Спиновой обменный оператор Дирака. Взаимодействие Ван-Флека-Гейзенберга.}
\end{document}
