%!TEX encoding = UTF-8 Unicode
\documentclass[a4paper, 14pt, russian]{article}
\usepackage[a4paper]{geometry}
\usepackage[T2A]{fontenc}
\usepackage[utf8]{inputenc}
\usepackage[russian]{babel}
\usepackage{physics}
\usepackage{tcolorbox}
\usepackage{hyperref}
\usepackage{fancyhdr}
\usepackage{indentfirst}
\usepackage{amssymb, amsmath, amsfonts}

\title{Физика магнитных явлений}
\author{Иосиф Давидович Токман}
\date{}

\newcommand{\be}{\begin{equation}}
\newcommand{\ee}{\end{equation}}
\newcommand{\bea}{\begin{equation}\begin{array}}
\newcommand{\eea}{\end{array}\end{equation}}

\newcommand{\pa}{\partial}
\newcommand{\rot}{\textbf{rot}~}
\renewcommand{\div}{\textbf{div}~}
\renewcommand{\grad}{\textbf{grad}~}

\setcounter{section}{14}

\begin{document}
	\maketitle

	Мы эксплуатируем гамильтониан Гейзенберга. Его форма - требует строгого
	доказщательства, однако интуитивно мы его понимаем. Все зависимости
	от координат зашиты в константе.

	В первом приближении мы посчитали спины просто векторами. Вообще
	мы можем вводить классическое описание далеко не всегда. Получились 
	у нас уравнения движения в формализме скобок Пуассона, вместо
	коммутаторов.

	Мы смогли получить некоторые особенности поведения таких систем из
	связанных спиновых частиц. Например смогли получить характер 
	намагниченности и спиновые волны. На самом деле 1D цепочка 
	спинов неустойчива.

	А что нам даст квантовое рассмотрение?

	При квантовом рассмотрении в гамильтониане Гейзенберга
	$\hat{\vec S}_i$ - оператор спина, действующий в пространстве спиноров,
	действующий на $i$-ом узле. Эти функции - функции столбцы, имеющие
	$2 S_0 + 1$ компонент. $S_0$ - максимальное значение проекции спина на ось 
	квантования. Матричные элементы операторов $\hat{\vec S}_i$ - 
	аналогичны рассотренным в самом начале.

	Вернемся к обычной квантовой механике - к матричной её части. Вместо 
	функций там столбцы, а вместо операторов - матрицы. А что есть базис?
	И нам нужно ещё знать, как оператор действует на этот базис. В нашем
	случае: $\ket{Y_{i,S_0,M_S}}$.Здесь $i$ - номер частицы, $S_0$ - максимальное
	возможное значение проекции спина, $M_S$ - точное значение проекции спина 
	в данном случае. В данном случае:
	\be
		\label{eq120}
		\hat{\vec S}_i^2 \ket{Y_{i,S_0,M}} = S_0 (S_0 + 1) \ket{Y_{i,S_0,M}};
	\ee
	\be
		\hat{S}_{iz} \ket{Y_{i,S_0,M}} = M_S \ket{Y_{i,S_0,M}};
	\ee

	И аналогично обычной квантовой мезханике можно ввести:
	\be
		\hat{S}_{i-} \ket{Y_{i,S_0,M}} = \sqrt{(S_0 + M)(S_0 - M + 1)} 
			\ket{Y_{i,S_0,M - 1}};
	\ee
	\be
		\hat{S}_{i+} \ket{Y_{i,S_0,M}} = \sqrt{(S_0 - M)(S_0 + M + 1)} 
			\ket{Y_{i,S_0,M + 1}};
	\ee

	Вместо $\ket{Y_{i,S_0,M_S}}$ введём $\ket{n_i}$ - это спиновая функция,
	описывающая состояние $i$ - ого спина с определённой проекцией на ось $OZ$.
	Тогда $M_{is} = S_0 - n_i$. Оно принимает значение: $n_i = 0,1,2\hdots$, 
	такие что $M_s = S_0 \hdots -S_0$. Удобно вввести оператор:
	\be
		\hat{a}^\dagger_i \ket{n_i} = \sqrt{n_i + 1} \ket{n_i + 1};
	\ee
	\be
		\hat{a}_i \ket{n_i} = \sqrt{n_i} \ket{n_i - 1};
	\ee

	Для этих операторов справедливы следующие коммутационные соотношения:
	\be
		[\hat{a}_i;\hat{a}_i^\dagger] = 1;
	\ee

	Это привычные нам орператоры рождения и уничтожения. Они часто применяются
	в многочастичных задачах или, например, в параболической яме. Используя
	их можно ввести оператор отклонения спина:
	\be
		\hat{n}_i = s_0 - \hat{M}_{is} = \hat{a}^\dagger_i \hat{a}_i;
	\ee

	В таком случае получим:
	\begin{multline}
		\label{eq121}
		\hat{S}_{i-} \ket{Y_{i,S_0,M_S}} = \hat{S}_{i-} \ket{n_i} = 
			\sqrt{(S_0 + M_{is})(S_0 - M_{is} + 1)} \ket{n_i + 1} = \\
			\sqrt{(2S_0 - n_i)(n_i + 1)} \ket{n_i + 1} = 
			\sqrt{2S_0} \hat{a}_i^\dagger \sqrt{1 - \frac{\hat{n}_i}{2S_0}} \ket{n_i};
	\end{multline}
	\begin{multline}
		\hat{S}_{i+} \ket{Y_{i,S_0,M_S}} = \hat{S}_{i+} \ket{n_i} = 
			\sqrt{(S_0 + M_{is} + 1)(S_0 - M_{is})} \ket{n_i - 1} = \\
			\sqrt{2S_0}\sqrt{1 - \frac{n_i -1}{2S_0}} \sqrt{n_i} \ket{n_i - 1} = 
			\sqrt{2S_0} \sqrt{1 - \frac{\hat{n}_i}{2S_0}} \hat{a}_i \ket{n_i};
	\end{multline}
	\be
		\hat{S}_{iz} = S_0 - \hat{a}_i^\dagger \hat{a}_i;
	\ee


	И мы хотели бы эту гадость линеаризовать. Если мы заинтересуемся состояниями,
	в которых $M_s$  мало отличается от $S_0$, тогда $n_i \ll 2S_0$, тогда
	мы имеем:
	\be
		\label{eq122}
		\hat{S}_{i+} \approx \sqrt{2S_0} \hat{a}_i;
	\ee
	\be
		\hat{S}_{i-} \approx \sqrt{2S_0} \hat{a}_i^\dagger;
	\ee

	Тогда гамильтониан Гейзенберга может быть записан в виде:
	\begin{multline}
		\label{eq123}
		\hat{H}_{ex} = - \frac{1}{2} J \sum_i (S_0 \hat{a}_i \hat{a}^\dagger_{i+1}
			+ S_0 \hat{a}_i^\dagger \hat{a}_{i+1} + 
			S_0 \hat{a}_i \hat{a}_{i-1}^\dagger + S_0 \hat{a}_i^\dagger \hat{a}_{i-1})\\ -
			\frac{1}{2} J \sum_i (S_0 - \hat{a}_i^\dagger \hat{a}_i)
			(S_0 - \hat{a}_{i+1}^\dagger \hat{a}_{i+1}) - 
			\frac{1}{2} J \sum_i (S_0 - \hat{a}_i^\dagger \hat{a}_i)
			(S_0 - \hat{a}_{i-1}^\dagger \hat{a}_{i-1});
	\end{multline}

	Причем мы рассматриваем только ближайших соседей. Перейдём в 
	Фурье-представление ($N$ - число спинов в цепочке,
	$R_i$ - дискретная координата спина):
	\be
		\hat{a}_i = \frac{1}{\sqrt N} \sum_{k} \hat{a}_{k} e^{ikR_i};
	\ee
	\be
		\hat{a}_i^\dagger = \frac{1}{\sqrt N} \sum_{k} \hat{a}_{k}^\dagger e^{-ikR_i};
	\ee

	Тогда вместо \ref{eq123} мы имеем:
	\begin{multline}
		\label{eq124}
		\hat{H}_{ex} = - \frac{1}{2} J \sum_k \big((\hat{a}^\dagger_k \hat{a}_k)
		(e^{ika} + e^{-ika}) + (\hat{a}_k \hat{a}_k^\dagger)(e^{ika} + e^{-ika})\big)\\
		+ 2S_0 J \sum_k \big((\hat{a}_k^\dagger \hat{a}_k) - S_0^2 J N \big)
	\end{multline}

	Чуть-чуть приведя:
	\be
		\label{eq125}
		\hat{H}_{ex} = \sum_k  J(2S_0)(1 - \cos ka)(\hat{a}_k^\dagger \hat{a}_k) - 
			- S_0^2 J N;
	\ee

	И окончательно можно видеть:
	\be
		\label{eq126}
		\hat{H}_{ex} = \sum_k \hbar \omega_k (\hat{a}_k^\dagger \hat{a}_k)- S_0^2 J N = 
			\sum_k \hbar \omega_k \hat{n}_k - S_0^2 J N;
	\ee

	А дисперсионка тут:
	\be
		\label{eq127}
		\omega_k = \frac{2S_0 J}{\hbar}(1 - \cos ka);
	\ee

	Это очень похоже на фононы. Константный член в уравнении
	\ref{eq126} - соответствует энергию в основном состоянии,
	а член $\hbar \omega_k \hat{n}_k$ - описывает возбуждение волны
	с частотой $\omega_k$, $\hat{n}_k$ - определяет энергию
	такой волны.

	Заметим, что \ref{eq127} в точности совпадает с дисперсионным
	соотношением, полученным при классическом рассмотрении.
	В линейном разложении:
	\be
		\omega_k \approx \frac{S_0 J a^2}{\hbar} k^2;
	\ee 

	В таком случае операторы $\hat{a}_k,~\hat{a}_k^\dagger$ 
	можно воспринимать, как операторы рождения и уничтожения
	кванта магнитной волны. такие кванты называаются магнонами.

	Так как $\hat{a}_i,~\hat{a}_i^\dagger$ - бозевские операторы,
	то и в случае волновых аналогов они тоже бозевские.
	таким образом в состоянии термодинамического равновесия:
	\be
		\label{eq128}
		n_k = (e^{\frac{\hbar \omega_k}{T}} - 1)^{-1};
	\ee

	А что с химпотенциалом? Это не обычные частицы. Их число
	не фиксированно, тогда их химпотенциал рапвен нулю.

	Используя \ref{eq128} вычислим термодинамически равновесную
	намагниченность цепочки спинов.
	\be
		\label{eq129}
		NS_z = NS_0 - \langle \sum_k \hat{a}_k^\dagger \hat{a}_k \rangle
			= NS_0 - \int_0^\infty \frac{dk}{e^{\frac{\hbar \omega_k}{T}} - 1} \frac{Na}{2\pi};
	\ee

	При малых $T$  число магнонов с большими $k$ экспоненциально
	мало. Поэтому верхний предел заменён на $\infty$.  Так как
	в случае цепочки $\omega_k \sim k^2$, то интеграл \ref{eq129}
	расходится в нижнем пределе. Это значит,
	что  при сколь угодно малой температуре не может реализоваться
	ферромагнитное упорядочение в магнитной цепочке.

	Проведённые нами рассмотрения для цепочки можно провести и для
	2D, и для 3D объектов спина. Закон дисперсии получился прежним:
	$\omega_k \sim k^2,~ ka \ll 1$. Тогда для 2D этот 
	интеграл также разошёлся бы.

	\be
		\langle \hat{a}_{\vec k}^\dagger \hat{a}_{\vec k} \rangle_{\text{2D}} \sim
			\int_0^\infty \frac{2 \pi k dk}{e^{\frac{\hbar\omega_k}{T}} - 1}
			\frac{Na^2}{(2\pi)^2}; 
	\ee

	Таким образом на плоскости не возможен ферромагнетизм.
	Естественно перейти к тому, что встречается в природе само по себе.
	А именно к 3D случаю. Там такой интеграл не расходится:
	Рассмотрим для начала гамильтониан (здесь суммирование
	по импульсу):
	\be
		\label{eq130}
		\hat{H}_{ex} = \frac{1}{2} z J_0 a^2 k^2 \hat{a}_k^\dagger \hat{a}_k
			- \frac{1}{2} Nz J^2 S
	\ee
	\be
		\omega_k \approx \frac{1}{2\hbar} z J S_0 a^2 k^2;
	\ee
	
	В таком случае среднее значение:
	\be
		\label{eq131}
		NS_z = NS_0 - \int_0^\infty \frac{4\pi k^2 dk}
			{e^{\frac{\hbar \omega_k}{T}} - 1} \frac{Na^3}{2\pi^3}
			= NS_0 \left(1 - \frac{T^{3/2}}{2^{-1/2} \pi^2 z^{3/2} S_0^{5/2} J^{3/2}}
			\int_0^\infty \frac{x^2 dx}{e^{x^2} - 1} \right);
	\ee

	Тогда с увеличением температуры у нас уменьшается упорядочение.
	\ref{eq131} описывает изменение намагниченности трёхмерного
	ферромагнетика при низких температурах, обусловленные 
	возбуждением спиновых волн (газа магнонов). Магноны ведут 
	себя как идеальный газ - это следствие нашей линеаризации.

	Теплоемкость газа магнонов. В рассматриваемом приближении
	газ магноно видеален. В состоянии термодинамического
	равновесия его энергия в соответствии с \ref{eq126}:
	\be
		\label{eq133}
		E_m = \sum_k \hbar \omega_k n_k \approx
			\int_0^\infty \frac{\hbar\omega_k}{e^\frac{\hbar\omega_k}{T} - 1} 
			\frac{4\pi k^2 N a^3 dk}{(2\pi)^2} = 
			\frac{NT^{5/2}}{2^{-1/2} \pi^2 (zJS_0)^{3/2}}
			\int_0^\infty \frac{x^4 dx}{e^{x^2} - 1};
	\ee

	Из \ref{eq133}  мы вычислим магнонную теплоёмкость:
	\be
		C_m = \frac{\pa E_m}{\pa T} \sim T^{3/2};
	\ee

	Таким образом теплоемкость металла, кроме теплоёмкости
	электронной и фононной части, содержит ещё и магнонную.

	Мы говорим в основном про ферромагнетизм, но есть ещё 
	и антиферромагнетизм. Там закон дисперсии для магнона 
	другой. В остальном все похоже. 

	\section{Релятивистские взаимодействия в 
	ферромагнитном кристалле.}

	Обменное взаимодействие, описываемое гамильтонианом Гейзенберга
	является электростатическим по своей природе. Описание
	ферромагнетика приводит к тому, что за основное состояние
	мы принимаем такое, в котором все спины сонаправленны 
	и направленны вдоль любой оси. Т.е. оси, направленной 
	произвольно, относительно кристаллографических осей кристалла.
	Однако обменное взаимодействие не единственное взаимодействие
	в ферромагнетике. Учтём релятивистские, спин-спиновые
	и спин-орбитальные взаимодействия.

	Спин-спиновое (дипольное) взаимодействие - энергия
	взаимодействия пары спинов уже была записана. В 
	классической электродинамике, такое взаимодействие называется
	диполь-дипольным. Поэтому о соответствующей энергии
	говорят, как об энергии дипольного взаимодействия.

	Важно подчеркнуть, что такое взаимодействие ведёт себя по закону
	$\sim 1 / r^3$.

	Рассмотрим это взаимодействие. Энергия каждой такой пары
	зависит от ориентации относительно прямой их соединяющей. 

	Такие прямы в кристалле обусловлены кристаллической
	решёткой. поэжтому энергия дипольного взаимодействия 
	зависит от ориентации намагниченности материала, относительно
	кристаллографичиских направлений. На лицо анизотропия.
	Можно показать, что в приближении сплошной среды
	выражение для энергии диполь-дипольного взаимодействия
	имеет вид:
	\be
		\label{eq135}
		E_m = -\frac{1}{2} \int_V d^3 \vec r \big(\beta_{\alpha\gamma} M_\alpha(\vec r)
			M_\gamma(\vec r) + \frac{4\pi}{3} \vec{M}^2 (\vec r)
			+ \vec{M}(\vec r) \vec{\mathcal H}_m(\vec r)\big);
	\ee

	Здесь $M$ - плотность магнитного момента, $\alpha,~\gamma$ - координаты, 
	$\beta_{\alpha \gamma}$ - тензор, зависящий от структуры кристалла.

	Первый член \ref{eq135} описывает эффекты анизотропии. Его Учёт 
	объясняет взаимодействие между близкими спинами, а уже остальные члены
	связаны с дальнодействующим характером диполь-дипольного взаимодействия.
	Поэтому они не связанны с симметрией кристалла.
	
	$\vec{\mathcal H}_m$ - напряжённость магнитного поля, соответствуюшая 
	заданному распределению.

	\be
		\label{eq136}
		\rot \vec{\mathcal H}_m = 0;
	\ee
	\be
		\div (\vec{\mathcal H}_m + 4\pi \vec M) = 0;
	\ee

	Спин-орбитальное взаимодействие. По поводу спин-орбитального
	взаимодействия можно заметить следующее. В изолированном атоме 
	энергия спин-орбитального взаимодействия  определяется взаимной 
	ориентацией спинового и орбитального моментов ($LS$ связь).

	И в атоме распределение электронной плотности в пространстве определяется 
	$\vec L$. В кристалле же симметрия в распределении электронной плотности
	определяется симметрией кристалла. Поэтому в кристалле спин орбитальное
	взаимодействие определяется ориентацией спинового момента относительно
	кристаллографических осей.

	Можно показать, что \textit{эффективный гамильттониан взаимодействия между спинами},
	обусловленный спин-орбитальным взаимодействием записывается в виде:
	\be
		\label{eq137}
		\hat{H}_{se} \sim \sum_{ij} \beta_{\alpha \gamma}(\vec{r}_{ij}) S_{i\alpha} S_{i\beta};
	\ee

	Снова этот тензор оказывается привящзанным к кристаллографическим направлениям.
	Гамильтониан \ref{eq137} получен исключением орбитальных переменных
	и потому зависит лишь от спинов. Информация об орбитальных переменных содержится в 
	тензоре и определяется симметрией кристалла.



\end{document}
