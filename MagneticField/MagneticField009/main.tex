%!TEX encoding = UTF-8 Unicode
\documentclass[a4paper, 14pt, russian]{article}
\usepackage[a4paper]{geometry}
\usepackage[T2A]{fontenc}
\usepackage[utf8]{inputenc}
\usepackage[russian]{babel}
\usepackage{physics}
\usepackage{tcolorbox}
\usepackage{hyperref}
\usepackage{fancyhdr}
\usepackage{indentfirst}
\usepackage{amssymb, amsmath, amsfonts}

\title{Физика магнитных явлений}
\author{Иосиф Давидович Токман}
\date{}

\newcommand{\be}{\begin{equation}}
\newcommand{\ee}{\end{equation}}
\newcommand{\bea}{\begin{equation}\begin{array}}
\newcommand{\eea}{\end{array}\end{equation}}

\newcommand{\pa}{\partial}
\newcommand{\rot}{\textbf{rot}~}
\renewcommand{\div}{\textbf{div}~}
\renewcommand{\grad}{\textbf{grad}~}

\setcounter{section}{14}

\begin{document}
	\maketitle

	Немного о терминологии. Спин спиновое взаимодействие в лоб - это то, что
	сы рассматривали, когда говорили о релятивистских эффектах. Это непосредственно
	спин-спин. Но спин орбитальное взаимодействие тоже присутствует и приводим гамильтониан
	к виду, где это похоже на спин-спиновое взаимодействие. Природа у него другая.

	Мы все это получали в предположении, что интегралы, где стоит бозевская функция 
	распределения, брались в пределе до бесконечности. Но строго говоря это так 
	лишь при низкой температуре. Что это за низкая температура?

	\begin{tcolorbox}
		\textbf{Задача:} Что такое низкие температуры? Чем она обусловлена?
		Какую точность обеспечивает это приближение? 
	\end{tcolorbox}

	Мы рассматриваем релятивистские взаимодействия в кристалле. Главный прорыв 
	до сих пор был в том, что мы орбъяснили спиновую упорядоченность 
	при помощи обменного взаимодействия.

	Есть однако более слабые взаимодействия. И, оказывается, они тоже важны.
	Мы писали, что есть некоторые члены, отвечающие кристаллической структуре, а есть
	члены, отвечающие чисто макроскопическим электродинамическим взаимодействиям.

	Мы получилти оператор, по форме совпадающим с первым членом. Этот оператор описывает эффекты
	анизотропии. Важно подчеркнуть, что этот тензор, как функция расстояния между спинами 
	быстро спадает с увеличением расстояния. Т.е. это взаимодействие 
	\textit{не дальнодействующее}.

	При переходе к приближению сплошной среды можно записать:
	\be
		\label{eq138}
		E_\text{se} = \frac{1}{2} \int \beta_{\alpha \gamma} M_\alpha(\vec r)
			M_\gamma(\vec r) d^3 \vec r;
	\ee

	\section{Магнито кристаллографическая анизотропия}

	Рассмотрим предыдущие выражения. Видно, что первые члены везде имеют одну
	и ту же структуру. Объединим в один член, который будет содержать
	информацию о симметрии кристалл и направлении вектора намагниченности.
	\be
		\label{eq139}
		E_\text{A} = \int_V \frac{1}{2} \beta_{\alpha \gamma} 
			M_\alpha (\vec r) M_\gamma (\vec r) d^3 \vec r;
	\ee

	Здесь есть информация о кристаллографической структуре. 

	При таком определении энергии кристаллографической магнитной анизотропии 
	под энергией магнито-дипольного взаимодействия следует понимать
	величину:
	\be
		\label{eq140}
		E_\text{MD} = - \frac{1}{2} \int \vec{M}(\vec r) 
			\vec{\mathcal H}_{M} (\vec r) d^3 \vec r;
	\ee

	Мы здесь видим члены не зависящие непосредственно от кристалла. Это классическое
	описание.


	\section{Энергия кристаллографической анизотропии. Энергия магнитно-дипольного взаимодействия.}

	Количественно магнитокристаллографическую анизотропию ферромагнитного кристалла принято 
	характеризовать соответствующими константами. Так из  \ref{eq139} видно, что
	плотность энергии магнитокристаллографической анизотропии в более общем виде 
	можно записать как:
	\be
		\label{eq141}
		\epsilon_\text{A} = \sum_{n_x,n_y,n_z} K_{n_x,n_y,n_z} 
			\alpha_x^{n_x} \alpha_y^{n_y} \alpha_z^{n_z};
	\ee

	Здесь $K$ - константы кристаллографической анизотропии, размерностью
	$\frac{\text{erg}}{\text{cm}^3}$. $\alpha$ - направляющие
	косинусов вектора намагниченности по направлению к кристаллографическим осям. 
	Причем комбинация $\alpha_{x},\alpha_{y}, \alpha_{z}$ - такие, что
	$\epsilon_\text{A}$ - инвариантно относительно элементов симметрии кристалла.
	$n_x,n_y,n_z$ - (степени) могут принимать значения $0,1,2,\hdots$, Причем
	$n_x + n_y + n_z$ -  чётное число, т.к. в этом случае $\epsilon_\text{A}$
	остаётся инвариантной относительно замены $t \rightarrow - t$, а 
	$\alpha_{x,y,z}$ -  меняют знак. 

	Пусть ось "лёгкого"  намагничивания совпадает с $z$. Тогда:
	\be
		\epsilon_\text{A} = K (\cos \theta)^2,~K < 0;
	\ee

	Очевидно, что тут есть 2-а минимума (и это почти всегда так).

	А если у константы другой знак $K > 0$ - то у нас получается
	миниамльное значение при $\pi/ 2$. Энергия магнито-дипольного
	взаимодействия характеризуется своей плотностью. Из \ref{eq140}
	мы можем получить:
	\be
		\label{eq142}
		\epsilon_\text{MD} = - \frac{1}{2} \vec M \vec{\mathcal H}_\text{M};
	\ee


	В случае однородно намагниченного ферромагнетика именно конкуренция этих двух
	плотностей энергии определяет направление намагниченности в состоянии 
	равновесия. 
	
	Рассмотрим пример: пластину ферромагнетика с легкой осью, перепендикулярной плоскости
	пластины. Пусть намагниченность имеет с осью $z$ (перпендикулярной плоскости) 
	угол $\theta$. Тогда полная плотность энергии:
	\be
		\epsilon = \epsilon_\text{A} + \epsilon_\text{MD} = 
			K (\cos \theta)^2 - \frac{1}{2} \vec{\mathcal H}_\text{M} \vec M;
	\ee

	Так как вне пластины $\vec B = \vec{\mathcal H}_\text{M} = 0$, 
	то можно видеть:
	\be
		\epsilon = K (\cos \theta)^2 + 2 \pi M^2 (\cos \theta)^2;
	\ee

	Поскольку у нас лёгкая ось. То $K < 0$, тогда если $\abs{K} > 2 \pi M^2$
	-  у нас намагниченность перпендикулярна, а если наоборот - то в плоскости
	пластины. Отсюда видно, что магнитостатическая энергия образца зависит от 
	направления намагниченности. Т.е. анизотропия в этом случае вызвана
	анизотропией формы тела. 

	Тензор размагничивающих коэффициентов. Пусть ферромагнетик таков, что энергия 
	магнитокристаллографической анизотропии много меньше магнито-дипольного взаимодействия.

	Тогда если изначально ферромагнетик намагничен однородно, то направление намагниченности
	определяется лишь магнито-дипольного взаимодействия. А последнее определяется
	формой ферромагнитного тела.

	Если форма тела - эллипсоид и тело однородно намагниченно $\vec M(\vec r) = \text{const}$,
	то магнитное поле \textit{внутри} магнетика - тоже однородное 
	$\vec{\mathcal H}_\text{M} = \text{const}$. Поэтому существует связь:
	\be
		\label{eq143}
		\mathcal{H}_{\text{M} i} = - 4 \pi N_{ij} M_j;
	\ee

	При этом $N$ - тензор размагничивания, определяемый геометрическими свойствами тела.
	В декартовой системе, где орты совпадают с главными осями эллипсоида - 
	тензор диагонализуется. 
	\be
		\label{eq144}
		N_{xx}^\text{main} \frac{1}{2} abc \int_0^\infty \frac{dS}{(a^2 + s)R_s};
	\ee


	И аналогично для других осей $yy,~zz$, а параметр:
	\be
		R_s = \sqrt{(a^2 + s)(b^2 + s)(c^2 + s)};
	\ee

	Тогда мы должны получить:
	\be
		\label{eq145}
		N_{xx}^\text{main} + N_{yy}^\text{main} + N_{zz}^\text{main} = 1;
	\ee

	В случае шара $N_{ii}^\text{main} = 1/3$ - все равны. Тогда в шарике:
	$\vec{\mathcal H}_\text{M} = - \frac{4\pi}{3} \vec M$. Если же 
	образец имеет форму всильно вытянутого вдоль оси $x$ циллиндра, то
	$N_{xx} = 0,~ N_{yy} = N_{zz} = \frac{1}{2}$ и энергия минимальна, если 
	намагниченность направленна вдоль оси $x$.

	Но однородная намагниченность - редкое явление, чаще весь предмет разбивается
	на кластеры (домены). 
	
	\section{Доменная структура ферромагнетика}

	Легко показать6 что энергия магнито-дипольного взаимодействия представима в виде:
	\be
		\label{eq146}
		E_\text{MD} = - \frac{1}{2} \int \vec{\mathcal H}_\text{M} \vec M dV = 
			\frac{1}{8\pi} \int \vec{\mathcal H}_\text{M}^2 dV;
	\ee

	Здесь интегрироввание происходит по всему пространству. Поэтому если 
	образец имеет макроскопические размеры, то при его намагничивании его энергия 
	магнитодипольного взаимодействия будет велика.

	Если же объект разобьется на домены с различным направлением намагниченности, 
	то энергия магнитостатического взаимодействия станет меньше.

	Пример: ферромагнитная пластина с лёгкой осью, перпендикулярной плоскости пластины.

	Но эта штука может разбиться на домены и линии магнитного поля "закольцуются" - 
	в результате поле вне пластинци уменьшится. Но почему все не превращается в сплошные
	мелкие диполи? Мешает обменное взаимодействие.

	Разбиение на домены приводит к снижению энергии. Область между 2-мя доменами
	называется доменной стенкой (или границей). В этой области происходит
	поворот вектора намагниченности. Тем самым в этой области энергия магнитной
	кристаллографической анизотропии выше, чем в самих доменах.
	Т.е. доменая стенка обладает энергией.

	Поэтому размеры доменной стенки и домена при которых полная энергия ферромагнетика
	минимальна имеют определённые значения. Рассмотрим отдельно доменную стенку.

	Рассмотрим ферромагнитный кристалл с анизотропией лёгкой оси $z$. Ось $x$
	перпендикулярна плоскости доменной стенки.

	Намагниченность в  соседних доменах направленна в противоположные стороны вдоль 
	лёгкой оси. Переход от одного домена к другому сопровождается
	поворотом вектора намагниченности в плоскости $yz$. Такая стенка
	называется блоховской. Но есть и другой тип стенки - Неймановская (магнитное поле
	"рыбкой" меняет направление).

	Начало координат находится в середине стенки. $M_0$ - намагниченность
	в домене вдали от стенки. Для простоты примем, что в доменной стенке 
	вектор намагниченности "поворачивается" в одной атомной плоскости на один и 
	тот же угол $\phi = \pi / N$, здесь $N$ - число атомных плоскостей
	в доменной стенке. Толщина такой стенки составляет в таком случае $\delta = N a$, где
	$a$  - межатомное расстояние.

	Энергия анизотропии, приходящаяся на один атом в междоменной стенке:
	\be
		\epsilon = K (\cos \theta)^2 a^3;
	\ee

	Тогда поверхностная плотность энергии анизотропии доменной стенки (энергия
	на единицу площади стенки):
	\be
		\label{eq143}
		\sigma_\text{A} = \frac{1}{a^2} \int_0^\pi K (\cos \theta)^2 a^3 
			\frac{d\theta}{\pi N^{-1}} - \frac{1}{a^2} \int_0^\pi
			K a^3 \frac{d\theta}{\pi N^{-1}} = \frac{1}{2}N a \abs{K};
	\ee

	Можно легко написать выражение для обменной энергии:
	\be
		\sigma_\text{ex} = \int_{-Na /2 }^{Na / 2} A (\pa_x \theta)^2 dx;
	\ee

	При этом  $\theta = \frac{\pi}{N} \frac{x}{a} + \frac{\pi }{ 2}$, в таком
	случае интегрируя получим:
	\be
		\label{eq148}
		\sigma_\text{ex} = \frac{A\pi^2}{Na};
	\ee

	В сумме видим:
	\be
		\label{eq149}
		\sigma_\text{wall} = \frac{1}{2} \abs{K} \delta + \frac{a\pi^2}{\delta};
	\ee

	Равновесная толщина стенки в таком случае:
	\be
		\label{eq150}
		\pa_\delta \sigma = 0 \rightarrow \delta = \pi \sqrt 2 \sqrt{\frac{A}{\abs{ K}}};
	\ee

	Тогда равновесное значение поверхностной плотности энергии доменной стенки.
	\be
		\label{eq151}
		\sigma_\text{wall} = \pi \sqrt 2 \sqrt{A \abs{K}};
	\ee

	Можно видеть, что стенки увеличивают энергию ферромагнетика. Заметим, что
	в блоховской стенке магнитостатическая энергия равна нулю.

	Энергия магнитодипольного взаимодействия многодоменного ферромагнетика.
	Рассмотрим анизотропию типа "лёгкая ось". При этом 
	$\abs{K} > 2 \pi M_0^2$. Пусть толщина пленки составляет
	$h$, а характерный размер домена - $d$. 
	
	Поулчается stripe - структура. На поерхности намагниченность рвётся. В каждом 
	домене намагриченности $M_0$. Напишем уравнения магнитостатики:
	\be
		\rot \vec{\mathcal H}_\text{M} = 0;~ 
		\div(\vec{\mathcal H}_\text{M} + 4 \pi \vec M) = 0;
	\ee

	Они аналогичны электростатическим уравнениям ($\div \vec M \rightarrow - \rho,~
	\vec{\mathcal H}_\text{M} \rightarrow \vec E,~M \rightarrow - \sigma$):
	\be
		\rot \vec E = 0;~
		\div \vec E = 4 \pi \rho;
	\ee

	Разложим в ряд Фурье:
	\be
		\label{eq152}
		\sigma(x) = \sum_0^\infty \frac{4M_0}{\pi (2 n + 1)} \sin \frac{(2n + 1)\pi x}{d};
	\ee

	А вне плоскости справедливо уравнение Лапласа:
	\be
		\nabla \phi = 0 \rightarrow \pa_{xx} \phi + \pa_{zz} \phi = 0;
	\ee

	Решение, записанное в виде ряда имеет вид:
	\be
		\label{eq153}
		\phi(x, z) = \sum_0^\infty b_n \sin \frac{(2n + 1)\pi x}{d} 
		\exp \pm \frac{(2n + 1) \pi z}{d};
	\ee

	Знаки в показателе экспоненты говорят о структуре поля сверху и снизу пластины.
	Всё поле оказывается экспоненциально спадающим.

	Граничные ус ловия (здесь другое $\sigma$):
	\be
		\label{eq154}
		E_z(z + 0) - E_z(z - 0) = 4 \pi \sigma;
	\ee

	Решая можно получить:
	\be
		\label{eq155}
		b_n = \frac{8 M d}{\pi (2 n + 1)^2};
	\ee

	Для простоты будем считтать, что толщина пластины $h \gg d$. Это означает,
	что поле, создаваемое "зарядами" на одной поверхности пластины слабо
	взаимодействуют с зарядами на противоположной.

	Вследствии этого энергию магнито-дипольного взаимодействия можно вычислить как:
	\be
		\label{eq156}
		E_\text{MD} = \frac{1}{8\pi} \int (\vec{\mathcal H}_\text{M})^2 dV = 
			\int \sigma(x) \phi(x, z = 0) dx dy;
	\ee


	Усредняя можем получить:
	\be
		\label{eq157}
		\overline{E}_\text{MD} = \frac{1}{2} \int_{-d}^{d} \sigma(x) \phi(x, z= 0) dx 
			= \frac{16 d M_0^2}{\pi^2} \sum_0^\infty (2n + 1)^{-3} 
			\approx 1.75 d M_0^2; 
	\ee

\end{document}
