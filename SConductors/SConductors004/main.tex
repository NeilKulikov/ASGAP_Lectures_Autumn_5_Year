%!TEX encoding = UTF-8 Unicode
\documentclass[a4paper, 14pt, russian]{article}
\usepackage[a4paper]{geometry}
\usepackage[T2A]{fontenc}
\usepackage[utf8]{inputenc}
\usepackage[russian]{babel}
\usepackage{physics}
\usepackage{hyperref}
\usepackage{fancyhdr}
\usepackage{indentfirst}
\usepackage{amssymb, amsmath}

\title{Теория Сверхпроводимости. Феноменологическая теория.}
\author{Курин}
\date{}

\newcommand{\be}{\begin{equation}}
\newcommand{\ee}{\end{equation}}
\newcommand{\bea}{\begin{eqnarray}}
\newcommand{\eea}{\end{eqnarray}}

\newcommand{\pa}{\partial}
\newcommand{\rot}{\textbf{rot}~}
\renewcommand{\div}{\textbf{div}~}
\renewcommand{\grad}{\textbf{grad}~}

\begin{document}
	\maketitle

	\section{Элементарная термодинамика сверхпроводников}

	Мы написали несколько потенциалов $E,~F,~W,~\Phi,~\Omega$,
	а также несколько аналогов с волной: $\tilde F$.

	Например для свободной энергии:
	\be
		\delta F = - S\delta T - p \delta V + \mu \delta N + \frac{1}{4\pi} \int \vec H  \delta \vec B dV;
	\ee
	
	Тогда её модификация:
	\be
		\tilde F = F_{0s} + \int_{V_s} \frac{H^2}{8\pi} dV - \int_\infty \frac{H^2}{8\pi} dV;
	\ee

	В случае маленького циллиндра в соленоиде мы можем пренебречь интегралом по всему объёему - 
	он просто изменит уровень отсчёта энергии. Тогда:
	\be
		\tilde{F}_s = F_{0s} (T,V_s,N_s) + V_s \frac{H^2}{8\pi};
	\ee

	Введём понятие равновесия фаз - нормальной и сверхпроводящей. Если у нас 
	есть обе фазы - тогда (если пренебречь слабыми парамагнитными свойствами):
	\be
		\tilde F = F_{0s} + F_n(T, V_N, N_N) + V_s \frac{H^2}{8\pi};
	\ee

	Тогда свободная энергия должна иметь минимум, его необходимым условием будет 
	зануление производных:
	\be
		\frac{\delta \tilde F}{\delta V_s} = \frac{\delta F_{0s}}{\delta V_s} -
			\frac{\delta F_N}{\delta V_n} + \frac{H^2}{8\pi} = -p_s + 
				\frac{H^2}{8\pi} + p_N;
	\ee

	А ещё длолжно быть такое же условие по числу частиц:
	\be
		\frac{\delta \tilde F}{\delta N_s} = \frac{\delta F_{0s}}{\delta N_s} -
			\frac{\delta F_{N}}{\delta N_N}; 
	\ee

	Что такое теплоемкость? По определению:
	\be
		C = \frac{\delta Q}{\delta T} = T \frac{\delta S}{\delta T};
	\ee

	Фишка в том, что сжимаемость металлов очень мала (как и у жидколстей, например).
	Тогда условие равновесия хим. потенциалов:
	\be
		\frac{\delta F_{0s}}{\delta N_S} = \frac{\delta F_{N}}{\delta N_N};
	\ee

	И если мы введём объёмную свободную энергию $F = V f$ (аналогично для S, N),
	и при этом $V$ - экстенсивная величина, а $f$ - интенсивная, тогда $f(T, n = N / V)$.
	Отсюда следует, что производные по числу частиц,  могут быть выражены через 
	производные по объёму ($\pa_N f = \pa_n f / V,~ \pa_v f = (-N/ V) \cdot \pa_n f$):
	\be
		N_s \pa_{N_s} f = - V \pa_N f;
	\ee

	Тогда применительно к нашей ситуации:
	\be
		V_s \frac{\delta f_{0s}}{\delta N_s} = V_n \frac{\delta f_{N}}{\delta N_N};
	\ee

	Если воспользуемся прошлым выражением для механического равновесия:
	\be
		\frac{V_s}{N_s} (p_s + f_{0s}) = \frac{V_n}{N_n} (p_N + f_N);
	\ee

	Здесь мы знаем:
	\be
		p_s = p_N + \frac{H^2}{8\pi};
	\ee

	А за счет чего у нас может происходить сжатие? За счет магнитного давления,
	а оно мало. Тогда:
	\be
		f_{0s} + \frac{H_c^2}{8\pi} = f_N;
	\ee

	Эта элементарная термодинамика была разработана очень быстро. Исходной 
	информацией у нас является кривая мейснера:

	%img001

	Продифференциировав условие равновесия хим. потенциалов мы получим:
	\be
		-s_S + \frac{1}{4\pi} H_c \pa_T H_c = - s_N;
	\ee
	
	Тогда из графика видно, что при фазовом переходе выделяется тепло.
	Что это значит? Здесь можНо увидеть фазовый переход 2-ого рода, и 
	увидеть аналогию - сверхпроводящий материал эквивалентен льду, а 
	нормальный - газу. 

	Например есть задача Капицы - как бы тро проплавит проволочка сгрузом 
	брусок льда?

	Как происходит переход в нашем случае?

	%img001

	Если поле чуть чуть превысит критическое - объём сверхпроводника 
	уменьшится. Значит температура в новой области нормального проводника понизится,
	значит начется процесс теплопроводности и сверхпроводник бюудет потихоньку
	уменьшаться. 

	Если мы второй раз продифференциируем уравнение для равновесия хим. потенциалов:
	\be
		- \pa_T s_S + \frac{1}{8\pi} \pa^2_{TT} H^2 = - \pa_T s_N;
	\ee

	Умножив на температуру получим теплоёмкости:
	\be
		c_S - c_N = \frac{T}{8\pi} \pa^2_{TT} H^2;
	\ee

	И тут мсы видим скачок теплоемкости - фазовый переход 2-ого рода.

	Термодинамика  - феноменологическая наука.

	\section{Уравнение Лондона}

	Оно описывает скинирование в сверхпроводнике.
	Напишем уравнение для модифицированной свободной энергии:
	\be
		\tilde F = F_{0s} (T,V,N) + \frac{H^2}{8\pi} V_s;
	\ee

	В таком случае у нас получится:
	\be
		\tilde F = F_{0s} - \int_{ V_{ex}} \frac{H^2}{8\pi} dV; 
	\ee
	\be
		F = F_{0s} - \int \frac{B^2}{8\pi} dV + \int n_s \frac{m v_s^2}{2} dV; 
	\ee

	При этом плостность тока:
	\be
		\vec j = e n_s \vec{v}_s = \frac{c}{4\pi} \rot \vec B;
	\ee

	Такмим образом получаем:
	\be
		F = F_{0s} + \int \frac{B^2}{8\pi} dV + \int \lambda^2 (\rot \vec B)^2 dV;
	\ee

	При этом $\lambda^2 = \frac{c^2 m}{4\pi e^2 n}$.
\end{document}
