%!TEX encoding = UTF-8 Unicode
\documentclass[a4paper, 14pt, russian]{article}
\usepackage[a4paper]{geometry}
\usepackage[T2A]{fontenc}
\usepackage[utf8]{inputenc}
\usepackage[russian]{babel}
\usepackage{physics}
\usepackage{hyperref}
\usepackage{fancyhdr}
\usepackage{indentfirst}
\usepackage{amssymb, amsmath}

\title{Теория Сверхпроводимости. Феноменологическая теория.}
\author{Курин}
\date{}

\newcommand{\be}{\begin{equation}}
\newcommand{\ee}{\end{equation}}
\newcommand{\bea}{\begin{eqnarray}}
\newcommand{\eea}{\end{eqnarray}}

\newcommand{\pa}{\partial}
\newcommand{\rot}{\textbf{rot}~}
\renewcommand{\div}{\textbf{div}~}
\renewcommand{\grad}{\textbf{grad}~}

\begin{document}
	\maketitle

	\section{Развитие термодинамического мышления}

	Мы рассматривали уравнение Лондонов. Мы написали:
	\be
		F = F_0(T,V) + \frac{1}{8\pi} \int \big(\vec{B}^2 + \lambda^2 (\rot \vec B)^2\big)dV;
	\ee

	При этом толщина скин-слоя:
	\be
		\lambda^{-2} = \frac{4\pi e^2 n_S}{m};
	\ee

	Тогда варьируя можем получить магнитное поле:
	\be
		\delta F = \int \frac{\delta F}{\delta \vec B} \delta \vec B dV;
	\ee

	В нашем случае получится:
	\be
		\delta F = \frac{1}{4\pi} \int \big(\vec B \delta \vec B + 
			\lambda^2 \rot \vec B \rot \delta \vec B\big) dV;
	\ee

	Можем воспользоваться свойством:
	\be
		\div [\vec a \times \vec B] = \vec B \rot \vec a + \vec a \rot \vec b;
	\ee

	В таком случае выйдет:
	\be
		\rot \vec B \rot \delta \vec B  = \delta B \rot \rot \vec B 
			- \div [\rot \vec B \times \delta \vec B ];
	\ee

	Отсюда получим вариационную производную:
	\be
		\frac{\delta F}{\delta \vec B} = \frac{1}{4\pi} 
			(\vec B + \lambda^2 \rot \rot \vec B) = 0;
	\ee

	Если введем лапласиан вектора, как:
	\be
				\rot \rot \vec B = \grad \div \vec B - \Delta \vec B;
			\ee

			Такой лапласиан удобен в декартовой системе координат.

			В силу $\div \vec B  = 0$ получим:
			\be
				\Delta \vec B - \frac{1}{\lambda^2} \vec B = 0;
			\ee

	Рассмотрим 3 задачи:
	\begin{itemize}
		\item Представим полупространство, наполовину заполненное сверхпроводником:
				%img001

			Тогда уравнение Лондонов превратится:
			\be
				\pa^2_{xx} B_z - \frac{1}{\lambda^2} B_z = 0;
			\ee 

			Тогда решение - спадающая экспонента.
			
			В СИ у нас уравнения:
			\begin{eqnarray}
				\rot H = j + \pa_t P;
				\rot E = - \pa_t B;
				B = \mu_0 H;
				D = \epsilon_0 \vec E;
			\end{eqnarray}

		\item  А если у нас сверхпроводящая пленка $-d/2,~d/2$ в вакууме.
				Для $B_z$ то же самое уравнение. Тогда:
				\be
					B_z = c_1 \cosh \frac{x}{\lambda} + c_2 \sinh \frac{x}{\lambda};
				\ee

				В силу четнорсти граничных условий:
				\be
					B_0 =c_1 \cosh \frac{d}{2\lambda};
				\ee 

				Отсюда:
				\be
					B_z = B_0 \frac{\cosh \frac{x}{\lambda}}{\cosh \frac{d}{2\lambda}};
				\ee


			\item А если про пленке течет ток?
				У нас будет скачок $2B_i = \frac{4\pi}{c} I$.

				Тогда поле в итоге:
				\be
					B = B_I \frac{\sinh \frac{x}{\lambda}}{\sinh \frac{d}{2\lambda}};
				\ee

	\end{itemize}


	\section{Динамический вывод уравнения Лондонов или двухжидкостная гидродинамика.}

		Можно представить что есть две жидкости - нормальная и аномальная. 
		Пусть есть 2-е концентрации $n_{N,S}, \vec{v}_{N,S}$. Тогда законы:
		\be
			\pa_t n_{N,S} + \div \vec{v}_{N,S} n_{N,S} = 0;
		\ee

		\be
			mn_{N,S}\big(\pa_t \vec{v}_{N,S} + (\vec{v}_{N,S},\nabla)\vec{v}_{N,S}\big) + \nabla p_{N,S} = 
				- e n_s \big(\vec E + \frac{1}{c} [\vec{v}_{N,S} \times \vec B]\big) + St;
		\ee

		Здесь столкновительный член зануляется для сверхпроводящих электронов6 а для обычных:
		\be
			St_{N} = mn_{N} v_N \nu;
		\ee

		Гидродинамическая скорость - усреднеена по объёму:
		\be
			\overline{v} = \frac{1}{V} \int \vec v d V;
		\ee

		Рассеяние в основном на дефектах решётки. Взаимное превращение может быть.
		Но если скорости малы, то можно пренебречь. 

		Здесь есть нелинейные члены. Но есть и член третьего порядка. Будем линеаризовать уравнения:
		\begin{eqnarray}
				n = n_0 + \delta n;\\
				\vec v = \delta \vec v;
		\end{eqnarray}

		В результате получим:
		\be
			\pa_t \delta n + n_0 \div \delta \vec v = 0;
		\ee

		А закон сохранения импульса:
		\be
			mn_0 \pa_t \delta \vec v + \pa_n p \delta n = - e n \vec E - St;
		\ee

		Тогда получим:
		\be
			\pa_t v_n + c_n^2 \grad n_n = - \frac{e}{m} \vec E -  \nu \vec{v}_n;
		\ee
		\be
			\pa_t v_s + c_s^2 \grad n_s = - \frac{e}{m} \vec E;
		\ee

		Здесь есть 2 типа движения: продольное и поперечное. Потенциальное и вихревое.

		Рассмотрим случай падения S (TE) поляризации. У нас есть только $E_y(z,x)$. 
		Его дивергенция равна нулю. Значит это вихревое поле. 
\end{document}
