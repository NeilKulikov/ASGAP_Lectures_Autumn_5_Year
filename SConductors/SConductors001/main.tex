%!TEX encoding = UTF-8 Unicode
\documentclass[a4paper, 14pt, russian]{article}
\usepackage[a4paper]{geometry}
\usepackage[T2A]{fontenc}
\usepackage[utf8]{inputenc}
\usepackage[russian]{babel}
\usepackage{physics}
\usepackage{hyperref}
\usepackage{fancyhdr}
\usepackage{indentfirst}
\usepackage{amssymb, amsmath}

\title{Теория Сверхпроводимости. Феноменологическая теория.}
\author{Курин}
\date{}

\newcommand{\be}{\begin{equation}}
\newcommand{\ee}{\end{equation}}
\newcommand{\bea}{\begin{eqnarray}}
\newcommand{\eea}{\end{eqnarray}}

\newcommand{\pa}{\partial}
\newcommand{\rot}{\textbf{rot}~}
\renewcommand{\div}{\textbf{div}~}
\renewcommand{\grad}{\textbf{grad}~}

\begin{document}
	\maketitle

	Есть еще микроскопическая теория сверхпроводимости - она более полная, но и сложная.
	Будет во встором семестре.

	\section{Литература}

	Зачёт будет похож на экзамен. Есть учебное пособие: "Феноменологическая теория сверхпроводимости".
	Можно найти его на сайте кафедры наноструктур: https://ipmras.ru/ $\rightarrow$ кафедра наноструктур 
	и наноэлектроники $\rightarrow$ учебное пособие.

	Ещё из литературы:
	\begin{itemize}
		\item Абрикосов "Теория нормальных металлов и сверхпроводников"
		\item Шмидт "Введение в сверхпроводимость"
		\item Типкхам
		\item Де Жен
		\item Ландау, Лифшиц Т. IX
		\item Фейнман "Статистическая механика"
	\end{itemize}

	Сверхпроводимость - явяление отсутствия сопротивления металлов при некоторой температуре.

	%img001

	Сопротивление это $R = U / I$. И оно как правило обусловлено рассеянием на фононах 
	кристаллической решётки. 

	%img002

	И ток течен без приложения напряжения бесконечно долго. Это открытие было совершено
	в 1911 году голландским физиком Кммерлингом-Оннсе. Он достиг температуры жидкого 
	$He^\text{4}$ и обнаружил, что в четырёхзондовой схеме достигается нулевое сопротивление.

	%img003

	А выглядело это как трубка с ртутью в жидком гелии: 

	%img004

	И это оказалось характерным для плохих полупроводников: Hg, Nb; а 
	для хороших, типа Au, Ag, Cu  - не наблюдается. И для сплавов был
	рекорд порядка 23K.

	Можно нарисовать историю сверхпроводников как график критической 
	температуры от времени:

	%img005

	Была открыта высокотемпературная сверхпроводимоть сотрудниками IBM:
	Мюллером и Беднордом. Первый состав $La_{2-x} Sr_x Cu O_4$ имел
	$T_c \approx 40\text{K}$. Потом был состав $YBaCuO$ с критической 
	температурой 90K, $BiSrCaCuO$ с температурой 90-110K. Они так хороши,
	потому что их критическая температура превосходит температуру 
	жидкого азота. А следующим этапом стали железные сверхпроводники 
	$\hdots FeAs$ с критической температурой порядка 55K.

	Была теория Гинзбурга-Ландау, которая появилась в 30-х годах. 
	А понимание (хотя бы качественное) появилось только в 1957 году
	- теория БКШ. Потом был открыт эффект Джозефсона в 1962 году.

	А ещё была теория Абрикосова-Гинзбурга-Легетта в 2003 году. 
	Эти явления тесно связаны со сверхтекучестью. 

	\section{Основные экспериментальные факты и их следствия.}
	\subsection{Электродинамика сверхпроводников.}

	Наблюдайется \paragraph{эффект Мейсснера-Оксенфельда}.
	Происходит выталкивание магнитного поля из сверхпроводника:

	%img006

	Вспомним уравнения Максвелла:
	\bea
		\div \vec B = 0;\\
		\div \vec D = 4 \pi \rho;\\
		\rot \vec E = - \pa_t \vec B / c;\\
		\rot \vec H = \pa_t \vec D + \frac{4\pi}{c} \vec j;
	\eea

	Все эти уравнения писались в аналогии с деформациями твёрдого тела.
	\be
		u_{ik} = \Lambda_{iklm} \sigma_{lm};
	\ee

	В даннном случае $D$ - от слова displacement - смещение. Но сейчас 
	мы все знаем, что есть микротеория всего этого. Будем рассматривать 
	уравнение Шрёдингера:
	\be
		\hat H \psi = \big(- \frac{\hbar^2}{2m} \Delta  + \frac{\alpha}{r}\big) \psi = E \psi;
	\ee

	А не падает он на центр за счет соотношения неопределённости $\Delta x \Delta p \sim \hbar$.
	Размер ядра порядка $10^{-13}~cm$, а радиус Бора $10^{-8}~cm$. А между ядром и электроном - поле.
	
	\subsection{Двухвекторная форма уравнений Максвелла}

	На самом деле его сейчас пишут так:
	\bea
		\rot \vec B = \frac{4\pi}{c} \vec j + \frac{1}{c} \pa_t \vec E;\\
		\rot \vec E = - \frac{1}{c} \pa_t \vec B;\\
		\div \vec E = 4 \pi \rho;\\
		\div \vec B = 0;
	\eea

	Тогда получится соотношение типа:
	\bea
		\vec B = \langle \vec{b}_{mic} \rangle;\\
		\vec E = \langle \vec{e}_{mic} \rangle;
	\eea

	А всякик сложные вектора получаются из усреднения микрополей. 
	Тогда и токи можно написать по-другому:
	\be
		\vec j = \pa_t \vec P + c \rot \vec M + \vec{j}_{ex};
	\ee

	И бывают микротоки. Они в квантовой механике пишутся как:
	\be
		\vec j = - e \frac{e\hbar}{2m} (\psi^{*} \nabla \psi _ c.c);
	\ee

	И тогда мы можем написать:
	\be
		\rot \vec B = \frac{4\pi}{c} (\pa_t \vec P + c \rot \vec M + \vec{j}_{ex}) 
			+ \frac{1}{c} \pa_t \vec E;
	\ee

	А материальные соотношения:
	\bea
		\vec D = \vec E + 4 \pi \vec P;\\
		\vec H = \vec B - 4 \pi \vec M;
	\eea

	В идеальной плазме поле оказывается вмороженным. А здесь оно выталкивается
	$B_i = 0$.
	Это квантовая плазма по-сути. А какие из этого следствия?

	\begin{itemize}
		\item $B_{n,ex} = 0$ - поле не входит в сверхпроводник. 
			Это следствие $\div \vec B = 0$.
		\item $\vec{B}_{\tau,ex} - \vec{B}_{\tau,in} = \frac{4\pi}{c} \vec{j}_\tau$
	\end{itemize}

	%img007
\end{document}
