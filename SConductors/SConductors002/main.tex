%!TEX encoding = UTF-8 Unicode
\documentclass[a4paper, 14pt, russian]{article}
\usepackage[a4paper]{geometry}
\usepackage[T2A]{fontenc}
\usepackage[utf8]{inputenc}
\usepackage[russian]{babel}
\usepackage{physics}
\usepackage{hyperref}
\usepackage{fancyhdr}
\usepackage{indentfirst}
\usepackage{amssymb, amsmath}

\title{Теория Сверхпроводимости. Феноменологическая теория.}
\author{Курин}
\date{}

\newcommand{\be}{\begin{equation}}
\newcommand{\ee}{\end{equation}}
\newcommand{\bea}{\begin{eqnarray}}
\newcommand{\eea}{\end{eqnarray}}

\newcommand{\pa}{\partial}
\newcommand{\rot}{\textbf{rot}~}
\renewcommand{\div}{\textbf{div}~}
\renewcommand{\grad}{\textbf{grad}~}

\begin{document}
	\maketitle

	

	\section{Элементарная электродинамика}

	Пусть есть сверхпроводник и есть два магнитных поля 
	$\vec{B}_e,~\vec{B}_i$ - внешнее и внутренее. Эффект 
	Мейснера заключается в том, что $\vec{B}_i = 0$.
	
	А ещё этот эффект наблюдается только в определёънной 
	области $B,T$:

	%img001

	Если мы применим теорему Гаусса:
	\be
		\div \vec B = 0 \rightarrow \vec{B}_{i,n} = \vec{B}_{e,n} = 0;
	\ee

	%img002

	Из этого следует, что есть поверхностный ток:
	\be
		j_y = \frac{c}{4\pi} B_{e,x};
	\ee

	А в векторной записи:
	\be
		\vec{j}_{sur} = \frac{c}{4\pi} [\vec n \times \vec{B}_e];
	\ee

	А ток объёмный связан с этим:
	\be
		\vec{j}_{sur} = \int_{-\infty}^{0} \vec j  dz;
	\ee

	То есть у нас есть скин эффект, даже при нулевой частоте, чего 
	в нормальных проводниках нет. Только называется это глубиной проникновения:
	\be
		\lambda \sim 10 \hdots 200~nm;
	\ee

	Обычно это считают $\delta$ функцией. Примерно это выглядит так:
	%img003

	Почему у нас магнитное поле спадает так же как и ток? Потому что 
	уравнения Максвелла:
	\be
		\rot \vec B = \frac{4\pi}{c} \vec j;
	\ee

	Что у нас получается? 
	\be
		\pa_z B_x = \frac{4\pi}{c} \vec j;
	\ee

	Но ещё у нас есть сила Лоренца:
	\be
		\vec f = \frac{1}{c} [\vec j \times \vec B];
	\ee

	В нашем случае это превратится в :
	\be
		f_z = - \frac{1}{4\pi} \pa_z B_x \cdot B_x
	\ee

	Если мы проинтегрируем силу, то получим давление:
	\be
		\vec{f}_{sur} = - \vec n \frac{B_e^2}{8\pi};
	\ee

	Какие законы сохранения?
	\be
		\pa_t \rho + \div \vec j = 0;
	\ee
	\be
		\pa_t W + \div \vec S = 0;
	\ee

	- Заряда и Энергии. Но можно написать и для импульса
	при помощи тензора натяжений:
	\be
		\pa_t \vec p + \div \hat T = 0;
	\ee

	А сам тензор натяжений:
	\be
		T_{ik} = \frac{1}{8\pi} (B_i B_k - \frac{1}{2} S_{ik} B^2);
	\ee

	Как обычно проводится эксперимент?
	%img004

	Напишем уравнение максвелла. но учтем, что есть
	2 тока - внешний и скин-тое. Тогда:
	\be
		\rot \vec B = \frac{4\pi}{c} \vec j = \frac{4\pi}{c}(\vec{j}_e + \vec{j}_i)'
	\ee

	Но тогда через намагниченность $\vec H = \vec B  - 4\pi \vec M$:
	\be
		\vec{j}_i = c \rot \vec M;
	\ee

	При этом стоит заметить, что:
	\be
		\div \vec H = - 4\pi \div \vec M \neq 0;
	\ee

	А вот вне образца у нас потенциальное магнитное поле
	в силу того, что $\rot \vec H  = 0$:
	\be
		\vec H = - \grad \psi;
	\ee

	%img005

	У нас сверхпроводник - идеальный диамагнетик - $\mu = 0$.
	Отсюда у нас условие:
	\bea
		\Delta \Psi = 0;\\
		\psi_e\rvert_s = \psi_i \rvert_s;\\
		\mu_e \pa_n \psi_e = \mu_i \psi_i = 0;
	\eea

	Тогда тангенциальная компонента поля на границе непрерывна.
	%img007

	Тогда у нас получается что-то вроде магнитного заряда.
	Его поле будет:
	\be
		\vec B = \vec{r} \frac{q}{r^3};
	\ee

	Есть еще 2 задачи:
	%img008

	\section{Термодинамика сверхпроводников}
	
	Здесь вводится понятие энтропии $S$, которая
	увеличивается $\delta S \geq 0$. Как с этим работать? 
	Пусть у нас есть коробка с двумя газами:
	%img009
	
	А как себя будет вести энтропия? 
	$S_1(E_1,V_1,N_1),~S_2(E_2,V_2,N_2)$ И силы между газами
	должны быть короткодействующими. В силу аддитивности:
	\bea
		S = S_1 + S_2;\\
		E = E_1 + E_2;
	\eea

	Но тогда получим:
	\bea
		\pa_{E_1} S = \pa_{E_1} S_1 + \pa_{E_1} S_2 = \pa_{E_1} S_1 + \pa_{E_2} S_2 = 0;
	\eea

	Но в силу $\delta Q = T \delta S$ получим:
	\be
		\pa_{E_1} S = \frac{1}{T_1} - \frac{1}{T_2} = 0;
	\ee

	Отсюда следует равенство температур. Из соотношения 
	$\delta E = T \delta S - p\delta V$. Тогда механическое 
	равновесие:
	\be
		\pa_{V_1}  S = \frac{p_1}{T_1}  - \frac{p_2}{T_2} = 0;
	\ee

	Аналогично химическое равновесие:
	\be
		\pa_{N_1}  S = \frac{\mu_1}{T_1}  - \frac{\mu_2}{T_2} = 0;
	\ee

	Можно менять перменнные в потенциалах при помощи 
	преобразований лежандра:
	\be
		\delta E = T \delta S - p \delta V + S \delta T - S \delta T; 
	\ee

	Тогда переходим к свободной энергии:
	\be
		\delta F  = \delta(E - TS) = - S \delta T - p \delta V;
	\ee

\end{document}
